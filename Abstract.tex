\documentclass[a4paper,11pt]{article}


\usepackage{natbib}
\usepackage{enumerate}
\usepackage[osf]{mathpazo}
\usepackage{lastpage} 
\pagenumbering{arabic}
\linespread{1}

\begin{document}

\begin{center}

%Title
\noindent{\Large{\bf{Macroevolution with living and fossil species}}}\\
\bigskip
%Author
\noindent{Thomas Guillerme}\\

\end{center}
%\section{Abstract}
Although many biodiversity studies focus on living species, the vast majority of species that ever lived are long extinct.
It is crucial to combine data from both living and fossil species to fully understand macroevolutionary patterns and processes.
This thesis focuses on ways to combine both living and fossil taxa into phylogenies and investigates how the resulting phylogenies can be used to investigate macroevolutionary questions.

In the first part of the thesis, I ran extensive simulation analyses to test the effect of missing data on phylogenetic topologies when using the Total Evidence method.
This method builds phylogenies using both molecular data for living taxa and morphological data for living and fossil taxa.
I tested how various proportions of missing morphological data among living taxa, fossil taxa, and the two combined, affected my ability to recover the correct tree topology.
I found that the amount of missing morphological data among living taxa was the most crucial aspect for accurately placing living and fossil taxa in the same phylogeny.
Following these conclusions, I recorded the amount of morphological data available for each mammalian order and tested whether this data was randomly distributed across the phylogeny or biased towards certain clades.
The results of this analysis showed that although morphological data is scarce for living mammals, it is at least generally randomly distributed across the phylogeny and therefore should not bias the placement of fossil taxa towards particular clades.

For the second part of the thesis, I used Total Evidence phylogenies to investigate whether mammals radiated during the Cenozoic in response to the infamous Cretaceous-Palaeogene (K-Pg) mass extinction event, 66 million years ago.
Previous studies show support for an effect of the K-Pg extinction event on mammalian diversification when using palaeontological data but no support using neontological data.
I used a novel time-slicing method for quantifying changes in morphological diversity (disparity) through time to describe the patterns of mammalian diversification across the K-Pg boundary.
I found no significant difference in disparity before and after the K-Pg boundary.
This suggests that, even though many terrestrial vertebrates (including the non-avian dinosaurs) went extinct during the K-Pg extinction event, it had no significant effect on mammalian morphological diversification.

Finally, I discuss future avenues of research for improving analyses that include living and fossil species as well as the advantages of using both living and fossil taxa when investigating macroevolutionary questions. 
I argue that all macroevolutionary studies should include both types of data to advance our understanding of biodiversity.

\end{document}
