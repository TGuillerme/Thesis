\chapter{Discussion}
\label{chap:discussion}

%---------------------
%
% DISCUSSION - MAKE IT SHORT TO! - TG: this version is a bit long I think, I'll shorten it up (biology letters style) after first round editing. 
% 
%---------------------

In the following chapter, I first discuss the implications of the results of the chapters \ref{chap:TEM_paper} and \ref{chap:missing_mammals} and then discuss the results of chapter \ref{chap:STD_paper}.
In both section, I discuss the several methodological caveats and propose future research avenues to solve these problems.
Finally, I discuss the overall importance of combining both living and fossil species into macroevolutionary studies.

\section{The future of the Total Evidence method}
The Total Evidence method seems to be one of the promising new ways of testing macroevolutionary hypotheses \citep[e.g.][]{ronquista2012,slaterphylogenetic2013,Wood01032013,beckancient2014,Dembo2015}.
However, as shown in the chapters \ref{chap:TEM_paper} and \ref{chap:missing_mammals} of the present thesis, this method seems to be sensitive to missing data.
As discussed in both chapters, increasing the number of morphological characters and the number of living taxa with coded morphological characters seems the most efficient way to improve the phylogenetic signal.
This data is vastly available and easy to access in Natural History collections worldwide and software infrastructure have been developed for facilitating collaborative data collection \citep{morphobank}.
Therefore, one can hope that the missing data problem among living species will be gradually solved with time.
Thus, in the following section, I will focus on the general problems with Total Evidence phylogenies that where not developed in the discussions of chapters \ref{chap:TEM_paper} and \ref{chap:missing_mammals}. 

Because these phylogenies contain both living and fossil taxa, the preferred way to date such trees is to use the tip-dating method \citep[e.g.][]{ronquista2012,Wood01032013,Dembo2015}.
This method relies on the age of the fossil taxa (treated as tips) to date the nodes divergence time rather than defining node calibrations \textit{a priori} \citep[cf. node-dating;][]{ronquista2012}.
This method has two main advantages: (1) it has been observed from empirical data that it improves the ability to recover the correct topology because it can use the stratigraphic age of the fossil taxa to favor some topological solution upon others \citep[typically by minimizing implied ghost lineages;][and personal communications]{BEASTmaster}; and (2) it reduces the confidence intervals at the node ages compared to a classic node-dating approach \citep{ronquista2012}.
This second point, however, has been revised by \cite{Arcila2015131} by comparing both method using the latests models for the node-dating method \citep[i.e. using the fossilised birth-death model;][]{heaththe2013} and showing the opposite effect (i.e. an increase in node age confidence interval with the tip-dating method).
It would therefore be interesting to run a similar analysis than in chapter \ref{chap:TEM_paper} but adding a dating aspect to it.
By comparing dated Total-Evidence matrices using both node-dating and tip-dating, one could formally test the two advantages outlined above as well as their resilience to missing data.

Additionally, a more general problem, is that the Total Evidence method relies on the M\textit{k} model \citep{lewisa2001} to measure the morphological distance between taxa.
This method is a generalisation of the Jukes-Cantor evolutionary model \citep[JC69;][]{jukes1969evolution} that allows a single mutation rate $\mu$ between all character states.
The JC69 model is a simplification of reality and was replaced with more generalised models closer to biological reality \citep[e.g. the GTR model allowing a different rate for each mutation;][]{tavare1986}.
It is therefore likely that the M\textit{k} model is also a crude underestimation of the reality of morphological evolution, especially since the assumption that their is a unique transition rate between character states has been shown to be wrong in at least some specific cases \citep[e.g. for Dollo traits that are traits that have been observed to evolve from only one state \textit{a} to \textit{b} but never from \textit{b} to \textit{a};][]{WrightDollo}.
\cite{spencerefficacy2013} even demonstrated that non-probabilistic method such as maximum parsimony outperforms the M\textit{k} model regarding the contentious placement of fossils such as \textit{Archaeopteryx}.
However, more recent and thorough simulations have demonstrated the opposite, even if it underestimates the reality of morphological characters evolution \citep{wrightbayesian2014}.
As the statistician George Box wrote, ``essentially, all models are wrong, but some are useful'' \citep{box1987empirical}.
This can be typically the case for the Total Evidence method: despite the three major caveats discussed above (missing data, dating, and morphological evolution), this methods remains the only efficient method to date to include the diversity of life both past and present.

One way to improve the Total Evidence method could be a \textit{Full} Total Evidence method.
In fact, the Total Evidence methods claims to be total because it uses both molecular and morphological data \citep{eernissetaxonomic1993}, however, this does not represents the \textit{totality} of data available to biologists.
Other sources of data such as traits (e.g. body mass), ecology (e.g. habitat) or biogeography could also be realistically added to Total Evidence methods with appropriate evolutionary models and hypothesis for each type of data \citep[e.g. respectively quantitative, multiple or geographic state speciation and extinction model -- Qua-Mu-GeoSSE models;][]{fitzjohndiversitree2012}.
However, such data sets could improve the Total Evidence trees but also make them more complex statistical by increasing the number of parameters and assumptions.
Finally, this is also likely to simply increase the data availability problem.

\section{Diversity is multidimensional}
% link to previous paragraph
One important point to keep in mind however, is that phylogenetic trees (whether they use all the available data or not) are merely tools for observing evolutionary patterns and testing evolutionary hypotheses.
For example, in chapter \ref{chap:STD_paper}, I use two independent tip-dated Total Evidence trees to test whether mass extinctions can influence surviving clade's morphological evolution.
I argue that, in such studies, the use of Total Evidence tree improves the timing of diversification events \citep[][ which is a crucial aspect when studying effect of mass extinctions which are finites points in time]{ronquista2012} or the estimation of morphological diversity \citep[increasing accuracy in reconstructing node's ancestral characters;][]{Finarelli2006}.
However, in this particular example, I used disparity (i.e. morphological -- or rather cladistic -- diversity) as a proxy for testing the effect of the K-Pg extinction.
Even though disparity analysis are becoming increasingly common in palaeobiology \citep[e.g.]{Butler2012,brusattedinosaur2012,toljagictriassic-jurassic2013,brusattegradual2014,bensonfaunal2014,Claddis,Close2015}, they still suffer from several biases.

%Morphological diversity
Firstly, morphological diversity is a complex concept to grasp or to interpret.
Describing the shape of an organism is not straightforward and many mathematical methods exist (e.g. Elliptic Fourier; \citealt{Fourier1982}; Procrustes; \citealt{JamesRohlf1993129}; Convex Hull; \citealt{ANDREW1979216}).
In biology, one major approach is to describe shape as a summary of an ordinated distance matrix based on procrustes \citep[i.e. a geometric morphometric approach][]{zelditch2012geometric}.
In studies using this approach, shape is approximated by actual continuous measurement collected from the organisms \citep[e.g.][]{friedmanexplosive2010,hopkinsdecoupling2013,finlay2015morphological}.
However, in our case, we used differences (read inter-taxon distances) between particular morphological features \citep[e.g.][]{foote1997evolution,Wills2001,Wesley-Hunt2005}.
This method has been criticised by some of their users to be biased by: (1) the fact that these morphological features are not randomly collected and can distort reality by emphasising differences in the taxonomic group of interest \citep{Hopkins24032015} or (2) that they are highly dependant on the quality of the fossil record \citep{Butler2012}.
However, these biases are overweighted by the advantages of (1) having many comparable morphological data among taxa \citep{Brusatte12092008} and by (2) the possibility of correcting for the fossil record quality through time \citep{Butler2012}.
Additionally, it has been shown that even though morphometric based and cladistic based disparity are different, they seem to capture the same signal \citep{foth2012different,hetherington2015cladistic}.

%Finally, which metric?
Secondly, disparity is an abstraction of morphological diversity: it is an unique value that describes and multidimensional transformation of an actual shape \citep{Wills1994,foote1997evolution}.
This can certainly lead to problems in the interpretation of such a value since each step have it's own caveats and limitations (i.e. describing the shape of an organism using morphometrics or cladistics and mathematically transforming this description into a matrix).
Classically people have used the four metrics proposed by \cite{Wills1994} (sum and product of variance and range) but several problems have never been explored.
\begin{enumerate}
\item firstly, additionally to the practical problems discussed in chapter \ref{chap:STD_paper} the present software implementations for calculating the sum and product of variance never integrate the covariance present in the ordinated matrix.
\item secondly, even though some attempts have been made for measuring the efficiency of these metrics \citep{Ciampaglio2001} there have been yet no global assessment of the statistical power of each metric for describing multidimensional space occupancy.
\item finally, these metrics are only describing the $n$ dimensions (i.e. the columns in the matrix) but are not directly describing the placement of the tips or nodes in the $n$ dimensions (cf. the distance between taxa and the centroid).
\end{enumerate}
Future developments of disparity through time studies would require a better understanding of the statistical performance of these disparity metrics and how each of them would be more appropriate to specific empirical situations.

%Multidimensionality
Finally, the exciting results from the latest disparity through time studies underline the importance of studying the multidimensionality of biodiversity \citep[cf. just taxonomic richness;][]{Butler2012,brusattedinosaur2012,toljagictriassic-jurassic2013,brusattegradual2014,bensonfaunal2014,Claddis,Close2015}.
It also encouraging to note that this is not only a palaeobiological approach to describing biodiversity but is also trending in other disciplines such as ecology \citep{DonohueDim}.
In fact, biodiversity is the combination of taxonomic diversity \citep[e.g.][]{Stadler12042011}, morphological diversity \citep[from cladistics or morphometrics;][]{hetherington2015cladistic} and phylogenetic diversity \citep[e.g. the evolutionary rates regimes;][]{Close2015}.
However, similarly to the comment on the \textit{Full} Total Evidence method above, this multidimensionality could also include biogeographical or ecological diversity.
Such analysis could lead to a better understanding of macroevolutionary patterns and could allow us to test more general evolutionary hypothesis such as the validity of the concept of ecological niches \citep{pearmanniche2008}.

\section{What is the real effect of combining?}
This whole thesis tackles practical and theoretical aspects of using both living and fossil species in macroevolutionary studies.
Several studies have already demonstrated the challenges and the importance of including fossils into phylogenies \citep[e.g.][]{ronquista2012,slaterphylogenetic2013,Wood01032013,beckancient2014,Dembo2015}.
However, all these studies (including the present thesis) do not focus on the effect of adding fossil taxa to phylogenies \textit{per se} but rather perform empirical or theoretical analysis while including fossil taxa and demonstrate the superiority of their findings upon previous studies.
In fact, even though there is a strong consensus on the importance of such analysis \citep{jacksonwhat2006,quentaldiversity2010,dietlconservation2011,slaterunifying2013,fritzdiversity2013,benton2015}, the effect of combining both living and fossil has yet, to my knowledge, never been tested in a theoretical way.
This might be due to the difficulties to propose a generalised theoretical framework on which to test the effect of combing living and fossil species.
Yet, it is important to note that this thesis, along with several other studies, actually investigated this effect on some empirical data sets and consistently found an important effect of adding fossil data in macroevolutionary studies \citep{Finarelli2006,Slateretal2012,slaterphylogenetic2013,SlaterPennel2014,pant2014complex,Mitchell2015}.

One question arising from these studies is whether there is a \textit{real} effect of combining both living and fossil species into macroevolutionary studies.
In fact, one can argue that the conclusions from these studies are linked to the peculiarity of the groups studied (or the simulation protocol) that displayed a rather dynamic evolution that can only be revealed by combining all available data.
Because the methods for combining living and fossil species are still challenging, it could be a futile and time consuming exercise in some scenarios such as: (1) when studying some clades that have no living relatives (e.g. Trilobita; \citealt{hopkinsdecoupling2013}; Pterosauria; \citealt{Butler2012}; etc.); (2) when studying clades with a really poor fossil record \citep[e.g. Aves where there are three orders of magnitude more known living than fossil taxa;][]{jetzthe2012,Mitchell2015}; (3) or when studying clades that have undergone a recent radiation \citep[e.g. Cichlidae][]{Genner01052007}.

Ironically, however, each of these three scenarios can also be used to demonstrate the importance of combining living and fossil taxa into macroevolutionary studies:
\begin{enumerate}
\item counter intuitively, combining clades with no living relatives might be really important for understanding macroevolutionary patterns in living taxa. For example morphological study of long extinct Ostracoderma (armoured jawless fishes) can help understanding characters evolution in later Gnatostoma \citep[jawed vertebrates;][]{Janvier2015}.
\item in fossil poor clades, the few available fossils can actually bring precious information on the early history of the group. For example, in Aves, disparity is underestimated when ignoring fossils, even if there are only a handful of fossils available \citep[e.g. 58 fossil genera against 604 living ones;][]{Mitchell2015}.
\item finally, excluding fossils from recent clades might also be detrimental to macroevolutionary interpretations, for example, in Lemuroidea, some sub-fossils species had a body mass several orders of magnitudes bigger than all living lemurs \citep{hartwig2002primate,Jungers2008} and only went extinct at latest around 600 years ago \citep{goodman2003introduction}.
\end{enumerate}

Therefore I argue that there is no biological justification to not jointly using living and fossil species since the effect of combining them can not be known \textit{a priori}.
Our knowledge in biology has tremendously advanced since the last half century ranging from the amazing revelation of the few glimpses of the deep past provided by the the fossil record to the understanding of the complexity and dynamics of modern ecosystems.
Because the inherent characteristic of the deep past is to be unknown and mysterious, it is therefore crucial to incorporate all of this knowledge in macroevolutionary studies to continue revealing the ``grandeur [of] this view of life'' \citep{darwin}.

%\bibliography{References}