\chapter{Discussion}
\label{chap:discussion}

%---------------------
%
% DISCUSSION - MAKE IT SHORT TO!
% 
%---------------------

\section{The future of the Total evidence method}
Combined with tip-dating is super interesting but we need more data.
To do so we can use plateforms such as morphobank and foster collaboration on big projects.
Also we can make all the data available blablabla.

However, there are some limitations:
Maybe tip-dating isn't that good? Compared to the nice recent node dating models... (Arcila)
Also the Mk model is really crude and overly simplistic.

One way to improve could be a REAL total evidence dating using also trait data, biogeography, etc...
In reality, all this parameters have an influence of lineages history and should technically be taken into account.
But data problem is likely to increase, an needs models need to be improved as well.
And in the end, how many parameters do we want?

\section{Diversity is multidimensional}
Diversity is often just seen as the sheer number of species.
However, the processes that led to this pattern is fundamentaly intangled with all the other aspects of diversity.
For example, specious rich groups have also so traits, etc...
It is important to disentangle.
But other dimensions as well: Ecological, life history, etc.
We need to take into account more of these "disparity" patterns to really understand what happened.
Especially when combining living and fossil, species richness is a really poor indicator of diversity.

However, this is more complex, species diversity is easy to interprate (many populations isolations through time) but disparity is a bit harder.
What IS disparity? What metric to use? How to express the changes etc...
Also, all these metrics are just using proxies.

But  The statistician George Box wrote "essentially, all models are wrong, but some are useful" \citep{box1987empirical}. % Modify
This is still really promising and can be improved first by underestanding how all this works in a theoretical way (building the models).
And only then apply it to observed patterns.

\section{What is the real effect of combining?}
Maybe only important when groups have actually a complex history?
Old clades might have no living descendants and the question is therefore N/A
Recent subclades maybe not have changed much in diversity so adding fossils might not change much.
But we never know! Example of the giant lemur (recently extinct).
