\chapter{Discussion}
\label{chap:discussion}


\section{The future of the Total evidence method}
Combined with tip-dating is super interesting but:
-Data limitations
-Problems with dating (Arcila)
-Better models for morphology?
One way could be a REAL total evidence dating using also trait data, biogeography, etc...
In reality, all this parameters have an influence of lineages history and should technically be taken into account.
But data problem is likely to increase, an needs models need to be improved as well.
And in the end, how many parameters do we want?

\section{Diversity is multidimensional}
It is important to disentangle
But other dimensions as well: Ecological, life history, etc.

\section{What is the real effect of combining?}
Maybe only important when groups have actually a complex history?
Old clades might have no living descendants and the question is therefore N/A
Recent subclades maybe not have changed much in diversity so adding fossils might not change much.
But we never know! Example of the giant lemur (recently extinct).

Future directions
-what should we do on the thesis to continue

Caveats
-the hard ones (I know this is broken)
-phylogenies are never perfect
-using proxies