\chapter{Discussion}
\label{chap:discussion}

%---------------------
%
% DISCUSSION - MAKE IT SHORT TO! - TG: this version is a bit long I think, I'll shorten it up (biology letters style) after first round editing. 
% 
%---------------------

\section{The future of the Total Evidence method}
The Total Evidence method seems to be one of the promising new ways of testing macroevolutionary hypotheses \citep[e.g.][]{ronquista2012,slaterphylogenetic2013,Wood01032013,beckancient2014,Dembo2015}.
However, as shown in the chapters 2 and 3 of the present thesis, this method seems to be sensitive to missing data \citep{GuillermeCooper,MissingMammals}.
As discussed in chapter two, the missing data can be separated in two categories: (1) the number of morphological characters and the number of living taxa with coded morphological characters and (2) the amount of missing data in the fossil record.
The first point is merely a technical problem since the infrastructures for coding large amount of characters exists \citep[e.g.][]{morphobank,O'Leary08022013} and the living taxa are vastly available in Natural History collection around the word.
Therefore, we can be confident that this issue will be solved with time.
However, the second point is more problematic since it solely depends on the stochasticity new palaeontological discoveries.

Some other limitations include the associated method for dating Total Evidence phylogenies.
Because these phylogenies contain both living and fossil taxa, the preferred method to date such trees is to use the tip-dating method \citep[e.g.][]{ronquista2012,Wood01032013,Dembo2015}.
\cite{ronquista2012} have demonstrated the advantage of the tip-dating on the node-dating method since it reduces the confidence intervals at the node ages compared to a classic node-dating approach.
However, more recently, \cite{Arcila2015131} have revised this claim by comparing both method using the latests models for the node-dating method \citep[i.e. the fossilised birth-death model][]{heaththe2013} and have shown the opposite effect.
These differences can be due to both the empirical approaches from both studies or the differences in node-dating models.
One solution to solve this issue would be a similar thorough simulation approach presented in chapter two for testing the effect of node-dating or tip-dating on divergence time estimations.

Additionally, the Total Evidence method relies on the M\textit{k} model \citep{lewisa2001} to measure the morphological distance between taxa.
This method is a generalisation of the Jukes-Cantor evolutionary model \citep{jukes1969evolution} that allows a single mutation rate $\mu$ between all four nucleotides.
This model of DNA evolution is a crude simplification of reality and was replaced with more generalised models closer to biological reality \citep[e.g. the GTR model allowing a different rate for each mutation;][]{tavare1986}.
It is therefore likely that the M\textit{k} model is a crude underestimation of the reality of morphological evolution, especially since the assumption that their is a unique rate of evolution between character states has been shown to be wrong in some specific cases \citep[e.g. for Dollo traits;][]{WrightDollo}.
\cite{spencerefficacy2013} even demonstrated that non-probabilistic method such as maximum parsimony outperforms the M\textit{k} model regarding the contentious placement of fossils such as \textit{Archaeopteryx}.
Yet, more recent simulations have clearly demonstrated the superiority of the M\textit{k} model over maximum parsimony at estimating correct topology, even if it underestimates the reality of morphological characters evolution \citep{wrightbayesian2014}.

One alternative to the Total Evidence method, the classic node-dating method, can therefore been seen as interesting solution to overcome this data problem.
This method has recently benefited from excellent incremental implementation such as the fossilised birth-death model \citep{heaththe2013} or standardisation for choosing calibration points \citep{Parham01032012}.
Yet, as the statistician George Box wrote, ``essentially, all models are wrong, but some are useful'' \citep{box1987empirical}.
I argue that this is exactly the case of the Total Evidence method: despite the three major caveats discussed above, this methods remains the only efficient method to date to include the diversity of life both past and present.

One research route to improve the Total Evidence method could be a \textit{Full} Total Evidence method.
In fact, the Total Evidence methods claims to be total because it uses both molecular and morphological data \citep{eernissetaxonomic1993}, however, this does not represents the \textit{totality} of data available to biologists.
Other sources of data such as traits (e.g. body mass), ecology (e.g. habitat) or biogeography could also be realistically added to Total Evidence methods with appropriate evolutionary models and hypothesis for each type of data \citep[e.g. respectively quantitative, multiple or geographic state speciation and extinction model - Qua-Mu-GeoSSE models;][]{fitzjohndiversitree2012}.
However, such data sets could improve the Total Evidence trees but also make them more complex statistical by increasing the number of parameters and assumptions.
Finally, this is also likely to simply increase the data availability problem.

\section{Diversity is multidimensional}
% link to previous paragraph
One important point however, is that phylogenetic trees (whether they use the Total Evidence method or not) are mainly tools for observing evolutionary patterns and proposing evolutionary hypotheses.
For example, in chapter 4, I use two independent tip-dated Total Evidence trees to test whether mass extinctions can influence surviving clade's morphological evolution.
I argue that the use of Total Evidence tree not only improves the timing of diversification events \citep[][; which is a crucial aspect when studying effect of mass extinctions which are finites points in time]{ronquista2012} or the estimation of morphological diversity \citep[increasing accuracy in reconstructing node's ancestral characters;][]{Finarelli2006} but is also desirable in a macroevolutionary way \citep{fritzdiversity2013,benton2015}.
However, in this particular example, I used disparity (i.e. morphological - or rather cladistic (see below) - diversity) as a proxy for testing this hypothesis.
Even though disparity analysis are becoming increasingly common in palaeobiology \citep[e.g.]{Butler2012,brusattedinosaur2012,toljagictriassic-jurassic2013,brusattegradual2014,bensonfaunal2014,Claddis,Close2015}, they still suffer from several biases.

%Morphological diversity
Firstly, morphological diversity is a complex concept to grasp or to interpret.
Describing the shape of an organism is not straightforward and many mathematical methods exist (e.g. Elliptic Fourier; \citealt{Fourier1982}; Procrustes; \citealt{JamesRohlf1993129}; Convex Hull; \citealt{ANDREW1979216}).
In biology, one major approach is to describe shape as a summary of an ordinated distance matrix based on procrustes \citep[i.e. a geometric morphometric approach][]{zelditch2012geometric}.
In studies using this approach, shape is approximated by actual continuous measurement collected from the organisms \citep[e.g.]{friedmanexplosive2010,hopkinsdecoupling2013,finlay2015morphological}.
However, in our case, we used differences (read inter-taxon distances) between particular morphological features \citep[e.g.][]{foote1997evolution,Wills2001,Wesley-Hunt2005}.
This method has been criticised by some of their users to be biased by: (1) the fact that these morphological features are not randomly collected and can distort reality \citep{Brusatte12092008}%develop
 or (2) that they are highly dependant on the quality of the fossil record \citep{Butler2012}.
However, these biases are overweighted by the advantages of (1) having many comparable morphological data among taxa \citep{Brusatte12092008} and by (2) correcting for the fossil record quality through time \citep{Butler2012}.
Additionaly, it has been shown that even though morphometric based and cladistic based morphological diversity are different, they seem to capture the same signal \citep{foth2012different,hetherington2015cladistic}.

%Multidimensionality
This trend shows the clear importance of treating biodiversity as a multidimensional rather than unidimensional metric \citep[similarly as in ecology;][]{DonohueDim}.
In fact, biodiversity is the combination of taxonomic diversity \citep[e.g.][]{Stadler12042011}, morphological diversity \citep[from cladistics or morphometrics;][]{hetherington2015cladistic} and phylogenetic diversity \citep[e.g. the evolutionary rates regimes;][]{Close2015}.
However, similarly to the comment on the \textit{Full} Total Evidence method above, this dimensionality could also include biogeographical or ecological diversity.

%Genetic diversity
In the end, it probably all boils down to genomic diversity since all these traits are highly correlated with genetics at a mechanistic level.
For example, taxonomic (as in species delimitations), morphology (as in heritable morphological features), ecology (as in the ``use'' of these morphological features) or even biogeography (as in drivers of population delimitations or local adaptations) all boil down to genetics.
However, two points still justify using multidimensional diversity rather than just genetic diversity: (1) firstly even though biology boils down to genetics in a Selfish Gene way (cite Dawkins), we are still far from understanding the genetic mechanisms that leads to all of theses aspects; (2) secondly, this can be only applicable to living taxa and ignores most of the \textit{actual} biodiversity \citep{novacek1992ext,raup1993extinction}.

All this is especially when combining living and fossil, species richness is a really poor indicator of diversity.

%Improvements?
This is still really promising and can be improved first by underestanding how all this works in a theoretical way (building the models).
And only then apply it to observed patterns.

\section{What is the real effect of combining?}
Maybe only important when groups have actually a complex history?
Old clades might have no living descendants and the question is therefore N/A
Recent subclades maybe not have changed much in diversity so adding fossils might not change much.
But we never know! Example of the giant lemur (recently extinct).

Tested effects : Finarelli and Flynn, 2006; Slater et al., 2012; Slater, 2013; Pant et al., 2014   PANT:http://www.biomedcentral.com/1471-2148/14/184


Mitchell: Simulation-based studies have shown that many commonly used methods lack the power to discriminate between different models reliably (Boettiger et al., 2012; Slater and Pennell, 2013), and the mismatch between the patterns informed by the extant-only comparative approaches and the patterns observed in the fossil record are stark.
Mitchell: Simulations and empirical results have shown that comparative methods have both low power (Boettiger et al., 2012) and an inability to predict non-monotonic changes in disparity.
Mitchell: by using clades with fossil data to find and fit reliable methods for inferring morphological evolution (e.g., subclade disparity through time), paleontologists and comparative biologists can work together to bring our data to as close to representative as possible.

Pant: Analyses based on extant taxa alone have the potential to oversimplify or misidentify macroevolutionary patterns. This study demonstrates the impact that integration of data from the fossil record can have on reconstructions of character evolution and establishes that body size evolution in sloths was complex, but dominated by trended walks towards the enormous sizes exhibited in some recently extinct forms.

\bibliography{References}