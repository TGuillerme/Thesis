\chapter{Discussion}
\label{chap:discussion}

%---------------------
%
% DISCUSSION
%
%---------------------

In this final chapter of my thesis, I first discuss the implications of the results of Chapters \ref{chap:TEM_paper} and \ref{chap:missing_mammals} and then of Chapter \ref{chap:STD_paper}.
In both sections, I discuss the methodological caveats of my approaches and propose future research avenues to solve these problems and expand into new areas.
Finally, I discuss the overall importance of combining both living and fossil species into macroevolutionary studies.

\section{The future of the Total Evidence method}
The Total Evidence method is a promising way of testing macroevolutionary hypotheses using extinct and extant taxa \citep[e.g.][]{ronquista2012,slaterphylogenetic2013,Wood01032013,beckancient2014,Dembo2015}.
However, as shown in the Chapters \ref{chap:TEM_paper} and \ref{chap:missing_mammals}, this method is quite sensitive to missing data.
As discussed in both chapters, increasing the number of morphological characters and the number of living taxa with coded morphological characters is the most efficient way to improve the phylogenetic signal. % NC: Explain more clearly. You mean to get accurate topologies really. The signal by itself is not useful.
This data is available and reasonably easy to access in natural history museum collections worldwide. 
In addition, software infrastructure has been developed to facilitate collaborative data collection of this kind \citep{morphobank}.
Therefore, hopefully the scarcity of morphological data for living species will be gradually solved with time, provided the funding exists to make such efforts possible.
However, there are other general problems with Total Evidence phylogenies that were not developed in the discussions of Chapters \ref{chap:TEM_paper} and \ref{chap:missing_mammals}. 

Firstly, dating Total Evidence phylogenies is difficult. 
Because these phylogenies contain both living and fossil taxa, the preferred way to date them is to use the tip-dating method \citep[e.g.][]{ronquista2012,Wood01032013,Dembo2015}.
This method relies on the age of the fossil taxa (treated as tips) to date the divergence times of the nodes, rather than defining node calibrations \textit{a priori} \citep[cf. node-dating;][]{ronquista2012}.
The tip-dating method has two main advantages: (1) it has been observed using empirical data that it improves the ability to recover the correct topology because it can use the stratigraphic age of the fossil taxa to favour some topological solutions more than others, typically by minimising implied ghost lineages \citep[][and personal communications]{BEASTmaster}; and (2) it reduces the confidence intervals of the node ages compared to a classic node-dating approach \citep{ronquista2012}.
This second point, however, has been questioned by \cite{Arcila2015131} who compared the tip-dating method and the latest models for the node-dating method, for example, the fossilised birth-death model \citep{heaththe2013}, and found the opposite effect, i.e. an increase in node age confidence interval with the tip-dating method.
It would therefore be interesting to run a similar analysis to the simulations in Chapter \ref{chap:TEM_paper} but adding a dating aspect to it.
By comparing dated Total-Evidence matrices using both node-dating and tip-dating, one could formally test the two advantages outlined above as well as their resilience to missing data.

Secondly, a more general problem is that the Total Evidence method relies on the M\textit{k} model \citep{lewisa2001} to measure the morphological distances among taxa.
This model is a generalisation of the Jukes-Cantor evolutionary model \citep[JC69;][]{jukes1969evolution} that allows a single mutation rate $\mu$ between all character states.
The JC69 model is a great simplification of reality and has been replaced in molecular phylogenetics with more complex models that are closer to biological reality \citep[e.g. the GTR model that allows a different rate for each different type of nucleotide change;][]{tavare1986}.
It is therefore likely that the M\textit{k} model is also a crude underestimation of the complex reality of morphological evolution, especially as the assumption that there is a unique transition rate between character states has been shown to be wrong in at least some cases \citep[e.g. for Dollo traits that are irreversible;][]{WrightDollo}.
\cite{spencerefficacy2013} even demonstrated that non-probabilistic methods such as maximum parsimony outperform the M\textit{k} model for the placement of contentious fossils such as \textit{Archaeopteryx}.
However, more recent and thorough simulations have demonstrated the opposite, %although the M\textit{k} model underestimates the complex reality of morphological evolution % NC: Is this a result from Aprils' paper? If so make it clearer it's not just another comment about MKV not being complex enough. If it is just another repeat of that just cut it here.
\citep{wrightbayesian2014}.  
%As the statistician George Box wrote, ``all models are wrong, but some are useful'' \citep{box1987empirical}.
%This can be typically the case for the Total Evidence method: despite the three major caveats discussed above (missing data, dating, and morphological evolution), this methods remains the only efficient method to date to include the diversity of life both past and present.
% NC: The Box quote applies to models, TEM is a method.
% NC: Need to finish this paragraph appropriately.

Although the Total Evidence method claims to be total because it uses both molecular and morphological data \citep{eernissetaxonomic1993}, this does not represent all of the data available to biologists.
Thus, one way to improve the Total Evidence method could be to use a \textit{Full} Total Evidence method.
Other sources of data such as traits (e.g. body mass), ecology (e.g. habitat) or biogeography could also be added to Total Evidence methods with appropriate evolutionary models and hypotheses for each type of data \citep[e.g. respectively quantitative, multiple or geographic state speciation and extinction model -- Qua-Mu-GeoSSE models;][]{fitzjohndiversitree2012}.
However, such data sets could improve the Total Evidence trees but also make them more complex statistical by increasing the number of parameters and assumptions.
Finally, this is also likely to simply increase the data availability problem.
% NC: Ummm my reaction to this paragraph is what are you talking about??? How would this help at all? Qua-Mu-GeoSSE are methods for looking at diversification rate shifts in response to quantitative, multiple, geographic traits. NOT a method of tree building. This paragraph is nonsense. Either develop these ideas more fully so it makes some kind of logical sense or just cut this.

% NC: Need to finish off this section
\section{Diversity is multidimensional} % NC: I don't like the section name here, think of a more accurate subheading rather than a catchy one.
% link to previous paragraph
One important point to keep in mind when building phylogenies, is that phylogenetic trees (whether they use all the available data or not) are tools for observing evolutionary patterns and testing evolutionary hypotheses, not the end point of the scientific endeavour.
% NC: I wonder if this sentence could be the end of the previous section?

% NC: Past tense now, you've done this! In the intro it's present tense as you're going to do it.

In Chapter \ref{chap:STD_paper}, I used two tip-dated Total Evidence trees to test whether the K-Pg mass extinction event influenced the diversity of mammals.
%I believe that the use of Total Evidence tree improves the timing of diversification events \citep[][ which is a crucial aspect when studying effect of mass extinctions which are finites points in time]{ronquista2012} or the estimation of morphological diversity \citep[increasing accuracy in reconstructing node's ancestral characters;][]{Finarelli2006}. % NC: Not sure how this is relevant.
Rather than using taxonomic richness as my proxy for diversity, I used disparity, i.e. morphological diversity (or in this case cladistic diversity), and tested whether mammalian disparity increased after the K-Pg extinction.
Although it is becoming increasingly common to estimate the disparity of clades in palaeobiology \citep[e.g.]{Butler2012,brusattedinosaur2012,toljagictriassic-jurassic2013,brusattegradual2014,bensonfaunal2014,Claddis,Close2015}, methods for estimating disparity suffer from several biases that I did not discuss in Chapter \ref{chap:STD_paper}.

%Morphological diversity - NC: Not really - more cladisto versus morpho space.
%Firstly, morphological diversity is a complex concept to grasp or to interpret.
% NC: Well no this isn't what you talk about in this paragraph. You talk about using a cladisto-space rather than a morpho-space. So just say that. Cut out irrelevant stuff. Remember the first line of the paragraph should be what the paragraph is about.
Describing the shape of an organism is not straightforward and many mathematical methods exist (e.g. Elliptic Fourier; \citealt{Fourier1982}; Procrustes; \citealt{JamesRohlf1993129}; Convex Hull; \citealt{ANDREW1979216}). % NC: Many mathematical methods exist to do what?
In biology, one approach is to describe shape as a summary of an ordinated distance matrix based on Procrustes % NC: Procrustes what? ordination? Analyses? theorem?
 \citep[i.e. a geometric morphometric approach][]{zelditch2012geometric}.

In studies using this approach, shape is approximated by actual continuous measurement collected from the organisms \citep[e.g.][]{friedmanexplosive2010,hopkinsdecoupling2013,finlay2015morphological}.
However, in our case, we used inter-taxon distances among particular morphological features \citep[e.g.][]{foote1997evolution,Wills2001,Wesley-Hunt2005}.
% NC: Get rid of the numbering and write the below out properly.
This method has been criticised by some to be biased by: (1) the fact that these morphological features are not randomly collected and can distort reality by emphasising differences in the taxonomic group of interest \citep{Hopkins24032015} or (2) that they are highly dependant on the quality of the fossil record \citep{Butler2012}.
However, these biases are overweighted by the advantages of (1) having many comparable morphological data among taxa \citep{Brusatte12092008} and by (2) the possibility of correcting for the fossil record quality through time \citep{Butler2012}.
Additionally, it has been shown that even though morphometric based and cladistic based disparity are different, they seem to capture the same signal \citep{foth2012different,hetherington2015cladistic}.
% NC: Also a lot of the justification for this is just "this data is available and there's a lot of it". Hard to get comparable measurement data across large phylogenies for anything else.

%Finally, which metric?
Secondly, disparity is an abstraction of morphological diversity: it is an unique value that describes and multidimensional transformation of an actual shape \citep{Wills1994,foote1997evolution}. % NC: It's not really "shape" per se. Looking at the cladistic characters it's all the stuff that makes up an organism. It is more the cladisto-space you're summarizing too, not individual species "shapes"
This can lead to problems in the interpretation of such a value since each step have its own caveats and limitations (i.e. describing the shape of an organism using morphometrics or cladistics and mathematically transforming this description into a matrix).
% NC: Again that is NOT what you are doing here, you're looking at sets of organisms

Classically people have used the four metrics proposed by \cite{Wills1994} (sum and product of variance and range) but several problems have never been explored.
% NC: Again I don't know if the numbering here helps. Bit lazy to have lots of lists.
\begin{enumerate}
\item firstly, in addition to the practical problems discussed in Chapter \ref{chap:STD_paper} the present software implementations for calculating the sum and product of variance never integrate the covariance present in the ordinated matrix.
\item secondly, even though some attempts have been made for measuring the efficiency of these metrics \citep{Ciampaglio2001} there been no global assessment of the statistical power of each metric for describing multidimensional space occupancy.
\item finally, these metrics are only describing the $n$ dimensions (i.e. the columns in the matrix) but are not directly describing the placement of the tips or nodes in the $n$ dimensions (cf. the distance between taxa and the centroid). % NC: Explain all of this more clearly
\end{enumerate}
Future developments of disparity through time studies would require a better understanding of the statistical performance of these disparity metrics and how each of them would be more appropriate to specific empirical situations. % NC: You can be a little more specific here about what you would do.

%Multidimensionality
Finally, the exciting results from the latest disparity through time stuides underline the importance of studying the multidimensionality of biodiversity \citep[cf. just taxonomic richness;][]{Butler2012,brusattedinosaur2012,toljagictriassic-jurassic2013,brusattegradual2014,bensonfaunal2014,Claddis,Close2015}.
It also encouraging to note that this is not only a palaeobiological approach to describing biodiversity but is also trending in other disciplines such as ecology \citep{DonohueDim}.
In fact, biodiversity is the combination of taxonomic diversity \citep[e.g.][]{Stadler12042011}, morphological diversity \citep[from cladistics or morphometrics;][]{hetherington2015cladistic} and phylogenetic diversity \citep[e.g. the evolutionary rates regimes;][]{Close2015}. % NC: Hmmm there are a lot of other aspects too. Be careful when making grand claims that you've actually read the literature (what about ecology, function, habitats, ecosystems, molecules, genes, etc?)
However, similarly to the comment on the \textit{Full} Total Evidence method above, this multidimensionality could also include biogeographical or ecological diversity.
Such analysis could lead to a better understanding of macroevolutionary patterns and could allow us to test more general evolutionary hypothesis such as the validity of the concept of ecological niches \citep{pearmanniche2008}. % NC: It's unclear HOW this would help. Need to make it clearer. Can't just randomly throw in comments like this with no justification. Also a LOT more recent lit than Pearmann on this.

% NC: The section below is really loose and arm wavey. Needs a complete re-write. i.e. start over.
\section{What is the real effect of combining living and fossil taxa in macroevolutionary studies?} %NC: What do you mean by a "real" effect?
This thesis tackles practical and theoretical aspects of using both living and fossil species in macroevolutionary studies.
% NC: Again, be careful here. You don't actually do that! Be more specific about what you actually really do. for example you haven't actually modified any methods for use with fossil and living taxa. They already existed. You modified the methods in other ways yes, but they could always cope with the data.
Several studies have already demonstrated the challenges and the importance of including fossils into phylogenies \citep[e.g.][]{ronquista2012,slaterphylogenetic2013,Wood01032013,beckancient2014,Dembo2015}.
However, all these studies (including my thesis) do not focus on the effect of adding fossil taxa to phylogenies but rather perform empirical or theoretical analyses while including fossil taxa and demonstrate the superiority of their findings to previous studies.
In fact, even though there is a strong consensus on the importance of such analysis \citep{jacksonwhat2006,quentaldiversity2010,dietlconservation2011,slaterunifying2013,fritzdiversity2013,benton2015}, the effect of combining both living and fossil has yet, to my knowledge, never been tested in a theoretical way.

This might be due to the difficulties to propose a generalised theoretical framework on which to test the effect of combining living and fossil species.
Yet, it is important to note that this thesis, along with several other studies, actually investigated this effect on some empirical data sets and consistently found an important effect of adding fossil data in macroevolutionary studies \citep{Finarelli2006,Slateretal2012,slaterphylogenetic2013,SlaterPennel2014,pant2014complex,Mitchell2015}. % NC: OK I see what you're trying to say in this paragraph. I think a lot of people are working on this, and actually I'm sure Graham has some stuff. I guess what you're trying to say is everyone is saying with fossils is better, but not many people are testing if it's true? Note that is is still an empirical question. GEIGER can deal with non ultrametric trees, so you could easily test some of this. Also I'm sure there's a Pennell and Slater paper on early burst that adds fossils and tests the effects, also something on OU somewhere. So I think you're wrong to say it hasn't been done at all, but it hasn't been done systematically or entirely quantitatively. 

One question arising from these studies is whether there is a \textit{real} effect of combining both living and fossil species into macroevolutionary studies.
% NC: Again what is a "real" effect? Do you mean an important effect? A Biologically meaningful effect?
In fact, one can argue that the conclusions from these studies are linked to the peculiarity of the groups studied (or the simulation protocol) that displayed a rather dynamic evolution that can only be revealed by combining all available data. % NC: How so? Explain.
Because the methods for combining living and fossil species are still challenging, it could be a futile and time consuming exercise in some scenarios such as: (1) when studying some clades that have no living relatives (e.g. Trilobita; \citealt{hopkinsdecoupling2013}; Pterosauria; \citealt{Butler2012}; etc.); (2) when studying clades with a really poor fossil record \citep[e.g. Aves where there are three orders of magnitude more known living than fossil taxa;][]{jetzthe2012,Mitchell2015}; (3) or when studying clades that have undergone a recent radiation \citep[e.g. Cichlidae][]{Genner01052007}. % NC: Explain

Ironically, % NC: Is it really ironic?
however, each of these three scenarios can also be used to demonstrate the importance of combining living and fossil taxa into macroevolutionary studies:
% NC: ARGH stop putting everything into lists and hoping you can just not explain anything properly!!!
\begin{enumerate}
\item counter intuitively, combining clades % NC: with what?
 with no living relatives might be really important for understanding macroevolutionary patterns in living taxa. For example morphological study of long extinct Ostracoderma (armoured jawless fishes) can help understanding characters evolution in later Gnatostoma \citep[jawed vertebrates;][]{Janvier2015}. % NC: explain
\item in fossil poor clades, the few available fossils can actually bring precious information on the early history of the group. For example, in Aves, disparity is underestimated when ignoring fossils, even if there are only a handful of fossils available \citep[e.g. 58 fossil genera against 604 living ones;][]{Mitchell2015}.
\item finally, excluding fossils from recent clades might also be detrimental to macroevolutionary interpretations, for example, in Lemuroidea, some sub-fossils species had a body mass several orders of magnitudes bigger than all living lemurs \citep{hartwig2002primate,Jungers2008} and only went extinct at latest around 600 years ago \citep{goodman2003introduction}. % NC: explain implications more clearly.
\end{enumerate}

% NC: Needs a bit more thought
Therefore I argue that there is no biological justification to not jointly using living and fossil species since the effect of combining them can not be known \textit{a priori}. %NC: This conclusion does not follow from earlier paragraphs at all!!!
Our knowledge in biology has tremendously advanced since the last half century ranging from the amazing revelation of the few glimpses of the deep past provided by the the fossil record to the understanding of the complexity and dynamics of modern ecosystems.
Because the inherent characteristic of the deep past is to be unknown and mysterious, it is therefore crucial to incorporate all of this knowledge in macroevolutionary studies to continue revealing the complexities of biodiversity.
%``grandeur [of] this view of life'' \citep{darwin}. % NC: This isn't the correct quote so don't use it, it just sounds forced.


%\bibliography{References}