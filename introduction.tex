\chapter{Introduction}
\label{chap:introduction}

%§ 1 - General intro: we need both living and fossils species to understand stuff
%General intro paragraph (we need both living and fossil species - to understand the present patterns we need to look at the past,; but the data is scarce so it’s more accurate to look at the present; but that ignores most of life history)

%(Maybe add a figure showing the real evolutionary history (a tree  with data) and what we actually know of it (a tree with a lot of data at the top)

%§ 2 - How can we combine them
%How can we combine them. Traditional approach down up (morphology) but modern approach top down (molecules)

%§ 3 - What can we do when we combine them?
%Interesting studies or examples?

%§ 4 - In this thesis I worked on both aspects: how to combine them (TEM + missing mammals) and what to do with them (STD)

%§ 5 -Introducing chapter 1
%TEM and missing data (link missing data)
In the first chapter, I runa long term and thorough simulation to test how robust are our phylogenetic inferences when we combine living taxa with molecular and morphological to fossil taxa with morphological data only.
I particularly focused on how missing data in both living and fossil taxa can affect topology.
I found that the number of living taxa with available data is essential to recover accurate topologies.
Therefore we need data for living mammals, how much of it is out there?
(This chapter is currently in review in Molecular Phylogenetics and Evolution - revisions).

%§ 6 -%ntroducing chapter 2
%(link missing data) Missing data in mammals (link mammals)
Following these results, I was interested in showing practical implications of this effect and monitored the morphological data availability for living mammals.
I downloaded all the recent available morphological matrices and counted the number of living mammals with available morphological data.
I then tested how these taxa where distributed accross the phylogeny to check if there weren't clustered in some specific clades.
I found that a lot of data is missing but that at least most of it is randomly distributed and should not drastically effect topology.
So it's not so bad, but what can we do with these total evidence trees?
(This chapter is an invited submission to a special issue in Biology Letters. Submission is due in December 2015).

%§ 7 -Introducing chapter 3
%(link mammals) STD with mammals (link to cool stuff)
One important step in explaining macroevolutionnary processes is to accurately describe the patterns.
For example, to explain the processes driving diversification or extinction during a mass extinction event, we need to accurately measure what's happening.
One classical example is the K-T extinction where the effects still remain unclear after so many years of research.
Because I showed in the previous chapters that mammals are ok for combinations, I studied how they get affected by K-T disparity wise (INTRODUCE DISPARITY FIRST).
I found that...
All the cool stuff we can do with TEM!
(This chapter will be submitted to Evolution).

%§ 8 - Introducing discussion
%(link to cool stuff) end.
Finally I will discuss all these cool stuff and how research might develop by doing the combinations of data and looking at more accurate descriptors of patterns (e.g. diversity AND disparity) blalbalblabla.

%§ 9 - Introducing discussion
In addition I was also interested in a side project on developing tools for phylogenetic correction that takes into account tree uncertainty.
I participated to a collaborative project exploring drivers of longevity accross birds and mammals led by Kevin Healy.
I developed the implementation to take tree uncertainty into account and this is now an available R pckage (mulTree).
%In addition to that I also did longevity.
%I was involved in developing the method and running the analysis for this paper. Phylogenetic correction is one crucial aspect in accurately describing maco patterns. In the paper, along with the main author, we developed and implemented a method for allowing to include phylogenetic uncertainty in generalized linear mixed models.