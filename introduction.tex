\chapter{Introduction}
\label{chap:introduction}

%Some quote from GG Simpson
%"Certainly paleontologists have found samples of an extremely small fraction, only, of the earth's extinct species, and even for groups that are most readily preserved and found as fossils they can never expect to find more than a fraction.""
%But I'm not sure, maybe not quote is better.


%---------------------
%
% GENERAL INTRO - MAKE IT SHORT
% 
%---------------------

% Why is all this important? Dig from Benton and Fritz et al
The amazing diversity of organisms living on the biosphere represents an overwhelmingly low fraction of the organisms that ever existed \citep{novacek1992ext,raup1993extinction}, yet, most of the work in biology focus solely on living species \citep{fritzdiversity2013,benton2015}.
Ignoring that can lead to missconceptions on species richness for example.
(quick example).
Another, more impressive, benefit of combining living and fossil taxa can be to look at the true ecological diversity.
For example, nowadays crocodilomorphs ecological diversity is low (fluvial or coastal).
But past biodiversity is way richer: terrestrial (cite), marine (cite) or even arboreal (cite).
And primates!
These effects of ignoring the fossil record do not only come down to changes in species or ecological diversity (i.e, there where more species and before) but also more technical aspects such as timing of diversification events \citep[e.g.][]{pyrondivergence2011}, relationships among lineages \citep[e.g.][]{manosphylogeny2007} or niche occupancy \citep[e.g.][]{pearmanniche2008}. % Rephrase
All this have led to consensus among scientists that we need to combine \citep{jacksonwhat2006,quentaldiversity2010,dietlconservation2011,slaterunifying2013,fritzdiversity2013,benton2015}.
%Add primate diversity figure? 
Yet in this primates example, it is really difficult, it demands to combine loads of stuff.

\section{Combining data, methods and disciplines}
Yet all this is still really complex, since the last decade, only few studies have used combined analysis in the clear aim of improving understanding of biodiversity.
For example PCM are not entirely good with non-ultrametric trees.
Or dating techniques are not perfect with fossils (Some calibration technique and Arcila).
This is primary due to a difference in data, methods and disciplines.
% Traditional approach down up (morphology) but modern approach top down (molecules)
The traditional way to combine both living and fossils data is using morphological data for both and draw conclusions on want happened.
However the problem is that such approach simply exclude living species (e.g. trilobites) or use living species just as a way to branch the fossil species in a macroevolutionary context (e.g. O'Leary or any other big cladistic paper?).
Also, the methods use to describe relations can be have big artefacts (e.g. parsimony) or other approaches (Mk) model can be oversimplistic (but still usuable).
Finally the data for fossils is usually restricted to morphology and the few ecological traits that can be extract from that (e.g. diet from teeth but not behaviour or population size)
Another approach for looking at macroevolution is to solely use living species (e.g. Jetz) one can look at the differences between DNA (many differences, good) and use models that are more realistic than Mk because of only four states(e.g. GTR).
Also this method can include time by calibrating the molecular clocks using fossils (Zuckerkandl).
However, appart from the use of the fossils occurence dates for making the clock tick, this approach ignores all clades that have no living descendant (the majority of clades!) and can even poorlierly estimate things from fossils.
...
Therefore, combining both data allows use to palliate to some of the problems of both approaches!

\section{Phylogenies with living and fossil species}
However, one succesfull method seems to come back in trend: the Total Evidence method (cite old papers + new papers) in the last four years (cite the new papers).
This method allows to combine both molecular data for living species and morphological data for both living and fossil species into phylogenies.
Doing so allows mainly to treat palaoentological and neontological equally, which is a first step for combining methods, data and disciplines.
However, because of the amount of data needed, this method is likely to suffer from missing data

\subsection{Effects of missing data on topological inference using a Total Evidence approach}
In the second chapter of this thesis I tackled the problem of missing data in Total Evidence matrices.
I ran long term and thorough simulation to test how robust are our phylogenetic inferences when we combine living taxa with molecular and morphological to fossil taxa with morphological data only.
I particularly focused on how missing data in both living and fossil taxa can affect topology.
I found that the number of living taxa with available data is essential to recover accurate topologies.
Therefore we need data for living species.
But this rises the question: how much of it is out there?

\subsection{Morphological data availability in living mammals}
Following this question, in the third chapter of my thesis, I looked at how many data was available in mammals.
Following these results, I was interested in showing practical implications of this effect and monitored the morphological data availability for living mammals.
I downloaded all the recent available morphological matrices and counted the number of living mammals with available morphological data.
I then tested how these taxa where distributed accross the phylogeny to check if there weren't clustered in some specific clades.
I found that a lot of data is missing but that at least most of it is randomly distributed and should not drastically effect topology.
Since data in mammals is improvable, but is ok at higher taxonomic levels, it is an excellent candidate group for building Total Evidence phylogenies to allow macroevolutionary studies including both living and fossil species.

\section{Total evidence phylogenies applications}
These trees can allows use to capture macroevolutionary or macroecological patterns more accurately and therefore propose more solid hypothesis on processes.
Slater and Beck have successfully build Total Evidence and tip-dated phylogenies.
We can use these phylogenies for answering many question such as body mass evolution (Slater) or timing of diversification (beck) and that improves the whole yoke.
Another interesting we can do with such phylogenies is to look at diversity through time more accurately.

\subsection{Cretaceous-Palaeogene extinction does not affect mammalian disparity}
% Cheesy quote for that bit "The most erroneous stories are those we think we know best - and therefore never scrutinize or question." Gould whenever
One interesting point about diversity it that it doesn't has to be just species richness but sometimes disparity can be important as well.
We can use both processes plus Total Evidence trees to better describe the macroevolutionary patterns.
These more accurate patterns can be used to explain the processes driving diversification or extinction during a mass extinction event, we need to accurately measure what's happening.
One classical example is the K-T extinction where the effects still remain unclear after so many years of research.
In the fourth chapter, I explore this question using Total Evidence trees and focusing on disparity rather than species diversity to see if mammals were affected by the K-Pg extinction event.
I found that mammals do not do a damn thing around the K-Pg boundary.

\section{Discussion} % TG: I think if the title stays as shitty, no need for a title.
This is just an example on how including both living and fossil species can change our vision of biodiversity.
In the last chapter, I will discuss potential more application but also problems that arise with such methods

%\bibliography{References}