\chapter{Introduction}
\label{chap:introduction}
%General intro paragraph (we need both living and fossil species - to understand the present patterns we need to look at the past,; but the data is scarce so it’s more accurate to look at the present; but that ignores most of life history)


%(Maybe add a figure showing the real evolutionary history (a tree  with data) and what we actually know of it (a tree with a lot of data at the top)

%Explain the global concept: how and why do we need to combine living and fossil species together.

%How can we combine them. Traditional approach down up (morphology) but modern approach top down (molecules)

%What can we do when we combine them?

%Introducing what’s the point of the thesis

%Introducing chapter 1
%TEM and missing data (link missing data)

%Introducing chapter 2
%(link missing data) Missing data in mammals (link mammals)

%Introducing chapter 3
%(link mammals) STD with mammals (link to cool stuff)

%Introducing discussion
%(link to cool stuff) end.

%In addition to that I also did longevity.
%I was involved in developing the method and running the analysis for this paper. Phylogenetic correction is one crucial aspect in accurately describing maco patterns. In the paper, along with the main author, we developed and implemented a method for allowing to include phylogenetic uncertainty in generalized linear mixed models.