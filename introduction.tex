\chapter{Introduction}
\label{chap:introduction}

%Some quote from GG Simpson
%"Certainly paleontologists have found samples of an extremely small fraction, only, of the earth's extinct species, and even for groups that are most readily preserved and found as fossils they can never expect to find more than a fraction.""
%But I'm not sure, maybe not quote is better.


%---------------------
%
% GENERAL INTRO - MAKE IT SHORT
% 
%---------------------

% Why is all this important? Dig from Benton and Fritz et al
The amazing diversity of organisms living on the biosphere represents an overwhelmingly low fraction of the organisms that ever existed \citep{novacek1992ext,raup1993extinction}, yet, most of the work in biology focus solely on living species \citep{fritzdiversity2013}.
Ignoring this, can lead to misinterpretation of macroevolutionary or macroecological patterns \citep{jacksonwhat2006,quentaldiversity2010,dietlconservation2011,slaterunifying2013,fritzdiversity2013,benton2015}.


Although most species that have ever lived are now extinct \citep{novacek1992ext,raup1993extinction}, the majority of macroevolutionary studies focus solely on living species (e.g. \citealp{meredithimpacts2011,jetzthe2012})


% Natalie: Interspecific competition is the negative effect one species has upon another by consuming, or controlling access to, a resource that is limited in availability (Keddy 1989).

Studying events in the evolutionary history of a taxonomic group, such as adaptive radiation or extinction, requires a fine-­‐scale and accurate resolution of their phylogenetic relationships through time. To achieve this, most scientists would agree that information about both extant and extinct species is needed. However, few efforts have been made to combine extant and extinct species in the same phylogenetic trees; instead phylogenetic trees usually contain only extant species. Because the vast majority of species in a lineage will be represented by extinct species, studies focusing on extant species contain less than 0.1\% of the lineage’s species richness. In some clades, ignoring extinct species may also obscure the true evolutionary history, species richness (i.e. Proboscidea), biogeography (i.e. Tinamiforms) or ecological diversity (i.e. Crocodilomorphs). Thus including the fossil record in these kinds of studies is essential to fully understand the evolutionary history of lineages.

In some clades, ignoring extinct species may also obscure the true evolutionary history or diversity of a clade. For example, in mammals the orders Perissodactyla (horses, rhinos and tapirs) and Proboscidea (elephants and mammoths) currently contain only a few taxa although they contained many more species in the past (Cifelli 1981, Antoine et al. 2003, Mihlbachler 2008, Seiffert et al. 2012). Crocodilomorph reptiles also had much greater species diversity (there were ~45 genera during the Paleogene1 but only nine today - Martin 2009) and larger geographic ranges (e.g. extinct Crocodilomorpha remains are found in northern France - Hua 1997) in the past. Finally, ignoring extinct diversity can lead to misinterpretation in comparative phylogenetic studies. For example, ignoring extinct Dinornithiforms (Moas) leads to misinterpretation of both the diversification pattern and biogeography of ratites (Jetz et al. 2012). Thus including the fossil record in these kinds of studies is essential to fully understand the evolutionary history of lineages (Slater et al. 2012).

% Add crocs et al example from steering committees reports

%(Maybe add a figure showing the real evolutionary history (a tree  with data) and what we actually know of it (a tree with a lot of data at the top)

% Bits
% Although most species that have ever lived are now extinct \citep{novacek1992ext,raup1993extinction}, the many large-scale macroevolutionary studies focus solely on living species (e.g. \citealp{meredithimpacts2011,jetzthe2012}).
% Throughout history, life on Earth has suffered a series of mass extinction events resulting in drastic declines in global biodiversity \citep[e.g.][]{RaupPT,BentonPT,rennetime2013,Brusatte2015}.
% There is an increasing consensus among evolutionary biologists that studying both living and fossil taxa is essential for fully understanding macroevolutionary patterns and processes \cite{slaterunifying2013,fritzdiversity2013,Wood01032013}.

% Ignoring fossil taxa may lead to misinterpretation of macroevolutionary patterns and processes such as the timing of diversification events \citep[e.g.][]{pyrondivergence2011}, relationships among lineages \citep[e.g.][]{manosphylogeny2007} or niche occupancy \citep[e.g.][]{pearmanniche2008}.
% For example, including both living and fossil taxa in evolutionary studies can improve the accuracy of timing diversification events (e.g. \cite{ronquista2012}, our understanding of relationships among lineages (e.g. \cite{beckancient2014}, and our ability to infer biogeographical patterns through time (e.g. \cite{Meseguer01032015}.
% However, the long-term effects of mass extinctions are more varied \citep{Erwin1998344}, and include increases in species richness in some clades \citep{friedmanexplosive2010}, species richness declines in others \citep{Benton85}, changes in morphological diversity \citep{Ciampaglio2001,Ciampaglio2004,kornextinction2013} and shifts in ecological dominance \citep[e.g.][]{Brusatte12092008,toljagictriassic-jurassic2013,bensonfaunal2014}.

% This has led to increasing consensus among evolutionary biologists that fossil taxa should be included in macroevolutionary studies \citep{jacksonwhat2006,quentaldiversity2010,dietlconservation2011,slaterunifying2013,fritzdiversity2013}.
% To do this, however, we need to be able to place living and fossil taxa into the same phylogenies; a task that remains difficult despite recent methodological developments \citep[e.g.][]{pyrondivergence2011,ronquista2012,BEASTmaster}.
% To perform such analyses it is necessary to combine living and fossil taxa in phylogenetic trees.
% One increasingly popular method, the Total Evidence method \cite{eernissetaxonomic1993,ronquista2012}, combines molecular data from living taxa and morphological data from both living and fossil taxa in a supermatrix (e.g. \cite{pyrondivergence2011,ronquista2012,schragocombining2013,slaterunifying2013,beckancient2014,Meseguer01032015}, producing a phylogeny with living and fossil taxa at the tips. 
% These shifts are characterized by the decline of one clade that is replaced by a different unrelated clade with a similar ecological role (e.g. Brachiopoda and Bivalvia at the end Permian extinction \citealt{Sepkiski1981,CLAPHAM01102006} but see \citealt{Payne22052014}). 

% These phylogenies can be dated using methods such as tip-dating \cite{ronquista2012,Wood01032013} and incorporated into macroevolutionary studies (e.g. \cite{ronquista2012,Wood01032013,slaterphylogenetic2013}.
% Shifts in ecological dominance are of particular interest because they are a fairly common pattern observed in the fossil record (e.g. Foraminifera; \citealt{D'Hondt01011996,Coxall01042006}; Ichtyosauria; \citealt{thorneresetting2011}; Plesiosauria; \citealt{bensonfaunal2014}) and are often linked to major macroevolutionary processes such as adaptive \citep{Losos2010} or competitive radiations \citep{Brusatte12092008}.

\section{Combining data, methods and disciplines}

grab and expand the examples from the introductions.

%§ 2 - How can we combine them
%How can we combine them. Traditional approach down up (morphology) but modern approach top down (molecules)
The traditional way to combine both living and fossils data is using morphological data for both and draw conclusions on want happened.
However the problem is that such approach simply exclude living species (e.g. trilobites) or use living species just as a way to branch the fossil species in a macroevolutionary context (e.g. O'Leary or any other big cladistic paper?).
Also, the methods use to describe relations can be have big artefacts (e.g. parsimony) or other approaches (Mk) model can be oversimplistic (but still usuable).
Finally the data for fossils is usually restricted to morphology and the few ecological traits that can be extract from that (e.g. diet from teeth but not behaviour or population size)
Another approach for looking at macroevolution is to solely use living species (e.g. Jetz) one can look at the differences between DNA (many differences, good) and use models that are more realistic than Mk because of only four states(e.g. GTR).
Also this method can include time by calibrating the molecular clocks using fossils (Zuckerkandl).
However, appart from the use of the fossils occurence dates for making the clock tick, this approach ignores all clades that have no living descendant (the majority of clades!) and can even poorlierly estimate things from fossils.

Therefore, combining both data allows use to palliate to some of the problems of both approaches!


%§ 3 - What can we do when we combine them?
%Interesting studies or examples?

%§ 4 - In this thesis I worked on both aspects: how to combine them (TEM + missing mammals) and what to do with them (STD)

\section{Phylogenies with living and fossil species}

TEM Intro on the data story.

\subsection{Effects of missing data on topological inference using a Total Evidence approach}

%§ 5 -Introducing chapter 1
%TEM and missing data (link missing data)
In the first chapter, I run long term and thorough simulation to test how robust are our phylogenetic inferences when we combine living taxa with molecular and morphological to fossil taxa with morphological data only.
I particularly focused on how missing data in both living and fossil taxa can affect topology.
I found that the number of living taxa with available data is essential to recover accurate topologies.
Therefore we need data for living mammals, how much of it is out there?
(This chapter is currently in review in Molecular Phylogenetics and Evolution - revisions).

\subsection{Morphological data availability in living mammals}
%§ 6 -%ntroducing chapter 2
%(link missing data) Missing data in mammals (link mammals)
Following these results, I was interested in showing practical implications of this effect and monitored the morphological data availability for living mammals.
I downloaded all the recent available morphological matrices and counted the number of living mammals with available morphological data.
I then tested how these taxa where distributed accross the phylogeny to check if there weren't clustered in some specific clades.
I found that a lot of data is missing but that at least most of it is randomly distributed and should not drastically effect topology.
So it's not so bad, but what can we do with these total evidence trees?
(This chapter is an invited submission to a special issue in Biology Letters. Submission is due in December 2015).

\section{Total evidence phylogenies applications}

Once we have these trees we can do loads of cool stuff (primates ideas).

\subsection{Cretaceous-Palaeogene extinction does not affect mammalian disparity}

% Cheesy quote for that bit "The most erroneous stories are those we think we know best - and therefore never scrutinize or question." Gould whenever

%§ 7 -Introducing chapter 3
%(link mammals) STD with mammals (link to cool stuff)
One important step in explaining macroevolutionnary processes is to accurately describe the patterns.
For example, to explain the processes driving diversification or extinction during a mass extinction event, we need to accurately measure what's happening.
One classical example is the K-T extinction where the effects still remain unclear after so many years of research.
Because I showed in the previous chapters that mammals are ok for combinations, I studied how they get affected by K-T disparity wise (INTRODUCE DISPARITY FIRST).
I found that...
All the cool stuff we can do with TEM!
(This chapter will be submitted to Evolution).

\section{Further directions} % TG: I think if the title stays as shitty, no need for a title.

%§ 8 - Introducing discussion
%(link to cool stuff) end.
Finally I will discuss all these cool stuff and how research might develop by doing the combinations of data and looking at more accurate descriptors of patterns (e.g. diversity AND disparity) blalbalblabla.
%Add the bits of discussion from TEM new conclusion

%\bibliography{References}