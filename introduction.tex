\chapter{Introduction}
\label{chap:introduction}

%Some quote from GG Simpson
%"Certainly paleontologists have found samples of an extremely small fraction, only, of the earth's extinct species, and even for groups that are most readily preserved and found as fossils they can never expect to find more than a fraction.""
%But I'm not sure, maybe not quote is better.


%---------------------
%
% GENERAL INTRO - MAKE IT SHORT
% 
%---------------------

% Why is all this important? Dig from Benton and Fritz et al

Today's amazing biodiversity represents only an overwhelmingly small fraction of the organisms that ever existed \citep{novacek1992ext,raup1993extinction}.
Even though the process that shaped the patterns observed today are influenced by evolutionary history \citep{fritzdiversity2013}, most of the scientific endeavour in biology focus solely on living species.
Ignoring that can lead to misinterpretation of macroevolutionary patterns and processes \citep{benton2015}.
For example, nowadays crocodilians constitute a species poor group \citep[25 species;][]{uetz2010original} with a low range shapes and environments \citep[marine or freshwater;][]{Martin2008}.
Therefore when studying macroevolutionary patterns among all vertebrates, their effect will be rather ``marginal'' \citep[e.g.][ suggests that terrestriality is a driver of diversification among living vertebrates]{Wiens2015}.
However, this group was much more diverse both in terms of species richness \citep[244 species reported in][]{Bronzati2015} or in terms shapes and environments \citep{stubbs2013}.
In the case of \cite{Wiens2015}, not including fossil species, conceal the true history of this clade and this might biase the conclusions of the study.

Besides, including fossil species not only accounts for groups that where more diverse in the past, it also highly improves our descriptions of macroevolutionary patterns such as the timing of diversification events \citep[e.g. significantly reducing node age confidence intervals;][]{ronquista2012}, the relationships among lineages \citep[e.g. solving some controversial fossil placement;][]{Dembo2015} or even gives a potential solution for understanding niche occupancy through time \citep[e.g.][]{pearmanniche2008}.
All this studies have led to a recent consensus among scientists that we need to combine both living and fossil species in macroevolutionary analysis \citep{jacksonwhat2006,quentaldiversity2010,dietlconservation2011,slaterunifying2013,fritzdiversity2013,benton2015}.
Yet, in practice, only few studies have actively focused on combining them since the last decade \citep[e.g.][]{ronquista2012,slaterphylogenetic2013,Wood01032013,beckancient2014,Arcila2015131,Dembo2015}. %TG: or not sure if this is the good string of cites. These papers are mainly about TEM and tip-dating.

% TG: add croc figure? with/without fossils?

%\section{Combining data, methods and disciplines}
This scarcity %of practical applications of combining living and fossils species
is probably due to the fundamental differences between the two approaches to study macroevolution by using either living (neontological) or fossil (palaeontological) data.

\begin{enumerate}
\item The Paleontological approach was heavily popularised by \cite{simpson1945} and is based on cladistic data of the fossil record (i.e. discrete morphological observation).
It relies on optimal criteria such as maximum parsimony \citep{Hennig1966,felsenstein2004} to resolve the relations among lineages and on stratigraphy to time such trees \citep{GoloboffTNT}.
This approach allows a direct interpretation of macroevolution in deep time and benefits from recent improvements both on data collection \citep[e.g. ``phenomics'';][]{O'Leary08022013} and on dating method \citep[e.g. the \textit{cal3} method;][]{Bapst2014}.
However, this approach does rarely takes into account full living diversity \citep[e.g. 119 fossil and 38 living primates in][]{ni2013oldest} and methods suffer from several biases \citep[e.g. parsimony][]{wrightbayesian2014}.

\item Conversely, the neontological approach uses the vast amount of available molecular from living species and is based on probabilistic methods (e.g. Maximum Likelihood or Bayesian).
This approach is based on evolutionary models that rely on the differences in DNA to resolve the relations among lineages and on some specific fossils' occurrence dates for timing the lineages divergence \citep[i.e. the molecular clock][]{zuckerkandl1965}.
There has been enormous improvements of this approach in the last decade on both the evolutionary models \citep[e.g.][]{bapsta2013,stadlerdating2013,heaththe2013} and on which fossils to use to calibrate the trees \citep{Donoghue2007424,Parham01032012}.
However, this approach uses only the ages of certain fossils instead of the vast amount of informations available from the fossil record (e.g. species richness, traits, biogeography, etc).
\end{enumerate}


\section{Phylogenies with living and fossil species}
Nonetheless, the last three years have seen the development of the new trending Total Evidence method \citep{ronquista2012,slaterphylogenetic2013,Wood01032013,schragocombining2013,beckancient2014,Arcila2015131,Dembo2015}.
This methods allow to combine both molecular data from living species and morphological data from living and fossil species in the same phylogenetic matrices.
It was first developed in the nineties \citep{eernissetaxonomic1993} but only recently successfully implemented in softwares \citep{Ronquist2012mrbayes,BEAST2}.
By using both available neontological and palaeontological data, this methods allows to better study macroevolutionary patterns and processes.
For example, it allowed great improvements on the estimation of divergence event \citep[e.g.][]{ronquista2012}; evolutionary rates \citep[e.g.][]{beckancient2014}; topology \citep[e.g.][]{Dembo2015}; traits evolution \citep[e.g.][]{slaterphylogenetic2013} or even speciation processes \citep[e.g.][]{Wood01032013}.
There is, however, one drawback to this method: because it needs both molecular data for living species and morphological data for living and morphological species, it is susceptible to suffer from great amounts of missing data.

\subsection{Effects of missing data on topological inference using a Total Evidence approach}
As a first part of this PhD thesis, in the second chapter, I tackled the problem of missing data in Total Evidence matrices.
I ran long term and thorough simulations to test whether the topologies inferred from Total Evidence matrices were stable to missing morphological data.
I removed morphological data from Total Evidence matrices via three parameters where data could be missing: (1) the number of living species with molecular data but no morphological data; (2) the amount of missing data in the fossil record and (3) the number of overall morphological characters in the matrix.
I modified the level of data in the three parameters and in their combination and then inferred the phylogenetic topology using both Maximum Likelihood and Bayesian approach.
Finally, I compared how the missing data parameters and their interactions as well as the phylogenetic inference method influenced the ability of estimating the correct topology.
I found that the number of living taxa with both morphological and molecular data is the essential to recover accurate topologies.
This study rose the question of how can we improve Total Evidence topologies and especially, how much morphological data is available for living taxa?

\subsection{Morphological data availability in living mammals}
%Stopped here
Following this question, in the third chapter of my thesis, I looked at how many data was available in mammals.
Following these results, I was interested in showing practical implications of this effect and monitored the morphological data availability for living mammals.
I downloaded all the recent available morphological matrices and counted the number of living mammals with available morphological data.
I then tested how these taxa where distributed accross the phylogeny to check if there weren't clustered in some specific clades.
I found that a lot of data is missing but that at least most of it is randomly distributed and should not drastically effect topology.
Since data in mammals is improvable, but is ok at higher taxonomic levels, it is an excellent candidate group for building Total Evidence phylogenies to allow macroevolutionary studies including both living and fossil species.

\section{Total evidence phylogenies applications}
These trees can allows use to capture macroevolutionary or macroecological patterns more accurately and therefore propose more solid hypothesis on processes.
Slater and Beck have successfully build Total Evidence and tip-dated phylogenies.
We can use these phylogenies for answering many question such as body mass evolution (Slater) or timing of diversification (beck) and that improves the whole yoke.
Another interesting we can do with such phylogenies is to look at diversity through time more accurately.

\subsection{Cretaceous-Palaeogene extinction does not affect mammalian disparity}
% Cheesy quote for that bit "The most erroneous stories are those we think we know best - and therefore never scrutinize or question." Gould whenever
One interesting point about diversity it that it doesn't has to be just species richness but sometimes disparity can be important as well.
We can use both processes plus Total Evidence trees to better describe the macroevolutionary patterns.
These more accurate patterns can be used to explain the processes driving diversification or extinction during a mass extinction event, we need to accurately measure what's happening.
One classical example is the K-T extinction where the effects still remain unclear after so many years of research.
In the fourth chapter, I explore this question using Total Evidence trees and focusing on disparity rather than species diversity to see if mammals were affected by the K-Pg extinction event.
I found that mammals do not do a damn thing around the K-Pg boundary.

\section{Discussion} % TG: I think if the title stays as shitty, no need for a title.
This is just an example on how including both living and fossil species can change our vision of biodiversity.
In the last chapter, I will discuss potential more application but also problems that arise with such methods

%\bibliography{References}