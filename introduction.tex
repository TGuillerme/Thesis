\chapter{Introduction}
\label{chap:introduction}

%Some quote from GG Simpson
%"Certainly paleontologists have found samples of an extremely small fraction, only, of the earth's extinct species, and even for groups that are most readily preserved and found as fossils they can never expect to find more than a fraction.""
%But I'm not sure, maybe not quote is better.


%---------------------
%
% GENERAL INTRO - MAKE IT SHORT
% 
%---------------------

% Why is all this important? Dig from Benton and Fritz et al

Today's amazing biodiversity represents only an overwhelmingly low fraction of the organisms that ever existed \citep{novacek1992ext,raup1993extinction}.
Yet, most of the scientific endeavour in biology focus solely on living species even though the process that shaped the patterns observed today are influenced by evolutionary history \citep{fritzdiversity2013}.
Ignoring that can lead to misinterpretation of macroevolutionary patterns and processes \citep{benton2015}.


\cite{Wiens2015} studied drivers of diversification among living vertebrates and found a significant correlation between high diversification rates and terrestrial habitat.
However, diversification is a measure of speciation through time and thus might be meaningless in the present.


-Crocodilomorphs


Besides, including fossil species not only accounts for groups that where more diverse in the past, it also highly improves our descriptions of macroevolutionary patterns such as the timing of diversification events \citep[e.g. significantly reducing node age confidence intervals;][]{ronquista2012}, the relationships among lineages \citep[e.g. solving some controversial fossil placement;][]{Dembo2015} or even gives a potential solution for understanding niche occupancy through time \citep[e.g.][]{pearmanniche2008}
All this studies have led to a recent consensus among scientists that we need to combine both living and fossil species in macroevolutionary analysis \citep{jacksonwhat2006,quentaldiversity2010,dietlconservation2011,slaterunifying2013,fritzdiversity2013,benton2015}.
Yet, in practice, only few studies have actively focused on combining them since the last decade \citep[e.g.][]{ronquista2012,slaterphylogenetic2013,Wood01032013,beckancient2014,Arcila2015131,Dembo2015}. %TG: or not sure if this is the good string of cites. These papers are mainly about TEM and tip-dating.

\section{Combining data, methods and disciplines}
This is probably due to the fundamental differences between the living (neontological) data and fossil (palaeontological) data and approaches.

Neontological approach can be seen as top-down.
A lot of data: phylogenetic since the advent of DNA but also ecological and life history.
Even though the molecular clock \citep{zuckerkandl1965} provides good method for the past
especially improved with good calibrations (parham) advanced models (Fossilised Birth Death) allowing diversification studies \citep{Stadler12042011}.
But they ignore fossils (appart from some calibrations).

The Paleontological approach on the other can be seen as bottom up.
It uses extensive morphological data from the fossil record \citep[e.g. phenomics;][]{}(O'Leary) and good estimation of traits (find a cite).
However, methods can lack of statistical power (parsimony vs. Mk) and is rarely linked to living species (find a cite)

But this can also be due to technical problems in methods.
For example PCM are not entirely good with non-ultrametric trees.
Or dating techniques are not perfect with fossils (Some calibration technique and Arcila).
Also, the methods use to describe relations can be have big artefacts (e.g. parsimony) or other approaches (Mk) model can be oversimplistic (but still usuable).


\section{Phylogenies with living and fossil species}
However, one succesfull method seems to come back in trend: the Total Evidence method \citep{eernissetaxonomic1993} in the last three years \citep{ronquista2012,slaterphylogenetic2013,Wood01032013,schragocombining2013,beckancient2014,Arcila2015131,Dembo2015}.
This method allows to combine both molecular data for living species and morphological data for both living and fossil species into phylogenies.
Doing so allows mainly to treat palaoentological and neontological equally, which is a first step for combining methods, data and disciplines.
However, because of the amount of data needed, this method is likely to suffer from missing data

\subsection{Effects of missing data on topological inference using a Total Evidence approach}
In the second chapter of this thesis I tackled the problem of missing data in Total Evidence matrices.
I ran long term and thorough simulation to test how robust are our phylogenetic inferences when we combine living taxa with molecular and morphological to fossil taxa with morphological data only.
I particularly focused on how missing data in both living and fossil taxa can affect topology.
I found that the number of living taxa with available data is essential to recover accurate topologies.
Therefore we need data for living species.
But this rises the question: how much of it is out there?

\subsection{Morphological data availability in living mammals}
Following this question, in the third chapter of my thesis, I looked at how many data was available in mammals.
Following these results, I was interested in showing practical implications of this effect and monitored the morphological data availability for living mammals.
I downloaded all the recent available morphological matrices and counted the number of living mammals with available morphological data.
I then tested how these taxa where distributed accross the phylogeny to check if there weren't clustered in some specific clades.
I found that a lot of data is missing but that at least most of it is randomly distributed and should not drastically effect topology.
Since data in mammals is improvable, but is ok at higher taxonomic levels, it is an excellent candidate group for building Total Evidence phylogenies to allow macroevolutionary studies including both living and fossil species.

\section{Total evidence phylogenies applications}
These trees can allows use to capture macroevolutionary or macroecological patterns more accurately and therefore propose more solid hypothesis on processes.
Slater and Beck have successfully build Total Evidence and tip-dated phylogenies.
We can use these phylogenies for answering many question such as body mass evolution (Slater) or timing of diversification (beck) and that improves the whole yoke.
Another interesting we can do with such phylogenies is to look at diversity through time more accurately.

\subsection{Cretaceous-Palaeogene extinction does not affect mammalian disparity}
% Cheesy quote for that bit "The most erroneous stories are those we think we know best - and therefore never scrutinize or question." Gould whenever
One interesting point about diversity it that it doesn't has to be just species richness but sometimes disparity can be important as well.
We can use both processes plus Total Evidence trees to better describe the macroevolutionary patterns.
These more accurate patterns can be used to explain the processes driving diversification or extinction during a mass extinction event, we need to accurately measure what's happening.
One classical example is the K-T extinction where the effects still remain unclear after so many years of research.
In the fourth chapter, I explore this question using Total Evidence trees and focusing on disparity rather than species diversity to see if mammals were affected by the K-Pg extinction event.
I found that mammals do not do a damn thing around the K-Pg boundary.

\section{Discussion} % TG: I think if the title stays as shitty, no need for a title.
This is just an example on how including both living and fossil species can change our vision of biodiversity.
In the last chapter, I will discuss potential more application but also problems that arise with such methods

\bibliography{References}