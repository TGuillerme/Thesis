\chapter*{Summary}
\chaptermark{summary}
\addcontentsline{toc}{chapter}{Summary}

Even if most of our current knowledge and tool-kits to study biodiversity focus on living species, the vast majority of the species that ever lived are long extinct.
Therefore, to properly understand the drivers of biodiversity through time, it is crucial to combine data and methods from both living and fossil species in order to better assess macroevolutionary and macroecological patterns.
This thesis focus on ways to combine both living and fossil species into phylogenies and investigates how these phylogenies can be used for accurately describing macroevolutionary patterns.
I studied how to use both living and fossil species along two axes: firstly, the ability of modern phylogenetic methods to deal with molecular data for living species and morphological data for both living and fossil species; and secondly, the practicality of using the resulting phylogenetic trees for more accurately describing patterns of diversification through space and time.

For the first part of this project, I ran extensive and thorough simulation analyses to test the effect of missing data on phylogenetic topologies when using jointly living and fossil data.
I tested how multiple levels of missing data among living species, fossil species and the two combined affected our ability to recover the correct tree topology.
I found that the amount of missing data among living species is the most crucial aspect for efficiently combining living and fossil species in the same phylogeny.
Following these conclusions, I performed a thorough review of the data available for living mammal species.
I measured the amount of morphological data available within each mammalian order and tested whether this data was randomly distributed along the phylogeny or biased towards certain clades.
The result of this analysis shows that although morphological data is scarce for living mammals, it is at least generally randomly distributed across the phylogeny.

For the final part of this thesis, I explored a way of using these phylogenetic trees containing both living and fossil species to measure patterns of diversification among mammals through time.
I measured changes in species richness as well as in morphological diversity (i.e. disparity) to describe the patterns of mammalian diversification across the infamous Cretaceous-Palaeogene (K-Pg) mass extinction event, 66 million years ago.
I found that, even though many terrestrial vertebrates went extinct, the K-Pg event had no significant effect on mammalian morphological diversification.