\chapter*{Summary}
\chaptermark{summary}
\addcontentsline{toc}{chapter}{Summary} % NC: Do trinity ask for a summary? It would be more usual to call this the Abstract. TG: To be honest I don't know. I took Rich's template and Adam seems to say TCD wants summary not abstract. Don't know what that's based on though. Easy enough to fix at any time though.
Although many biodiversity studies focus on living species, the vast majority of species that ever lived are long extinct.
It is therefore crucial to combine data from both living and fossil species to fully understand macroevolutionary patterns and processes.
This thesis focuses on ways to combine both living and fossil taxa into phylogenies and investigates how the resulting phylogenies can be used to investigate macroevolutionary questions.

In the first part of the thesis, I ran extensive simulation analyses to test the effect of missing data on phylogenetic topologies when using the Total Evidence method.
This method builds phylogenies using both molecular data for living taxa and morphological data for living and fossil taxa.
I tested how various proportions of missing morphological data among living taxa, fossil taxa, and the two combined, affected my ability to recover the correct tree topology.
I found that the amount of missing morphological data among living taxa was the most crucial aspect for accurately placing living and fossil taxa in the same phylogeny.
Following these conclusions, I performed a systematic review of the coded morphological data available for living mammal species.
I recorded the amount of morphological data available for each mammalian order and tested whether this data was randomly distributed across the phylogeny or biased towards certain clades.
The results of this analysis showed that although morphological data is scarce for living mammals, it is at least generally randomly distributed across the phylogeny and therefore would not biase the placement of fossil taxa towards particular clades. %NC: Why is this important?

For the second part of the thesis, I used Total Evidence phylogenies containing both living and fossil taxa to investigate whether mammals radiated during the Cenozoic in response to the infamous Cretaceous-Palaeogene (K-Pg) mass extinction event, 66 million years ago.
Until now, this question was debated with support from an effect of the K-Pg event from palaeontological data but none from neontological data.
I used a novel time-slicing method for quantifying changes in morphological diversity (disparity) through time to describe the patterns of mammalian diversification across the K-Pg boundary.
I found no significant difference in disparity before and after the K-Pg boundary.
This suggests that, even though many terrestrial vertebrates (including the non-avian dinosaurs) went extinct during the K-Pg extinction event, it had no significant effect on mammalian morphological diversification.
% NC: You might like to add to this using the abstract and conclusion/discussion of the STD chapter. It's a bit light on methods and short compared to the previous paragraph. TG: I've modified a bit above (underlining the palaeo/neonto debate for later on). Also might be good to add the sentence below but it makes the abstract a bit long.
%These results refute a popular believe suggesting that mammals could only start diversifying after the extinction of the dominant non-avian dinosaurs and shows the advantage of combining living and fossil species for solving macroevolutionary hypothesis.

% NC: This needs a concluding paragraph to summarize what you found, future directions, implications etc. Can be quite short. Basically = using fossils and living species is a good idea! Maybe leave til you've finished this part of the overall discussion. TG: how's that?
Finally, I discuss future avenues of research for improving the problems with combining living and fossil species (e.g. the missing data issue) as well as the advantages of such combinations on investigating macroevolutionary questions (e.g. the effect of the K-Pg event on mammalian diversification).
I argue that including both sources of data in macroevolutionary studies is a really promising way to increase our knowledge in biology in general.



