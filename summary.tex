\chapter*{Summary}
\chaptermark{summary}
\addcontentsline{toc}{chapter}{Summary}

Even if much of our current knowledge and tool-kits to study biodiversity focus on living species, the vast majority of the species that ever lived are long extinct.
To properly understand the drivers of biodiversity through time, it is crucial to combine data and methods from both living and fossil species in order to fully understand macroevolutionary and macroecological patterns.
This thesis focus on ways to combine both living and fossil species into phylogenies and investigates how these resulting phylogenies can be used for accurately describing macroevolutionary patterns.

For the first part of this project, I ran extensive and thorough simulation analyses to test the effect of missing data on phylogenetic topologies when using the Total Evidence method.
The Total Evidence method allows to use both molecular data for living species and morphological for living and fossil species.
I tested how multiple levels of missing morphological data among living species, fossil species and the two combined affected our ability to recover the correct tree topology.
I found that the amount of missing morphological data among living species is the most crucial aspect for efficiently combining living and fossil species in the same phylogeny.
Following these conclusions, I performed a thorough review of the data available for living mammal species.
I measured the amount of morphological data available within each mammalian order and tested whether this data was randomly distributed along the phylogeny or biased towards certain clades.
The result of this analysis shows that although morphological data is scarce for living mammals, it is at least generally randomly distributed across the phylogeny.

For the final part of this thesis, I explored a way of using these phylogenetic trees containing both living and fossil species to measure patterns of diversification among mammals through time.
One debated question in mammlain history is whether they radiated during the Cenozoic in response to the infamous Cretaceous-Palaeogene (K-Pg) mass extinction event, 66 million years ago.
I measured changes in morphological diversity (i.e. disparity) to describe the patterns of mammalian diversification across the K-Pg boundary.
I found no significant difference in disparity before and after the K-Pg boundary.
This suggests that, even though many terrestrial vertebrates went extinct during the K-Pg event, it had no significant effect on mammalian morphological diversification.