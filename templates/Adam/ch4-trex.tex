\chapter{Did \textit{Tyrannosaurus rex} tend towards scavenging with age?}
\label{chap:trex}

%\thesischapter{Ontogenetic change in Tyrannosaurus rex did not lead to it adopting a scavenging niche}
\textit{Authors:} Adam Kane, Kevin Healy, Graeme Ruxton \& Andrew Jackson
% Adam Kane$^1$, Kevin Healy$^1$, Graeme Ruxton$^2$ \& Andrew Jackson$^1$

\vspace{10 mm}
\noindent
\textit{\uppercase{A}uthor contributions}
I conceived the idea, collected the data used for the scaling relationships, created and ran the IBM, interpreted the results and wrote the manuscript; KH collected the data used for the scaling relationships, calculated their output, helped with interpretation of results and commented on the manuscript; GR supplied data and commented on an early draft of the manuscript; AJ gave feedback on the manuscript. 


\vspace{10 mm}

\noindent
\textit{Status:} This manuscript is being prepared for publication. Target journal is Biology Letters.

\newpage

\noindent

\section{\uppercase{A}bstract}

The feeding ecology of \textit{Tyrannosaurus rex} remains controversial, with polarised suggestions of either a predatory or scavenging lifestyle. We move away from this dichotomy and propose the animal incorporated more carrion in its diet as it aged. We hypothesise this to be the case owing to factors related to its drastic morphological change across ontogeny notably an increased availability of carrion through dominance of interspecific competitors at carcasses; by exploiting resources unavailable to its smaller competitors such as skeletal material; and by having a larger capacity to store meat as it grew. We develop an individual based model to address our question. Surprisingly, our results show \textit{T.rex} was very unlikely to have specialised on carrion at any life stage because the cost of movement would have been prohibitively expensive as it increased in mass. In spite of our suggested advantages, scavengers require a more economical model of locomotion while foraging than \textit{T.rex} could have managed.



%Scavengers need an economical mode of locomotion to cover wide areas while foraging. This was not true for \textit{T.rex}.
\newpage

\section{\uppercase{I}ntroduction}

Animals can avoid direct competition with conspecifics by foraging at different times and locations, or by targeting different resources. This partitioning of resources can occur across ontogeny when the life stages are distinct; so, for example, an agile juvenile may focus on smaller, faster prey than a less nimble adult. Ontogenetic dietary partitioning is known across a diversity of species including crocodiles, insects, fish etc. \citep{platt2006food,winemiller1989ontogenetic,steyn1980breeding,hirai2002ontogenetic,knoff2008ontogenetic} and is often related to changes in body size \citep{woodward2002body} but can also arise through other means such as different habitat use patterns \citep{carrier2010sharks}.
Here we suggest that such partitioning may have taken place in the theropod dinosaur \textit{Tyrannosaurus rex}. \textit{T. rex} fits into theropod 'morphotype one' as defined by Weishampel \citep{weishampel2004dinosauria}. The carnivorous animals in this category are extremely large, exceeding 10 m in length, have huge skulls, short forelimbs and as such have no living analogues which makes it difficult to draw any conclusions about their mode of life \citep{weishampel2004dinosauria}. 
However, as a species, \textit{T.rex} is atypical because of the large number of fossils available for study \citep{brusatte2010tyrannosaur}. So, for instance, it is known that it exhibited remarkable rate of growth through its development \citep{brusatte2010tyrannosaur} and underwent dramatic ontogenetic changes in its morphology which has led some researchers to argue for a concomitant change in its ecological habits \citep{brusatte2010tyrannosaur}. For instance, adults had a powerful, deep, robust skull with thick teeth in contrast with the more gracile features of juveniles. One suggested result of this change was a dietary shift from fast prey items to larger more cumbersome herbivores \citep{brusatte2010tyrannosaur}. However, we instead explore whether such marked changes caused an increased tendency towards scavenging given that a scavenging lifestyle has been proposed for the animal in recent times \citep{horner1993complete, ruxton2004obligate}. \\
\indent
\cite{ruxton2004obligate} argue there is a "a clear evolutionary pressure towards large size for both birds and mammals when they are feeding on large carcasses. This is largely because larger animals can consume more food from each discovered food fall, and carry greater body reserves, than can small ones."
A pressure that lends support to our proposal. Indeed, a number of recent studies have looked at the theoretical possibility of obligate scavenging across species including \textit{T. rex} \citep{ruxton2004energetic,ruxton2013endurance,ruxton2005searching,ruxton2004obligate,ruxton2003could,carbone2011intra}. For example, Horner pointed to its slow speed, reduced forearms, large olfactory bulb and incredible bite force, to support this view \citep{horner1994steak,horner1993complete}. Brown hyenas (\textit{Hyaena brunnea}) can detect carcasses 2 km downwind \citep{mills1984comparative} and the olfactory bulbs of \textit{T. rex} indicate an impressive ability in this respect \citep{witmer2009new}. Fossil material from a herbivore has also been discovered with bite marks located in a flesh-poor region suggesting a scavenging event after the preferred sections of the animal were consumed \citep{longrich2010cannibalism}. Energetic approaches have concluded both for \citep{ruxton2003could} and against \citep{carbone2011intra} \textit{T.rex} being a scavenger. The latter study argued that interspecific competition would undermine the possibility of a scavenging \textit{T.rex} since smaller, more numerous competing species would find and consume any carcass before the larger \textit{T.rex} could benefit. Obligate scavengers are rare among terrestrial vertebrates with even the most common terrestrial example, the brown hyena, displaying a large degree of variation between scavenging behaviour and active predation \citep{devault2003scavenging,sinclair1995serengeti}.  Instead, most carnivores are opportunistic and will take carrion as well as actively hunting prey \citep{devault2003scavenging} and mounting evidence suggests that \textit{T.rex} behaved similarly to extant opportunistic predators such as hyenas. The most compelling evidence for a predatory lifestyle comes in the form of a subadult \textit{T.rex} tooth that was found embedded in a hadrosaur tail \citep{depalma2013physical}. The wound had healed around the tooth indicating that the prey item escaped and so this was an active predation attempt \citep{depalma2013physical}. \\
\indent
In this study we move away from the polarised predator-scavenger debate and look at the possibility that \textit{T.rex} underwent an ontogenetic dietary shift, increasing the proportion of carrion in its diet as it aged. We hypothesise this to be the case on the basis of increased availability of carrion through domination of interspecific competitors at carcasses, by exploiting resources unavailable to its smaller competitors and by having a larger capacity to store meat as it increased in size ontogenetically.  This shift would not only see adult \textit{T.rex} avoiding intraspecific competition with younger conspecifics but also with interspecific carnivores of the time. Among contemporary competitors from the Late Maastrichtian Hell Creek formation, the one tonne \textit{Albertosaurus} may have been an immature \textit{T.rex} and \textit{Troodon} and other dromaeosaurids were wolf-sized creatures that would be easily dominated by an adult \textit{T.rex} \citep{horner2011dinosaur,carr2004diversity,farlow2002body}. Their small size also means they would leave much of the skeletal material of prey/carcasses untouched. 
However, as alluded to earlier, the morphology of an adult \textit{T. rex} skull suggests an ability to process bone which would be a great boon to a scavenging animal if it were capable of monopolising this resource. Direct evidence comes in the form of distinctive wear marks on its tooth apices \citep{farlow1994wear,schubert2005wear}. The animal also had an enormous bite force, with one estimate putting it at 57000 Newtons \citep{bates2012estimating}. This is noted as being powerful enough to break open skeletal material during feeding \citep{rayfield2001cranial}. A positive allometric scaling relationship in bite performance during ontogeny has also been recorded \citep{bates2012estimating}. This is a similar pattern to that observed in loggerhead sea turtles (\textit{Caretta caretta}) where the bite force exhibits positive allometry across ontogeny allowing adults access to hard benthic prey, a resource inaccessible to other durophagous competitors \citep{marshall2012ontogenetic}. Further, \textit{T. rex} coprolites were discovered with bone fragments, prima facie evidence that it did consume bone \citep{chin2003remarkable}. Osteophagy is known in extant taxa such as the Bearded Vulture (\textit{Gypaetus barbatus}) and hyena species \citep{hone2010feeding}.  Some fat-rich mammalian bones have an energy density (6.7 kJ/g) comparable with that of muscle tissue \citep{margalida2008bearded,brown1989study}, making skeletal remains an enticing resource for any scavenger that could process them.  \\
\indent
We use an energetics method based on a series of scaling relationships to look at how the energy balance of \textit{T.rex} changed across ontogeny and test whether this could have pushed it towards obligate scavenging. We parameterise an agent-based model which better captures the impact of competition than the numerical approaches of previous studies. 

\section{\uppercase{M}ethods}
\subsection{Model parameters}


We took estimated parameter values from the literature for mass (kg), hip height (cm), detection distance (km), search time (fraction of day), energy content of bone (kJ/kg), energy content of flesh (kJ/kg), density of carcasses (kg/km$^2$/day), the size of competitors and the size of carcasses etc. to estimate several scaling relationships describing theropod foraging behaviour including walking speed (m/s), basal metabolic rate (W) and cost of transport (J/m). This is basically an 'energy in' minus 'energy out' calculation i.e. net energy gain (table \ref{tab:model_param}). 

%------------------------------------------

\begin{table}[H]
\small %!htb keeps the table in this section before moving onto the next block of text
		\caption[Model Parameters] %This goes into  your list of tables
				{Model parameters and their values for \textit{T.rex} foraging models.  
} 
		\input{chap4/tables/model_param}
		\label{tab:model_param}
	\end{table}

%------------------------------------------

The daily costs of theropods of various masses were derived using the scaling relationship of mesothermic dinosaurs from \cite{grady2014evidence} for basal metabolic rates. We used these for the \textit{T. rex} ontogenetic stages. We then took the scaling relationships outlined in \cite{pontzer2009biomechanics} to calculate the cost of searching during foraging. This was estimated using the mass-specific locomotor cost of transport that is based on the hip height of the animal, its mass and its speed to calculate the cost of searching over a 12 hour day (table \ref{tab:model_param}). Specifically, the cost of transport, in joules, is the energy required to move 1 kg of mass over 1 metre. From this we can calculate the cost for the whole animal: cost of transport x mass x walking speed (table \ref{tab:scav_mass}). 

We used an estimate of energy density of carrion from the Serengeti as an analogue for terrestrial Mesozoic systems \citep{ruxton2003could,carbone2011intra}. The amount of food available is derived as a function of the energetics of the food source. The energy available from the carcasses in flesh and bone was calculated using allometric scaling relationships (table \ref{tab:model_param}). We used the size bins of a previous study for both the body mass of the prey items and the predators (table \ref{tab:dino_mass}) \citep{carbone2011intra}. So for example, 49.3\% of carrion mass is packaged up in 75 kg carcasses. \cite{carbone2011intra} use mass-abundance relationships (e.g. Damuth's Law) to determine the actual population size of the various species and we followed their approach here. 

%------------------------------------------

\begin{table}[H]
\small %!htb keeps the table in this section before moving onto the next block of text
		\caption[Dinosaur carcass sizes] %This goes into  your list of tables
				{Mass categories for carcasses and allometric scaling of bone with body mass across carcass categories. The contribution column defines the percentage of carrion that is made up by a given species category.  
} 
		\input{chap4/tables/dino_mass}
		\label{tab:dino_mass}
	\end{table}

%------------------------------------------

\begin{table}[H]
\small %!htb keeps the table in this section before moving onto the next block of text
		\caption[Scavenger mass categories] %This goes into  your list of tables
				{Mass categories, gut capacities, walking speeds and daily energetic cost for the ontogenetic stages of \textit{T. rex} and its competitors as used in the individual-based model.  
} 
		\input{chap4/tables/scav_mass}
		\label{tab:scav_mass}
	\end{table}

%------------------------------------------

\subsection{Individual-based model}
We created a spatially explicit agent-based model using the above calculated parameters to allow us better understand the effect of competition on the ontogenetic stages of \textit{T.rex} if they were restricted to scavenging (table \ref{tab:dino_mass}). Our model was designed in the program NetLogo \citep{tisue2004netlogo}. The simulation space was a 50x50km square corresponding to a 2500km$^2$ landscape. Unfortunately, it is not clear if there was a habitat partitioning across ontogeny in Tyrannosaurus which may have implications for factors such as transport costs \citep{ruxton2014energetic}. 

The model had a series of mobile agent types which corresponded to \textit{T.rex} ontogenetic stages and its competitors (table \ref{tab:scav_mass}). There was one \textit{T.rex} ontogenetic stage per run e.g. a 100 kg individual. The initial state of the model had all dinosaurs and carcasses located randomly in the environment. The dinosaurs then set off in a random direction at their assigned speed searching for carrion (table \ref{tab:scav_mass}). They maintained the same walking speed for the duration of the model and had a constant turning rate. Upon finding a carcass in its visual field (500 m) the \textit{T.rex} walked towards it and started to feed, extracting energy until sated as defined by its gut capacity (table \ref{tab:model_param}) or until the carcass was consumed. The competitors behaved similarly only they were incapable of feeding on the bones of carcasses. If nothing remained for the animals to eat and they were not entirely sated they began to forage again. If they were full they moved away from the carcass and wandered until the next day whereupon their gut capacity went back to 0. The gut passage time of \textit{T.rex} is difficult to ascertain but we know it was relatively quick given the state of consumed material in its coprolites \citep{chin1998king,chin2003remarkable}. Without being fed upon, a carcass decayed down to bone and then to nothing because of mammalian scavengers, invertebrates, bacteria etc. \citep{sinclair1995serengeti}. This process took seven days \citep{carbone2011intra}. Competition in our model followed a simple, 'if bigger than me, then avoid' rule, and applied to the interaction between the \textit{T.rex} life stages and their competitors. A foraging day in our model was 12 hours. At the end of the day the 'energy in' of the \textit{T.rex} was subtracted from the 'energy out' (table \ref{tab:scav_mass}). We ran the model for 200 days for each of the seven \textit{T.rex} life stages.  


\section{\uppercase{R}esults}
Our results show that under the conditions of our model \textit{T.rex} was in negative energy balance if it were an obligate scavenger. We can see from figure \ref{fig:trex_energy} that, in contrast to our hypothesis, the net energy intake of \textit{T.rex} from scavenging decreases with their body mass. The horizontal dashed line represents the point where the energy out is balanced by the energy in. Even the average of the smallest ontogenetic stage (a 100 kg individual), falls below this line. 
%------------------------------------------

\begin{figure}[H]
\centering
\includegraphics[keepaspectratio, totalheight=0.7\textheight]{chap4/figures/trex_energy}
		\caption[\textit{T.rex} energy balance] %This goes into  your list of tables
				{The energy balance of the \textit{T.rex} ontogenetic stages. The dashed horizonal line represents the point, below which, the animal has a negative energy balance. Each point in the figure is the mean of the 200 day model run for the given body size with its standard deviation.   
} 
		\label{fig:trex_energy}
	\end{figure}

%------------------------------------------

\section{\uppercase{D}iscussion}
In order to be an effective terrestrial vertebrate scavenger an animal must be able to traverse huge distances at low cost \citep{ruxton2004obligate}. This is why soaring flight allows vultures to occupy the obligate scavenger niche. The cost of locomotion seems to be one of the main barriers precluding \textit{T.rex} from being an obligate scavenger. A lot of energy would have been required to propel an eight tonne adult around its habitat in search of a patchily distributed resource. The increasing variation in energy balance seen in figure \ref{fig:trex_energy} is a result of gut capacity and carcass size distributions. The smaller individuals can fill their gut to capacity every time they encounter a carcass; by contrast the larger animals will only rarely encounter a carcass of sufficient mass to sate them entirely. As such it would be interesting to investigate the effect of different carcass mass distributions on \textit{T.rex} scavenging behaviour. Our model is limited in depending on data from long extinct ecosystems. Consequently there are many unknowns. We do not know how carrion was distributed for instance or if there was habitat partitioning across the ontogeny of \textit{T.rex}. The costs of movement for the \textit{T.rex} stages are based on the most recent studies of dinosaur metabolism \citep{grady2014evidence} but it is unlikely that we will ever know the species level adaptations that some animals had such as the ability of vultures to decrease their metabolic rate at night \citep{ruxton2002modelling}. However, the model is flexible and sensitivity analyses can be readily performed to account for the large margin of error when dealing with prehistoric systems. A development of the model could allow for differing detection radii and turning rates for example.  \\
\indent
Ecologically, \textit{T.rex} was unusual. Census records show it to have had a population higher than would be expected for a typical apex predator suggesting that its mode of life was atypical with Horner and colleagues positing a hyena-like niche for the animal \citep{horner2011dinosaur,mills1984comparative}. The radical change in morphology with ontogeny would then appear to represent a dietary niche shift which has been suggested previously \citep{brusatte2010tyrannosaur}. The change from a gracile to robust form may have allowed the animal to switch its targets from small prey items to the huge armoured herbivores of the time (table \ref{tab:dino_mass}). Species such as \textit{Triceratops} and \textit{Ankylosaurus} would have made formidable opponents for a predaceous \textit{T.rex}. To be able to dispatch such prey items quickly with its enormous bite force would be of great advantage. It has been suggested that an ambush strategy seems the most probable gambit it used to hunt \citep{krauss2013biomechanics}. It could rush from cover and use its arms to hold a prey item while biting it \citep{krauss2013biomechanics}. This is not to discount scavenging entirely. Such was its likely competitive dominance \textit{T. rex} may well have been an effective kleptoparasite. Spotted hyenas (\textit{Crocuta crocuta}) can get over a quarter of their carcasses by stealing from other predators \citep{curio1976ethology}. 
Although we have framed our study such that it focuses on \textit{T.rex}, the same energetic barriers likely prevented other theropods in the morphotype one category \citep{weishampel2004dinosauria} from evolving into scavenging specialists. 


