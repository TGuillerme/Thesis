\chapter{Discussion}
\label{chap:discussion}

\section{\uppercase{H}ow the environment selects for scavengers}
The resource environment of a scavenger is a patchily distributed one, because it is difficult to predict both when and where a carcass is produced. As a result of this, any animal existing as a scavenger must maximise its detection capabilities and minimise its locomotory costs. The ease with which natural selection pushes an animal towards these optima is dependent on the environment the species is in. Land-based scavengers can be thought of as existing in a 2-dimensional plane while foraging for carrion directly. The range at which they can detect carcasses is thus defined by the radius of their sensory organs, usually the visual and olfactory senses which is further modulated by the habitat type. By flying, vultures have effectively added an extra spatial dimension, i.e. the vertical component, to their sensory environment. This allows them to look down on a landscape where they are unencumbered by obstacles that would obstruct the view of a terrestrial scavenger. Interestingly, animals like hyenas can and do take advantage of the directed flight of vultures because it indicates they have discovered a carcass \citep{sinclair1995serengeti}. But this is an indirect method of detection and, as I showed in chapter 2, scrounging is also utilised by vultures. Vultures have a further advantage in improving their encounter rate in terms of cost of movement. In general, flight is a cheaper means of locomotion than running \citep{tucker1975energetic}, but vultures extend this advantage further by engaging in soaring instead of flapping flight, which is even cheaper energetically \citep{hedenstrom1993migration}. Thus, as we saw from the results of chapter 1, they can cover huge distances over the course of a day, something I showed was impossible for \textit{T.rex} in chapter 4. \\ \indent In ancient ecosystems, the volant pterosaurs have also been postulated as occupying a vulture-like niche \citep{witton2008reappraisal}. These animals could reach enormous sizes and look to have engaged in soaring flight \citep{witton2008reappraisal}. So they certainly existed in an environment that can support scavengers. However, the inflexibility of their necks and straight, rather than hooked jaw morphology argues against their existing as obligate scavengers \citep{witton2008reappraisal}. As yet, no one has ever attempted an energetics approach for this group. This is likely due to the many uncertainties over their biology \citep{witton2010size}; which is even more difficult to predict than that of the dinosaurs. \\ \indent It is worth noting that the existence of an obligate scavenger in a marine setting also remains hypothetical \citep{britton1994marine,ruxton2004energetic}. Carrion in this environment is produced by dead fish and marine mammals when their carcasses descend to the sea floor. This low-light environment means animals detect resources through chemo- and mechanoreception \citep{ruxton2004energetic}. Detection distances are far lower than they would be in the air (< 100 m) as a result. However, water is a medium that is conducive to low-cost movement \citep{tucker1975energetic} and so may be able to support a small obligate scavenging fish which would gain the further benefit of low-cost ectothermic metabolism \citep{ruxton2004energetic}. Although, for the time being this remains conjectural. \\ \indent In sum, it seems that aerial foraging allows for a better medium for locomotion and grants access to more information on the location of resources. Natural selection has driven vultures to occupy a niche in which only the most specialised of species could exist.  


\section{\uppercase{A}ll models are wrong}
My thesis has a variety of models, which focus on different biological scales, from the individual to the population. I would like to discuss my philosophy of modelling in light of these methods. \cite{norgaard1989case} argues the case for methodological pluralism when we are dealing with complex systems. And this seems to me to be the right approach for ecology which is clearly a study of complexity. Of course no one person can be an expert in all methods but the collaborative nature of science means this is easily compensated for. The statistician George Box wrote "essentially, all models are wrong, but some are useful" \citep{box1987empirical}. What he meant by this is when we model a system, we necessarily abstract and simplify so we can get at the question we seek to address. A one to one representation of a system would be the system itself! \\ \indent That we can address questions on the feeding ecology of long extinct dinosaurs speaks to the advantages of modelling. By drawing sensible analogies to modern day systems and carrying out a sanity check on the model parameters we can come to a reasonable conclusion that \textit{T.rex} was not an obligate scavenger. The complexity needed in a model is dependent on the hypothesis one is looking to explore. So the question of the necessity of supplementary feeding for vultures in chapter 3 called for a relatively complex model because there is an applied relevance. That said, we should always invoke the principle of parsimony when selecting the most suitable method for the question at hand. This is especially important for papers with an application outside of pure theoretical advancement of the field, like in conservation, where we want as many people to understand the implications of the research as possible \citep{kane2012suggestion}.  

\section{\uppercase{T}he value of data}
It still remains that a model is only as good as the data used to create it. 'Garbage in garbage out' is worth remembering. This is an issue given some of the most fundamental biological data on scavengers are still missing such as their life expectancies or metabolic rates. In general, scavengers are often discounted from descriptions of ecosystems \citep{devault2003scavenging} with \cite{wilson2011scavenging} calculating a 16 fold underestimation of scavenging in food-web research. Specific to this work, information on the life history of many vulture species are still unknown \citep{de2009database} and there is a geographical bias in that most studies on vulture systems are based in Africa \citep{moleon2014inter}. This has to be reversed if we are to have any success in conserving them. Indeed, in India there has been a move to build population viability models in order to determine if the \textit{Gyps} species are recovering now that diclofenac has been prohibited \citep{baral2013population}. Yet, we do not have data on age-specific survival rates for the birds limiting the scope of the models that have beeen created \citep{baral2013population}. The authors of that study have called on others to establish vulture banding projects to improve on this deficiency \citep{baral2013population}. 


I was fortunate that data of such quality were available to populate the models in chapter 3. The creation of similar models for threatened vulture populations would be a great boon to their conservation. The steady advance of technology does give us reason to be hopeful. Trackers can now record spatial data at a temporal and spatial scale impossible ten years ago. So, for example, it is now feasible to retrieve information on the behaviour of the animals remotely \citep{nathan2012using,tomkiewicz2010global} allowing us to investigate interesting questions such as use of supplementary feeding sites \citep{monsarrat2013predictability}. We can also monitor biodiversity by means of satellite remote sensing \citep{JPE:JPE12261}. Though there is still plenty of room for field work to ground truth these observations. The digital age with its big data should see the gaps in our basic understanding of these species being filled. 


\section{\uppercase{C}onservation prospects}
  The identification and description of the producer-scrounger system in chapter 2 highlights the problem of focusing on one species in a study system. I would argue that this is a clear example of why conservation needs to take place at the level of the ecosystem. We are dealing with systems whose individual components interact, creating an emergent complexity. A species does not exist in isolation and consequently the best way to sustain it is to maintain its environment. Of the 23 vulture species, 14 are in danger of extinction \citep{ogada2012dropping}. Although there are many causes of their decline \citep{mundy1992vultures}, poisoning, both accidental and from direct persecution, has been identified as the main agent responsible. It is their very nature that renders vultures particularly susceptible to poisoning given they forage and feed communally \citep{ogada2012dropping}. A single poisoned carcass can wipe out over a 100 birds \citep{mundy1992vultures}. The clear cut case in India of the effect of diclofenac makes the recent authorisation of its use in Spain all the more exasperating \citep{COBI:COBI12271}. As I mentioned earlier the conservation of such a wide ranging species has to occur internationally but this is predicated on sensible national decisions. In addition to direct human-caused threats such as poisoning, climate change may add another threat to obligate scavengers. Of particular note is the increase in decomposition rate of carrion with temperature which could result in a seasonal reduction of carrion availability given predicted warming trends \citep{moleon2014inter}. Unfortunately the low reproductive rates of these birds means their populations are particularly sensitive to increased mortality \citep{ogada2012dropping}. 

\section{\uppercase{F}uture directions}

  The idea of having a big picture also applies to the science of scavenger ecology in general. \cite{moleon2014inter} call attention to the inter-specific interactions linking predators, live prey, vultures and carrion production in terrestrial ecosystems. Such is their efficiency at consuming carrion, vultures may indirectly affect ungulate populations in forcing carnivores to hunt i.e. by denying them an easy meal. A relationship that was revealed in India only after the crash of its vulture population \citep{moleon2014inter}. But vultures could also reduce pressure on potential prey items by acting as producers of carcass location to mammalian scroungers. And yet, there has never been a quantitative study examining the effect vultures have on ungulate numbers \citep{moleon2014inter,elbroch2013nuisance}. All of this is to emphasise the connectedness of species interactions in an ecosystem from which vultures are no exception. \\
  \indent Supposing we can maintain their numbers, I want to expand here on one of the most important questions in vulture behavioural ecology. That is \textit{how do vultures forage as a group?} Because of their dependence on thermals for flight, vultures are often concentrated in relatively tight clusters around these hot air pockets \citep{xirouchakis2009foraging}. This means one bird can track the descent of another to a carcass i.e. a local enhancement effect is at play. However \cite{houston1974food} argued that the directed movement of this second bird would cause a third to follow it and so on, causing a chain reaction that could pull in birds from up to 35 km away. He suggested that this would explain the sudden arrival of so many individuals to the carcass in a short space of time. By contrast \cite{cortes2014bird} found the numbers of birds arriving at a carcass were better explained by the initial local enhancement effect. By tagging a large group of the birds during their breeding season so they would forage from the same location and tracking them at high resolution we could properly address how the birds find carrion. As \cite{cortes2014bird} note these hypotheses have different implications for conservation management because if birds form a chain they will be far more sensitive to population declines. Hopefully with the cost of such technology decreasing this will be achievable in the near future. These studies focus on the highly social \textit{Gyps} vultures but there is considerable variation in foraging behaviour outside of this genus. A large scale comparative analysis of vulture foraging behaviour is lacking. In addition to improving our understanding of vultures a cross species comparison of habitat use would mean we could more easily conserve these species. Behavioural differences within vultures are only recently being discovered but there does seem to be a significant variability in the search efficiencies of species \citep{spiegel2013factors}. Again, the conservation implications of this is significant. Notably, supplementary feeding will impact birds that intensely forage in relatively small areas (e.g. Lappet-faced Vultures) differently to non-territorial wide-ranging species (e.g. African White-backed Vultures) \citep{spiegel2013factors}. \\
 \indent 
It seems to me that vultures are well-placed to benefit from citizen science \citep{silvertown2009new}. Public campaigns are often run in southern Africa to encourage people to report on resightings, creating data which can be used to ascertain the health of the bird population \citep{monadjem2014effect}. That some section of the public are interested in supporting and being involved in science is beyond dispute. One need only look to the huge sums of money donated to crowd-sourcing initiatives where the incentive to donate is little more than pure interest in the study. Including an interested lay public in science has the effect of promoting awareness about the species too. In the case of vultures this would be a tremendous boon for their PR, often reviled as they are for the most superficial of reasons.


 %Although predators are known to exert top down control on an ecosystem because they actively kill their prey, obligate scavengers can exert a similar effect. By denying 

 % Adaptations of morphology and behaviour in light of environmental constraints; how does the environment affect the ability to exist as a scavenger? Vultures leave the 2D landscape that makes obligate scavenging so difficult for land animals; deep sea system; pterosaurs as potential scavengers in ancient systems 

%Among the Cathartidae, Turkey Vultures are known to have a highly attuned olfactory system that allows them to detect odours from decaying carrion while flying.  