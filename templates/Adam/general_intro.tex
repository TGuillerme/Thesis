\chapter{General introduction}
\label{chap:intro}
\section{\uppercase{T}he biology of scavengers}
Foraging for food is a primary concern for animal life \citep{danchin2008behavioural} with the diet of a species often apparent from its morphology. Consider carnivorous animals that, in general, have to find, stalk and kill their prey. Take the example of a cheetah which has a series of adaptations that speak to its carnivorous lifestyle such as its canine teeth, agility and acceleration \citep{bissett2007habitat}. We would expect obligate scavengers, that are completely dependent on carrion resources, to be unusual in their adaptations because, while they too have to be able to consume flesh, they have no need to hunt or subdue prey. Rather, they must cover an area large enough to encounter a sufficient number of carcasses which are, by their very nature, patchily distributed. 

As a group, scavengers are understudied \citep{sekercioglu2006increasing,selva2007nested,wilson2011scavenging}. \cite{devault2003scavenging} propose that this is due to both human disgust at carrion itself and the difficulty in determining if an ingested prey item was killed or scavenged. It is now recognised that scavengers have an important role in keeping energy flows at a higher trophic level in food webs than decomposers because they consume relatively more carrion \citep{devault2003scavenging}. (Here I consider decomposers as invertebrates and microorganisms whereas scavengers are typically vertebrates.) % A scavenging link in a food web also transfers more energy (e.g. in terms of of amount of carbon) from 'prey' to consumer than one of predation \citep{wilson2011scavenging}. 
Scavengers also provide useful ecosystem services by acting as barriers to the spread of disease by quickly consuming rotting carcasses which have often died from contagious illness \citep{ogada2012effects,devault2003scavenging}. 

Although many terrestrial and marine predators scavenge to some extent \citep{britton1994marine,devault2003scavenging,ruxton2004energetic} the rarity of modern day \textit{obligate} scavengers in the animal kingdom suggests this is a specialised niche to occupy \citep{ruxton2004energetic,ruxton2004obligate}. \cite{devault2003scavenging} argue that competition between scavengers and decomposers acts as the barrier to fully obligate scavenging owing to the high costs involved in detecting and detoxifying carrion. Certainly, scavengers need to be efficient at finding carrion soon after it has been produced (i.e. after the death of the animal from whatever cause) because toxic "products of decomposition" will render the flesh inedible with time \citep{devault2003scavenging}. The problem of decomposing resources is also felt by herbivores that have to contest with the microbes that spoil fruit in an effort to monopolise it \citep{ruxton2014fruit}. Some studies have looked at the theoretical possibility of obligate scavenging in species from ecosystems in the recent and ancient past \citep{ruxton2013endurance,ruxton2004obligate,ruxton2003could,carbone2011intra}, the latter being particularly revealing of the sorts of species that could inhabit the niche in the absence of vultures. For example, an energetics approach concluded the carnivorous dinosaur \textit{Tyrannosaurus rex} could have existed as an obligate scavenger \citep{ruxton2003could}. 

Indeed, the convergent groups of Old and New World vultures are the only representatives to inhabit this niche today \citep{houston2001condors}. These birds have a suite of adaptations that allow them to flourish as obligate scavengers. Their efficiency is illustrated by cases of predators like bears and wolves benefiting by taking more carrion in their diet in areas bereft of vultures through competitive release \citep{devault2003scavenging}. In flight, birds possess a huge advantage over any terrestrial obligate scavenger. Flight affords them the ability to range over a much larger area and detect carrion from an elevated vantage point. \cite{pennycuick1972soaring} conservatively estimated that a \textit{Gyps} vulture could identify activity at a carcass 4 km away. Further, vultures take advantage of soaring in thermals \citep{mundy1992vultures}. This economical means of locomotion, whereby the birds circle around pockets of hot air which provide uplift, allows them to cover huge distances at a low energetic cost (figure \ref{fig:scatter_home}). At large body masses flapping flight is prohibitively expensive in energetics terms \citep{hedenstrom1993migration}. These advantages underscore the importance of the environment, in this case an aerial one, in enabling a scavenging lifestyle. Relative to a terrestrial setting, both locomotion and resource detection are easier for aerial scavengers \citep{tucker1975energetic,ruxton2004obligate}. \\ \indent
\cite{ruxton2004obligate} argue that large body size is another adaptive response to scavenging. Cinereous Vultures (\textit{Aegypius monachus}) and condors (\textit{Vultur gryphus}, \textit{Gymnogyps californianus}) all have body masses that can exceed 10 kg and represent some of the heaviest bird species capable of flight \citep{ferguson2001raptors,donazar2002effects}. Many of the other 23 species of vulture are at the upper end of the scale in terms of body mass \citep{ruxton2004obligate}.  As carrion generally occurs in large 'packages', bigger birds can stock up and create body reserves which will benefit them during times of insufficiency. Large body mass also confers dominance advantages during agonistic interactions \citep{kruuk1967competition,KaneVul}. \\ \indent
Social behaviours have been theorised as further adaptations to improve foraging efficiency because they allow for active and passive information transfer \citep{jackson2011evolutionary,wakefield2013space,moleon2014inter}. Many vulture species, notably those in the genus \textit{Gyps}, display social behaviours such as communal nesting and group foraging \citep{mundy1992vultures}.  By nesting communally the birds are concentrated in space so at the start of the foraging day they will form foraging groups. This may lead to local enhancement effects such that one bird will follow another that descends to a carcass \citep{jackson2011evolutionary}.  \\ \indent Some of these adaptations mean vultures are incapable of killing prey themselves; for instance their wing morphology renders them far less agile than a raptorial counterpart \citep{ruxton2004obligate}. The selective pressures that push mammals and reptiles towards scavenging do not seem to undermine their ability to hunt in the same way, perhaps explaining the absence of obligate scavengers in these groups \citep{ruxton2004obligate}. 

\begin{figure}[H] %!htb keeps the figure in this section before moving onto the discussion
	  \centering
	  \includegraphics[width=0.8\textwidth,natwidth=610,natheight=642]{chap1/figures/scatter_home.pdf}
	    \caption[Comparative home range of birds] %This is the label in table of contents
	    {Comparative home range of birds (107 species from 11 orders), with \textit{Gyps} vultures highlighted in red, non \textit{Gyps} vultures in black and all other species in grey. The vulture species are \textit{Gyps corprotheres}, \textit{Gyps fulvus}, \textit{Aegypius monachus}, \textit{Cathartes aura}, \textit{Gypaetus barbatus}, \textit{Gyps bengalensis}, \textit{Gyps tenuirostris} and \textit{Neophron percnopterus}. Home range was calculated using minimum convex polygon methods. Data are log transformed \citep{peery2000factors,garcia2011ranging,gilbert2007vulture,kruger2014trends,vasilakis2005breeding,dwyer2010ecology,donazar1996communal,stroem2001home,buenestado2008habitat,gilbert2005behaviour,nesbitt1990home,pejchar2005hawaiian,springborn2005home,elchuk2003home,rhim2006home,legagneux2009variation,rolando1998factors,hoffman1991spring,dreitz2005movements,hansbauer2008comparative,garza2005home,vega2003home,stober2006variation,novoa2006home,holbrook2011home,fearer1999relationship,brandt2008breeding,giesen1992winter}.}%this is under the figure
	  \label{fig:scatter_home}
	\end{figure}

\section{\uppercase{C}onservation status}
Avian scavengers represent an ecologically important guild that has suffered severely from the effects of anthropogenic change \citep{ogada2012dropping} which is not helped by our ignorance of the group. The importance of vultures to ecosystem functioning is best illustrated by the sudden collapse of their numbers in the Indian subcontinent. By feeding on the carcasses of drug-treated cattle, the vultures were poisoned en masse by diclofenac \citep{oaks2004diclofenac}. Their absence led to an increase in feral dogs and a resultant spike in rabies' incidence among humans \citep{markandya2008counting}. Aside from poisoning, both targeted and incidental, vultures are endangered by wind turbines, electricity pylons, habitat destruction, food loss and poaching \citep{monadjem2003threatened,virani2011major,martin2012visual}. Despite these threats they have proved to be an adaptable group, illustrated by their willingness to feed on the carrion of domesticated animals \citep{mundy1992vultures} and their use of electricity pylons as roosting locations which has caused an expansion in their range \citep{phipps2013power}. Efforts at conserving vultures look to increase the availability of food \citep{piper2005supplementary}, rehabilitate injured individuals \citep{monadjem2014effect}, prevent the spread of poisons \citep{green2004diclofenac}, increase public awareness of the group's function etc. \citep{monadjem2004vultures}. However, the effect of implementing these strategies remains poorly understood \citep{monadjem2014effect}. We can take solace in the recent spike in research on vultures which has been brought about by high profile population crashes like that of India \citep{manga2006vulture}. Hopefully, such catastrophes won't be necessary to keep biologists focused on this group in the future. 

\section{\uppercase{R}esearch outline}
This thesis draws on a wealth of data collected by field researchers in Africa to answer questions on the foraging ecology of \textit{Gyps} vultures, moving from individual based methods to a study of population dynamics. The possibility of obligate scavengers in past ecosystems is an interesting area because it reveals the adaptations that permit an animal to exist in the niche. As such I use some of the theory and methods from my previous work in describing the feeding ecology of \textit{Tyrannosaurus rex} whose predatory tendencies have been called into question. 
\vspace{10 mm}

\textbf{Chapter 2}:
In order to effectively conserve species we must know the extent of their home range and how this can vary across time. Here I describe the home range of 29 Cape Vultures caught, released and tracked in southern Africa. My analysis shows the species range significantly farther as immature birds and, as adults, are constrained by the breeding season. Such variation means there is no easy solution to conservation efforts like supplementary feeding sites which need to be tailored to cater for all life stages. 
\vspace{10 mm}


\textbf{Chapter 3}:
As social birds, vultures often rely on each other to find food. They frequently follow the descent of another individual in the hope that it has discovered a carcass. In a Kenyan study site, I show that vultures also take cues from scavenging eagles. I propose that the eagles are used by the vultures in two ways: first they act as indicators of the presence of food; and second the eagles can tear open the hide of a carcass with their relatively stronger beaks, providing a resource that would otherwise be much more difficult for the vultures to access. In each case the larger vultures can then displace the eagles through competitive dominance and monopolise the remaining food. These newly identified social interactions among different species highlight the vital importance of applying integrated management strategies to conserve endangered vulture species.
\vspace{10 mm}

\textbf{Chapter 4}:
Carrion ecology, the fate of animal carcasses, is a crucial component of every ecosystem. Mismanagement can lead to environmental disasters, most notably the decimation of Asian vulture populations which fed on the carrion of drug-treated cattle. Here, I provide data on the carrion ecology of an ecosystem in Swaziland and predict the future trends of its vulture populations. Using novel methods I show that, despite a closure in supplementary feeding sites, these species have enough food from wild carcasses to survive. But only for the time being. Therefore, I recommend a dedicated vulture restaurant initiative to ensure the population survives.
\vspace{10 mm}

\textbf{Chapter 5}:
The feeding ecology of \textit{Tyrannosaurus rex} has long been debated with distinct 'predator versus scavenger' camps. I move away from this polarised debate and explore the effect of the drastic ontogenetic morphological changes the animal underwent. Specifically, I ask whether these changes resulted in it incorporating ever more carrion in its diet. To address my question I created an individual-based model based on the energetics of its movement and the amount of carrion available at the time only to find that the reverse is true. The cost of movement meant searching for patchily-distributed carrion became prohibitively expensive as \textit{T.rex} grew. Instead, my results suggest the dinosaur switched from small prey items to more cumbersome, armoured species as it developed into its robust adult form. 

\section{\uppercase{A}dditional work}
In addition to the chapters enclosed in this thesis, I have also been involved in the following research during my studies:\\
\begin{singlespace}
Healy, K., Finlay, S., Guillerme, T., Kane, A., Kelly, S., McClean, D., Kelly, D., Donohue, I., Jackson, A.L., \& Cooper, N. (2014). Ecology and mode-of-life explain lifespan variation in birds and mammals. Proceedings of the Royal Society B: Biological Sciences 281.1784: 20140298. \\
\end{singlespace}

\noindent
I was involved with the conception, data collection and write-up of this paper. \\
\begin{singlespace}
Wakefield, E.D., Bodey, T.W., Bearhop, S., Blackburn, J., Colhoun, K., Davies, R., Dwyer, R.G., Green, J., Gr{\'e}millet, D.,Jackson, A.L., Jessopp, M.J., Kane, A., Langston, R.H.W., Lescro{\"e}l, A., Murray, S., Le Nuz, M., Patrick, S.C., P{\'e}ron, C., Soanes, L., Wanless, S., Votier, S.C., \& Hamer, K.C. (2013). Space Partitioning Without Territoriality in Gannets. Science. 341, 68-70. doi: 10.1126/science.1236077 \\
\end{singlespace}
\noindent
	I took the lead in developing the individual based models for this paper and taught one of the lead authors (TWB) about the method, working closely with him in developing a suitable model for the gannet system. \\
\begin{singlespace}
Monadjem, A., Wolter, K., Neser, W., \& Kane, A. (2013). Effect of rehabilitation on survival rates of endangered Cape vultures. Animal Conservation. doi: 10.1111/acv.12054 \\
\end{singlespace}
\noindent
	I was again involved in the data analysis and write-up of this manuscript and forged some links with two vulture conservationists based in South Africa. \\
\begin{singlespace}
Monadjem, A., Kane, A., Botha, A., Dalton, D., \& Kotze, A. (2012). Survival and Population Dynamics of the Marabou Stork in an Isolated Population, Swaziland. PloS one, 7(9), doi: 10.1371/journal.pone.0046434 \\
\end{singlespace}
\noindent
	I was involved in the data analysis and write-up of this manuscript. This was my first full peer-reviewed paper. \\
\begin{singlespace}
Kane, A. (2012). A suggestion on improving mathematically heavy papers. Proceedings of the National Academy of Sciences. 109(45) E3058-E3059. doi: 10.1073/pnas.1212310109 \\
\end{singlespace}
	\noindent
	This letter was a response to an article highlighting the divide between theoretical and empirical                   biologists owing to different competencies in mathematics. 

