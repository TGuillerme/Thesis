\documentclass[twoside,12pt]{mythesis} %this is the mythesis.cls file
% twoside,openright,versioninfo

% Consider:
% \newcommand{\ie}{i.\,e.}
% \newcommand{\Ie}{I.\,e.}
% \newcommand{\eg}{e.\,g.}
% \newcommand{\Eg}{E.\,g.}
\usepackage{graphicx}
\usepackage{amsmath}
\usepackage{booktabs}
\usepackage{pdfpages} % new addition
\usepackage{xparse}
\usepackage{float}
\usepackage{setspace} % new addition

\usepackage{geometry} % new addition 
 \geometry{
 a4paper,
 total={210mm,297mm},
 left=35mm,
 right=20mm,
 top=20mm,
 bottom=20mm,
 }

% http://www.tug.dk/FontCatalogue/allfonts.html
\usepackage[T1]{fontenc}
\usepackage[default]{gillius}

%\usepackage[utf8]{inputenc}
%\usepackage{tgadventor} %% Added to change font
 


% \usepackage{xltxtra} % new addition
%\usepackage{cite}
% \usepackage{epigraph} can add an inspirational quote with this

 %\usepackage[round]{natbib}

%You might need to load other packages here...

% \RequirePackage{refstyle}
% \renewcommand{\figref}{\Figref}
% \renewcommand{\tabref}{\Tabref}

%% Symbols
% unresolved:
\newcommand{\ud}{\mathrm{d}}
\newcommand{\drel}{\ensuremath{r_{\mathrm{rel}}}}
\newcommand{\nmax}{\ensuremath{n_\mathrm{max}}}

%Information for the title page
%Some of this is hard coded in mythesis.cls but you can over write if you need to
\title{The \uppercase{e}cology of \uppercase{o}bligate \uppercase{s}cavengers from \uppercase{i}ndividual \uppercase{b}ehaviour to \uppercase{p}opulation \uppercase{d}ynamics}
%
\author{Adam Kane}
%
\month{\textsc{January}} \year{2015}
\previousdegrees{B.Sc.(Hons), University College Dublin, 2007\\%
  M.Sc., Dublin City University, 2010}
\degreetitle{Doctor of Philosophy}
\institution{Trinity College Dublin}
\school{School of Natural Sciences}
\department{Zoology}

%End of preamble. 

\begin{document}
%\raggedright

%-------------------------------------
% \fontfamily{qag}\selectfont

% added this to change font
%-------------------------------------
\maketitle %puts in your title

\chapter*{Declaration}
I declare that this thesis has not been submitted as an exercise for a degree at this or any other university and it is, unless otherwise referenced, entirely my own work.
I agree to deposit this thesis in the University's open access institutional repository or allow the library to do so on my behalf, subject to Irish Copyright Legislation and Trinity College Library conditions of use and acknowledgement.

\vspace{10 mm}

\noindent{Adam Kane}

\chapter*{Summary}
Studies on vultures are on the rise, and just as well given the sharp declines in many of the 23 species. Indeed it seems these population crashes are responsible for this research boost. However, there remain obvious gaps in our knowledge when it comes to the world's only terrestrial example of vertebrate obligate scavengers. It has been suggested that a human aversion to carrion is one of the reasons scavengers are reviled by the public and understudied by science. Broadly, the following body of work is an attempt to fill in some of these gaps. To achieve this I use a variety of methods and look at a range of species. I start by giving a description of the impressive ranging abilities of one species, the Cape Vulture, in Southern Africa, and identify causes of variation in these home ranges. I then move to Kenya where I identify a producer-scrounger game occuring between the \textit{Gyps} vultures and scavenging eagles of the area. I use a suite of empirical data and modelling approaches to show vultures use eagles in identifying the location of carcasses. The results of which underscore the importance of ecosystem-level management. Then in chapter 3, I collect data on the food required by the avian scavenging guild of Swaziland and use a novel method, namely population dynamics P systems, to ascertain if the wild fauna of the country can meet this need. Although sufficient for the time being carrion will soon become limiting and I advocate the creation of supplementary feeding sites in the country to counter this. Finally, I visit the Cretaceous Period where I use my understanding of modern day carrion ecology to explore a putative scavenger in \textit{Tyrannosaurus rex}, the infamous carnivorous dinosaur. A proposed hypothesis that \textit{T.rex} took on more carrion in its diet across ontogeny is shown to be false owing to the huge cost of movement as the animal grew. 

\chapter*{Acknowledgements}
The infamous 'they' warn that, with a bad supervisor, a PhD is a hellish experience. Thank God then for Andrew Jackson who made the previous three years of 'work' one of the most enjoyable periods of my life. The wonderful atmosphere of the zoology department certainly helped too. Luke McNally and Mafalda Viana guided me from my first year as the senior PhDs in the office. Deirdre McClean and Se{\'a}n Kelly were my early morning catalysts whose mania ensured no Monday morning was ever a drudge. While Kevin Healy's cynicism and creativity complemented this well. The academic odd couple of Thomas Guillerme and Sive Finaly along with their supervisor Natalie Cooper were always there to help me get unstuck. Joe Colgan deserves mention for advertising the department in such an appealing light; I was encouraged to start a PhD given his experience (Alain Finn take note). All the other members of NERD club, I thank you for stimualting discussions. \\
\indent
In the field Ara Monadjem is owed a debt for preventing mine on more than one occasion which made our collaborations a lot more successful. Mduduzi Ngwenya and the members of All Out Africa all helped to navigate me through my time in Swaziland. \\
\indent
Thanks too to my co-authors outside of Trinity College, Kerri, Walter, Graeme, Antoni, M.Angels, and Darcy. You provided the data and helped me with the analysis, all of which made this piece of work possible. \\
\indent
My mam and dad were responsible for getting me interested in science in the first place and for it they, along with my brothers Alex and Tom, have suffered/listened to someone who has no problem telling them how interesting it is. \\
\indent
Dank je wel to Gwendoline, we began at the same time as the PhD, and she has made my time outside of zoology just as enjoyable as my time spent within. 


%\chapter*{Abstract}
\chaptermark{abstract}
\addcontentsline{toc}{chapter}{Abstract}

Here is the abstract of my thesis.
%
It's pretty amazing as you can see. It features fieldwork and experiments and stuff! 
%

%%% Local Variables:
%%% TeX-master: "thesis"
%%% TeX-PDF-mode: t
%%% End:
 %inputs abstract.tex
%\chapter*{Preface} %the * removes the numbers
\addcontentsline{toc}{chapter}{Preface}

Several chapters from my thesis have been published elsewhere:

\lettherebespace
\textsc{\Chapref{velociraptor}} has been previously published as:
%
\begin{previouspaper}
  FitzJohn R.G., Maddison W.P., and Otto S.P. 2009. Estimating
  trait-dependent speciation and extinction rates from incompletely
  resolved phylogenies.  Systematic Biology 58:595--611.
\end{previouspaper}
%
Here I explain what I did for this paper, and what the other coauthors did. Can just repeat this for each chapter if all are published already or in prep. Basically for anything you do with other people.

 %inputs preface.tex
\allcontents %tells it to make a table of contents with figure and table lists too
%% There is currently a problem with spacing somewhere so that Table of
% Contents, List of Tables, and List of Figures have the wrong amount
% of space.  Others are OK though...
\chapter*{Acknowledgements}
\addcontentsline{toc}{chapter}{Acknowledgements}

I would like to acknowledge Dr Rich FitzJohn for letting me use his thesis template!

%%% Local Variables:
%%% TeX-master: "thesis.tex"
%%% TeX-PDF-mode: t
%%% End:
 %inputs acknowledgements.tex
\cleardoublepage
\mainbody

\chapter{General introduction}
\label{chap:intro}
\section{\uppercase{T}he biology of scavengers}
Foraging for food is a primary concern for animal life \citep{danchin2008behavioural} with the diet of a species often apparent from its morphology. Consider carnivorous animals that, in general, have to find, stalk and kill their prey. Take the example of a cheetah which has a series of adaptations that speak to its carnivorous lifestyle such as its canine teeth, agility and acceleration \citep{bissett2007habitat}. We would expect obligate scavengers, that are completely dependent on carrion resources, to be unusual in their adaptations because, while they too have to be able to consume flesh, they have no need to hunt or subdue prey. Rather, they must cover an area large enough to encounter a sufficient number of carcasses which are, by their very nature, patchily distributed. 

As a group, scavengers are understudied \citep{sekercioglu2006increasing,selva2007nested,wilson2011scavenging}. \cite{devault2003scavenging} propose that this is due to both human disgust at carrion itself and the difficulty in determining if an ingested prey item was killed or scavenged. It is now recognised that scavengers have an important role in keeping energy flows at a higher trophic level in food webs than decomposers because they consume relatively more carrion \citep{devault2003scavenging}. (Here I consider decomposers as invertebrates and microorganisms whereas scavengers are typically vertebrates.) % A scavenging link in a food web also transfers more energy (e.g. in terms of of amount of carbon) from 'prey' to consumer than one of predation \citep{wilson2011scavenging}. 
Scavengers also provide useful ecosystem services by acting as barriers to the spread of disease by quickly consuming rotting carcasses which have often died from contagious illness \citep{ogada2012effects,devault2003scavenging}. 

Although many terrestrial and marine predators scavenge to some extent \citep{britton1994marine,devault2003scavenging,ruxton2004energetic} the rarity of modern day \textit{obligate} scavengers in the animal kingdom suggests this is a specialised niche to occupy \citep{ruxton2004energetic,ruxton2004obligate}. \cite{devault2003scavenging} argue that competition between scavengers and decomposers acts as the barrier to fully obligate scavenging owing to the high costs involved in detecting and detoxifying carrion. Certainly, scavengers need to be efficient at finding carrion soon after it has been produced (i.e. after the death of the animal from whatever cause) because toxic "products of decomposition" will render the flesh inedible with time \citep{devault2003scavenging}. The problem of decomposing resources is also felt by herbivores that have to contest with the microbes that spoil fruit in an effort to monopolise it \citep{ruxton2014fruit}. Some studies have looked at the theoretical possibility of obligate scavenging in species from ecosystems in the recent and ancient past \citep{ruxton2013endurance,ruxton2004obligate,ruxton2003could,carbone2011intra}, the latter being particularly revealing of the sorts of species that could inhabit the niche in the absence of vultures. For example, an energetics approach concluded the carnivorous dinosaur \textit{Tyrannosaurus rex} could have existed as an obligate scavenger \citep{ruxton2003could}. 

Indeed, the convergent groups of Old and New World vultures are the only representatives to inhabit this niche today \citep{houston2001condors}. These birds have a suite of adaptations that allow them to flourish as obligate scavengers. Their efficiency is illustrated by cases of predators like bears and wolves benefiting by taking more carrion in their diet in areas bereft of vultures through competitive release \citep{devault2003scavenging}. In flight, birds possess a huge advantage over any terrestrial obligate scavenger. Flight affords them the ability to range over a much larger area and detect carrion from an elevated vantage point. \cite{pennycuick1972soaring} conservatively estimated that a \textit{Gyps} vulture could identify activity at a carcass 4 km away. Further, vultures take advantage of soaring in thermals \citep{mundy1992vultures}. This economical means of locomotion, whereby the birds circle around pockets of hot air which provide uplift, allows them to cover huge distances at a low energetic cost (figure \ref{fig:scatter_home}). At large body masses flapping flight is prohibitively expensive in energetics terms \citep{hedenstrom1993migration}. These advantages underscore the importance of the environment, in this case an aerial one, in enabling a scavenging lifestyle. Relative to a terrestrial setting, both locomotion and resource detection are easier for aerial scavengers \citep{tucker1975energetic,ruxton2004obligate}. \\ \indent
\cite{ruxton2004obligate} argue that large body size is another adaptive response to scavenging. Cinereous Vultures (\textit{Aegypius monachus}) and condors (\textit{Vultur gryphus}, \textit{Gymnogyps californianus}) all have body masses that can exceed 10 kg and represent some of the heaviest bird species capable of flight \citep{ferguson2001raptors,donazar2002effects}. Many of the other 23 species of vulture are at the upper end of the scale in terms of body mass \citep{ruxton2004obligate}.  As carrion generally occurs in large 'packages', bigger birds can stock up and create body reserves which will benefit them during times of insufficiency. Large body mass also confers dominance advantages during agonistic interactions \citep{kruuk1967competition,KaneVul}. \\ \indent
Social behaviours have been theorised as further adaptations to improve foraging efficiency because they allow for active and passive information transfer \citep{jackson2011evolutionary,wakefield2013space,moleon2014inter}. Many vulture species, notably those in the genus \textit{Gyps}, display social behaviours such as communal nesting and group foraging \citep{mundy1992vultures}.  By nesting communally the birds are concentrated in space so at the start of the foraging day they will form foraging groups. This may lead to local enhancement effects such that one bird will follow another that descends to a carcass \citep{jackson2011evolutionary}.  \\ \indent Some of these adaptations mean vultures are incapable of killing prey themselves; for instance their wing morphology renders them far less agile than a raptorial counterpart \citep{ruxton2004obligate}. The selective pressures that push mammals and reptiles towards scavenging do not seem to undermine their ability to hunt in the same way, perhaps explaining the absence of obligate scavengers in these groups \citep{ruxton2004obligate}. 

\begin{figure}[H] %!htb keeps the figure in this section before moving onto the discussion
	  \centering
	  \includegraphics[width=0.8\textwidth,natwidth=610,natheight=642]{chap1/figures/scatter_home.pdf}
	    \caption[Comparative home range of birds] %This is the label in table of contents
	    {Comparative home range of birds (107 species from 11 orders), with \textit{Gyps} vultures highlighted in red, non \textit{Gyps} vultures in black and all other species in grey. The vulture species are \textit{Gyps corprotheres}, \textit{Gyps fulvus}, \textit{Aegypius monachus}, \textit{Cathartes aura}, \textit{Gypaetus barbatus}, \textit{Gyps bengalensis}, \textit{Gyps tenuirostris} and \textit{Neophron percnopterus}. Home range was calculated using minimum convex polygon methods. Data are log transformed \citep{peery2000factors,garcia2011ranging,gilbert2007vulture,kruger2014trends,vasilakis2005breeding,dwyer2010ecology,donazar1996communal,stroem2001home,buenestado2008habitat,gilbert2005behaviour,nesbitt1990home,pejchar2005hawaiian,springborn2005home,elchuk2003home,rhim2006home,legagneux2009variation,rolando1998factors,hoffman1991spring,dreitz2005movements,hansbauer2008comparative,garza2005home,vega2003home,stober2006variation,novoa2006home,holbrook2011home,fearer1999relationship,brandt2008breeding,giesen1992winter}.}%this is under the figure
	  \label{fig:scatter_home}
	\end{figure}

\section{\uppercase{C}onservation status}
Avian scavengers represent an ecologically important guild that has suffered severely from the effects of anthropogenic change \citep{ogada2012dropping} which is not helped by our ignorance of the group. The importance of vultures to ecosystem functioning is best illustrated by the sudden collapse of their numbers in the Indian subcontinent. By feeding on the carcasses of drug-treated cattle, the vultures were poisoned en masse by diclofenac \citep{oaks2004diclofenac}. Their absence led to an increase in feral dogs and a resultant spike in rabies' incidence among humans \citep{markandya2008counting}. Aside from poisoning, both targeted and incidental, vultures are endangered by wind turbines, electricity pylons, habitat destruction, food loss and poaching \citep{monadjem2003threatened,virani2011major,martin2012visual}. Despite these threats they have proved to be an adaptable group, illustrated by their willingness to feed on the carrion of domesticated animals \citep{mundy1992vultures} and their use of electricity pylons as roosting locations which has caused an expansion in their range \citep{phipps2013power}. Efforts at conserving vultures look to increase the availability of food \citep{piper2005supplementary}, rehabilitate injured individuals \citep{monadjem2014effect}, prevent the spread of poisons \citep{green2004diclofenac}, increase public awareness of the group's function etc. \citep{monadjem2004vultures}. However, the effect of implementing these strategies remains poorly understood \citep{monadjem2014effect}. We can take solace in the recent spike in research on vultures which has been brought about by high profile population crashes like that of India \citep{manga2006vulture}. Hopefully, such catastrophes won't be necessary to keep biologists focused on this group in the future. 

\section{\uppercase{R}esearch outline}
This thesis draws on a wealth of data collected by field researchers in Africa to answer questions on the foraging ecology of \textit{Gyps} vultures, moving from individual based methods to a study of population dynamics. The possibility of obligate scavengers in past ecosystems is an interesting area because it reveals the adaptations that permit an animal to exist in the niche. As such I use some of the theory and methods from my previous work in describing the feeding ecology of \textit{Tyrannosaurus rex} whose predatory tendencies have been called into question. 
\vspace{10 mm}

\textbf{Chapter 2}:
In order to effectively conserve species we must know the extent of their home range and how this can vary across time. Here I describe the home range of 29 Cape Vultures caught, released and tracked in southern Africa. My analysis shows the species range significantly farther as immature birds and, as adults, are constrained by the breeding season. Such variation means there is no easy solution to conservation efforts like supplementary feeding sites which need to be tailored to cater for all life stages. 
\vspace{10 mm}


\textbf{Chapter 3}:
As social birds, vultures often rely on each other to find food. They frequently follow the descent of another individual in the hope that it has discovered a carcass. In a Kenyan study site, I show that vultures also take cues from scavenging eagles. I propose that the eagles are used by the vultures in two ways: first they act as indicators of the presence of food; and second the eagles can tear open the hide of a carcass with their relatively stronger beaks, providing a resource that would otherwise be much more difficult for the vultures to access. In each case the larger vultures can then displace the eagles through competitive dominance and monopolise the remaining food. These newly identified social interactions among different species highlight the vital importance of applying integrated management strategies to conserve endangered vulture species.
\vspace{10 mm}

\textbf{Chapter 4}:
Carrion ecology, the fate of animal carcasses, is a crucial component of every ecosystem. Mismanagement can lead to environmental disasters, most notably the decimation of Asian vulture populations which fed on the carrion of drug-treated cattle. Here, I provide data on the carrion ecology of an ecosystem in Swaziland and predict the future trends of its vulture populations. Using novel methods I show that, despite a closure in supplementary feeding sites, these species have enough food from wild carcasses to survive. But only for the time being. Therefore, I recommend a dedicated vulture restaurant initiative to ensure the population survives.
\vspace{10 mm}

\textbf{Chapter 5}:
The feeding ecology of \textit{Tyrannosaurus rex} has long been debated with distinct 'predator versus scavenger' camps. I move away from this polarised debate and explore the effect of the drastic ontogenetic morphological changes the animal underwent. Specifically, I ask whether these changes resulted in it incorporating ever more carrion in its diet. To address my question I created an individual-based model based on the energetics of its movement and the amount of carrion available at the time only to find that the reverse is true. The cost of movement meant searching for patchily-distributed carrion became prohibitively expensive as \textit{T.rex} grew. Instead, my results suggest the dinosaur switched from small prey items to more cumbersome, armoured species as it developed into its robust adult form. 

\section{\uppercase{A}dditional work}
In addition to the chapters enclosed in this thesis, I have also been involved in the following research during my studies:\\
\begin{singlespace}
Healy, K., Finlay, S., Guillerme, T., Kane, A., Kelly, S., McClean, D., Kelly, D., Donohue, I., Jackson, A.L., \& Cooper, N. (2014). Ecology and mode-of-life explain lifespan variation in birds and mammals. Proceedings of the Royal Society B: Biological Sciences 281.1784: 20140298. \\
\end{singlespace}

\noindent
I was involved with the conception, data collection and write-up of this paper. \\
\begin{singlespace}
Wakefield, E.D., Bodey, T.W., Bearhop, S., Blackburn, J., Colhoun, K., Davies, R., Dwyer, R.G., Green, J., Gr{\'e}millet, D.,Jackson, A.L., Jessopp, M.J., Kane, A., Langston, R.H.W., Lescro{\"e}l, A., Murray, S., Le Nuz, M., Patrick, S.C., P{\'e}ron, C., Soanes, L., Wanless, S., Votier, S.C., \& Hamer, K.C. (2013). Space Partitioning Without Territoriality in Gannets. Science. 341, 68-70. doi: 10.1126/science.1236077 \\
\end{singlespace}
\noindent
	I took the lead in developing the individual based models for this paper and taught one of the lead authors (TWB) about the method, working closely with him in developing a suitable model for the gannet system. \\
\begin{singlespace}
Monadjem, A., Wolter, K., Neser, W., \& Kane, A. (2013). Effect of rehabilitation on survival rates of endangered Cape vultures. Animal Conservation. doi: 10.1111/acv.12054 \\
\end{singlespace}
\noindent
	I was again involved in the data analysis and write-up of this manuscript and forged some links with two vulture conservationists based in South Africa. \\
\begin{singlespace}
Monadjem, A., Kane, A., Botha, A., Dalton, D., \& Kotze, A. (2012). Survival and Population Dynamics of the Marabou Stork in an Isolated Population, Swaziland. PloS one, 7(9), doi: 10.1371/journal.pone.0046434 \\
\end{singlespace}
\noindent
	I was involved in the data analysis and write-up of this manuscript. This was my first full peer-reviewed paper. \\
\begin{singlespace}
Kane, A. (2012). A suggestion on improving mathematically heavy papers. Proceedings of the National Academy of Sciences. 109(45) E3058-E3059. doi: 10.1073/pnas.1212310109 \\
\end{singlespace}
	\noindent
	This letter was a response to an article highlighting the divide between theoretical and empirical                   biologists owing to different competencies in mathematics. 

 %This is a ch1-introduction.tex file with contents of intro chapter



%\thesischapter{Movement ecology of the Cape Vulture}
%  {Adam Kane, Kerri Wolter, Walter Neser, Andrew L Jackson \& Ara Monadjem}

% \epigraph{Adam Kane, Kerri Wolter, Walter Neser, Ara Monadjem} can add an inspirational quote with this
\chapter{Movement ecology of the Cape Vulture}
\label{chap:introduction}

\textit{Authors:} Adam Kane, Kerri Wolter, Walter Neser, Andrew L Jackson \& Ara Monadjem
%Adam Kane$^1$, Kerri Wolter$^2$, Walter Neser$^2$, Andrew L Jackson$^1$ \& Ara Monadjem$^3$

\vspace{10 mm}
\noindent
\textit{\uppercase{A}uthor contributions}
I conceived the idea, analysed the data, interpreted the results and wrote the manuscript;
AM supplied data and advised on data analysis; 
AJ gave feedback on drafts of the manuscript;
KW and WN were responsible for the capture and tagging of the birds. 

\vspace{10 mm}

\noindent
\textit{Status:} This manuscript is being prepared for publication. Target journal is Bird Conservation International.

\newpage


\noindent

\section{\uppercase{A}bstract}

	Identifiying the areas animals use and how this varies across time is vital to conservation efforts esepcially in the case of wide-ranging vultures that traverse over international borders. Here we conducted a large scale analysis of the movement ecology of 29 Cape Vultures (\textit{Gyps coprotheres}) that were tracked for an average of 311 days over Southern Africa. We describe differences in home range size between adults and immatures as well as seasonal variation. We show that, in line with smaller scale studies, immature birds have larger home ranges than adults. This is likely due to strong competition at feeding sites near colonies and the constraint on adults of having dependent-young. There is also a significant effect of seasonality such that adult birds have smaller home ranges during the dry season. This is also likely a result of breeding birds being constrained by their nests as well as increased food availability provided by higher ungulate death rates. We discuss the conservation implications of these results with special mention given to vulture restaurants. We advocate the use of both frequently stocked restaurants close to colonies to benefit adults and infrequently stocked sites farther away from colonies for the benefit of immature birds. 
\newpage

\section{\uppercase{I}ntroduction}

	The Cape Vulture (\textit{Gyps coprotheres}) is a large (9 kg), obligate scavenging vulture known from southern Africa \citep{mundy1992vultures}. It is IUCN red listed as vulnerable with a declining population. Given their longevity and low reproductive output (clutch size is rarely more than one) Cape vultures are very sensitive to reductions in their survival rates \citep{phipps2013power,monadjem2014effect}. The bird is known to be a wide-ranging species \citep{bamford2007ranging}. Indeed \textit{Gyps} vultures \citep{monsarrat2013predictability} have a much larger home range than that of closely related members of the Accipitriformes \citep{peery2000factors} owing to their dependence on patchily distributed carrion \citep{ruxton2004obligate}. Such large home ranges make conserving these species a challenging task given that they may fly over many countries with different agendas pertaining to wildlife management \citep{lambertucci2014apex}. Seasonal and ontogenetic differences in home range have been illustrated before in \textit{Gyps} vultures \citep{monsarrat2013predictability, phipps2013power}. This is significant because specific life stages can have a disproportionate impact on the population growth rate, with the adult stage being most sensitive in the case of the Cape vulture \citep{monadjem2012survival,monadjem2014effect}.  \\
 \indent As such, the identification of home range is of great importance in conservation efforts, especially in recognising that home ranges are not static over either long or short periods \citep{burt1943territoriality}.  Here we define home range as "that area traversed by the individual in its normal activities of food gathering, mating, and caring for young" \citep{burt1943territoriality}. We used a large dataset of GPS tracked individuals, to test two intuitive predictions. First, we expected that the home range of immature birds would be larger than that of adults due to adults being constrained by their young \citep{mundy1992vultures}, the competition suffered by immature birds at feeding sites near colonies which forces them to forage farther afield and, the larger body mass of adults which can hamper ability to take off  \citep{robertson1986feeding,mundy1992vultures}. This is in line with previous studies based on fewer individuals \citep{mundy1992vultures, bamford2007ranging, phipps2013power}. Secondly, we hypothesised that adults would have a smaller home range during the breeding season compared to the non-breeding season, again because they have dependants as well as poor flying conditions during this period \citep{mundy1992vultures, monsarrat2013predictability}. The breeding season also coincides with the dry season for Cape vultures, this is a period during which the ungulates suffer an increased death rate and provide more carrion to the scavengers \citep{mundy1992vultures}. This is another factor that could reduce the home range of the birds given that they would encounter food more often during this period. 

Elucidating the movement ecology of these birds at different times, both seasonal and ontogenetic, is important because vulture conservation efforts frequently make use of long-term supplementary feeding sites \citep{gilbert2007vulture}. The benefits these sites may have could be maximised by knowing when the birds will most often avail of them and which life stages \citep{monsarrat2013predictability}. 



	%--------------------------------------
% Scatter-plot with Home Ranges

	
%------------------------------------------	
	

\section{\uppercase{M}ethods}
	In total, we analysed the movements of 29 Cape Vultures. These birds were captured by VulPro, a vulture conservation organisation, at trapping localities in South Africa using a walk-in trap \citep{diekmann2004capture,phipps2013power}. See \cite{phipps2013power} for details of their capture and release. Seven of these individuals have been used in a previous study investigating the effects of power lines on Cape Vulture movements \citep{phipps2013power}. The birds were separated into adult (> 5 years) and immature (< 5 years) categories \citep{piper1981estimates} (Table \ref{tab:homerange}). The sex of every bird was not known so we could not include it as an explanatory variable for home range variation; however it should be noted that adult \textit{Gyps} vultures share parental duties and would not be expected to vary much in this respect \citep{houston1976breeding}. 
 We calculated the home range for each individual using minimum convex polygon (MCP) \citep{mohr1947table} and kernel utilisation distribution methods (KUD) (both set to 95\%, such that 5\% of the most extreme points were removed)\citep{worton1989kernel}. We computed mean and standard deviation summary statistics for these home range areas for the individual birds. Then we used Wilcoxon-Mann-Whitney rank sum tests (the non-parametric alternative to the t-test) to determine if there was a difference in home range size between adult and immature Cape Vultures; and used the same method to test for a difference in home range with season (wet versus dry). The dry season was defined as May-August \citep{cooper1988foliage,mundy1992vultures}. Note that we did not know if a given adult was actually breeding. \\
 \indent For a subset of the birds, namely those that were tracked at a higher temporal resolution (once every 15 minutes cf. the adults who were tracked at 7.00, 11.00 and 15.00 each day), we estimated the distance they covered during a day (Table \ref{tab:dist_immature}). These were all immature individuals. We carried out our analysis using base R and the R package adehabitatHR \citep{RCran,calenge2013adehabitathr}. 

	%--------------------------------------
% Map with Home Ranges
\begin{figure}[H] %!htb keeps the figure in this section before moving onto the discussion
	  \centering
	  \includegraphics[width=0.7\textwidth]{chap1/figures/map.pdf}
	    \caption[Map of Cape Vulture distribution with home ranges] %This is the label in table of contents
	    {Distribution of the Cape Vulture in Southern Africa (data from Bird Life International in grey) and Home Range (KUD 95\%) of the 7 immature birds tracked at high resolution. }%this is under the figure
	  \label{fig:map}
	\end{figure}
	%,natwidth=610,natheight=642
%------------------------------------------	

\section{\uppercase{R}esults}
	Mean KUD home ranges were 282,255 km$^2$ (sd $\pm 288,460$ km$^2$) for immature birds and 110,181 km$^2$ (sd $\pm 130,464$ km$^2$) for adults (Table \ref{tab:homerange}). The Wilcoxon-Mann-Whitney rank sum tests showed this was a significant difference according to the KUD measure (p = 0.037). By comparison the MCP difference gave a p-value of 0.057 (figure \ref{fig:homerange}). The home ranges of the 7 high resolution birds are shown in figure \ref{fig:map}.

%-----------------------------------------------
%Home range figure	
%I'm not sure how to get the picture onto the same page as the table but there must be a way
	\begin{figure}[H] %!htb keeps the figure in this section before moving onto the discussion
	  \centering
	  \includegraphics[width=0.8\textwidth,natwidth=610,natheight=642]{chap1/figures/homerange.pdf}
	    \caption[Comparison of the home range of adult and immature birds. ] %This is the label in table of contents
	    {Comparison of the home range of adult and immature birds based on KUD. The median is shown by the horizontal bar within the box. The whole box captures the interquartile range of the home ranges i.e. the 25 to 75 
percentile. There is an outlier for both the adult and the immature birds.}%this is under the figure
	  \label{fig:homerange}
	\end{figure}

%--------------------------------------	

	Mean KUD for adults during wet and dry season were 80,719 km$^2$ (sd $\pm 137,799.1$ km$^2$) and 36,138 km$^2$ (sd $\pm 50,978$ km$^2$) respectively. Mean KUD for immatures during wet and dry season were 58,236 km$^2$ (sd $\pm 74,593$ km2) and 74621 km$^2$ (sd $\pm 99,304$ km$^2$) respectively. There was a significant difference between wet and dry season for adult birds (p = 0.03837) but not for immature birds (p = 0.5176) using KUD methods.

%------------------------------------------
	
	\begin{table}[H] %!htb keeps the table in this section before moving onto the next block of text
		\caption[Home range data] %This goes into  your list of tables
				{Home range (KUD and MCP) data on the Cape Vultures which were divided into immature and adult birds. Immature values are in the top half of the table, adults the bottom half.} %I used the HR abbreviation because otherwise the table was too wide for the page
		\input{chap1/tables/homerange}
		\label{tab:homerange}
	\end{table}
%-------------------------------------
	
	The average daily distance covered by the 7 high resolution birds was approximately 72 km if we used every relocation (Table \ref{tab:dist_immature}). However when we used only the start and end relocations for a given date, i.e. "as the crow flies", the average was reduced to just over 33 km as a result of the loss of smaller scale movements. 
	
%-------------------------------------	
	\begin{table}[H]
		\caption[Daily distance travelled by immature birds]
				{Average daily distance travelled by the immature birds that were tracked at a high resolution (once per 15 minutes) and the distance "as the crow flies" between the start and end points of the given date. }
		\input{chap1/tables/distance_immature}
		\label{tab:dist_immature}
	\end{table}

\vspace{10 mm}


\newpage
\section{\uppercase{D}iscussion}
The foraging day of a Cape vulture typically takes place between 10.00 and 18.00 (local time) \citep{mendelsohn2005observations}. They usually fly between an altitude of 250 - 350 m above ground \citep{mundy1992vultures}. The daily foraging distance reported here (an average of over 70 km) illustrates the impressive abilities of immature Cape Vultures in flight. In contrast, adults of the species have been calculated as covering a radius of 54 km in a day \citep{mundy1992vultures}. By way of comparison with immature vultures of other species, \cite{phipps2013foraging} showed immature African White-backed Vultures (\textit{Gyps africanus}) flew on average 33.39 km per day. Note that those estimates were based on locations recorded three times daily, a much coarser resolution, and thus nearly identical to our "as the crow flies" measure (mean = 33 km). This difference between adults and immatures accounts for the disparity seen in the home range of the Cape Vulture life stages. \\
 \indent
	Indeed, our large dataset on tracked Cape Vultures demonstrates a significant effect of both age and season on home range size. As mentioned above, these results are most probably the result of adults having to care for dependent young as well as the dominance hierarchy that exists across ontogeny \citep{duriez2012intra,mundy1992vultures}. We should also consider the effect of weather \citep{shepard2013daily}. The mass of an adult bird can be prohibitive when cold temperatures result in the formation of weak thermals. These cliff faces deflect wind upwards and are used by the soaring vultures to offset their sink rate \citep{shepard2013energy}. This can restrict adult flight time during winter months especially at colony cliffs in open savanna \citep{mundy1992vultures}. There is also the effect of differential seasonal mortality. The higher death rate of ungulates during the dry season means the vultures will encounter more carrion and won't have to travel as far. 

	\indent The difference in home range size between Cape Vulture life stages has implications for its conservation. We know from population growth models showing that adults have a disproportionate effect on the growth rate of Cape vulture populations \citep{monadjem2014effect}. Habitat fragmentation is known to increase the size of an animal's home range because it causes a reduction in resource density \citep{haskell2002fractal}. To counter this, conservationists can supply supplementary food \citep{piper2005supplementary}. However, these sites impact adult and immature birds in different ways depending on how they are run.  \cite{duriez2012intra} showed that, at artificial feeding sites, close to colonies and regularly stocked with food, adult vultures (\textit{Gyps fulvus}) dominated young individuals who were left to fight over scraps. But "light" feeding sites, located farther from colonies and supplied less regularly, were preferred by immature birds. Conservationists should therefore use a mix of "light" and "heavy" feeding sites if they are to effectively manage both ontogenetic stages. \cite{komen1991energy} calculated the energy requirements for this species across its ontogeny, a piece of research that could be put to good use when supplying supplementary feeding sites at different times. We also have excellent data on the population sizes of Cape Vulture colonies in South Africa \citep{whittington2011monitoring}. By knowing where they forage, how much food they require and variability in the species home range we are well placed to sustain this species. However, conservation actions must be at an international level \citep{lambertucci2014apex} given that Cape Vultures are not impeded by national boundaries. 


%\end{document}
 

\chapter{Vultures acquire information on carcass location from scavenging eagles}
\label{chap:scrounger}

%\thesischapter{Vultures acquire information on carcass location from scavenging eagles}
%  {Adam Kane, Andrew L Jackson, Darcy L Ogada, Ara Monadjem & Luke McNally}
\textit{Authors:} Adam Kane, Andrew L Jackson, Darcy L Ogada, Ara Monadjem \& Luke McNally
%Adam Kane$^1$, Andrew L Jackson$^1$, Darcy L Ogada$^2$, Ara Monadjem$^3$ \& Luke McNally$^4$

\vspace{10 mm}
\noindent
\textit{\uppercase{A}uthor contributions} 
I conceived the idea, analysed the videos for arrival times and competitive interactions, created and ran the IBMs, ran the statisitcs on these sections, helped LM develop the game theory model and wrote the manuscript; 
AJ developed the permutation tests and gave input on the manuscript; DO recorded the videos on which this study is based and gave feedback on the manuscript; AM gave feedback on the manuscript; LM created the game theory model, advised on the statistics and gave feedback on the manuscript.

\vspace{10 mm}

\noindent
\textit{Status:} This manuscript has been published at Proceedings of the Royal Society B.

\newpage

\noindent

\section{\uppercase{A}bstract}

Vultures are recognised as the archetypal scavengers of the natural world. While it is well known that vultures use conspecific social information as they forage, the possibility of inter-guild social information transfer and the resulting multi-species social dilemmas has not been explored. Here we use data on arrival times at carcasses to show that such social information transfer occurs with raptors acting as producers of information and vultures acting as scroungers of information. We develop a game-theoretic model to show that competitive asymmetry, whereby vultures dominate raptors at carcasses, predicts this evolutionary outcome. The eagles, which arrive earlier, benefit from gaining a finder's fee enabling them to exist in a producer role. We support our theoretical prediction using empirical data from competitive interactions at carcasses. Finally, we use an individual based model to show that these producer-scrounger dynamics lead to vultures being vulnerable to declines in raptor populations. Our results show that social information transfer can lead to important non-trophic interactions among species and highlight important potential links among social evolution, community ecology and conservation biology. With vulture populations suffering global declines our study underscores the importance of ecosystem-based management for these endangered keystone species. 
\newpage

\section{\uppercase{I}ntroduction}

Animals base their decisions on both personal and public information \citep{dall2005information,schmidt2010ecology,sumpter2008information,couzin2009collective}. This is applicable to every facet of an animal's life, be it feeding, movement, mating etc. with high fidelity information allowing an individual to make decisions conducive to its survival \citep{dall2005information,mcnamara2010information,danchin2004public}. Public information can be separated into that which is gained from conspecifics and that from heterospecifics \citep{dall2005information}. Conspecific information transfer is essential for basic behavioural functions like sexual reproduction or cooperative hunting \citep{handegard2012dynamics}. But species overlap in the resources they use \citep{fedriani2000competition,kruuk1967competition}, and the environments they inhabit \citep{fedriani2000competition,kruuk1967competition} which gives the possibility of inter-guild information transfer \citep{seppanen2007social,seppanen2007interspecific,forsman2009experimental}. \\
\indent
Consider the social \textit{Gyps} vultures, a group that is known to forage collectively for carrion. In flight they appear to keep in visual contact with conspecifics \citep{houston1974food}. Once one vulture discovers and descends to a carcass the information is conveyed to others in the area, this activity can create a local enhancement effect \citep{jackson2008effect}. But such social behaviour renders vultures' foraging efficiency susceptible to population declines; with every individual lost, the network is less effective at detecting carrion \citep{jackson2008effect}. 
Although vultures are the most well-known group of avian scavengers there are a number of other species within the family Accipitridae such as eagles (hereafter raptors) that take carrion as a significant proportion of their diet \citep{mundy1992vultures}. Coexistence among all of these species is possible by both temporal and resource partitioning \citep{mundy1992vultures,houston1975ecological}. But with any shared resource, direct interactions between them will result, and the possibility of social information transfer among species emerges. \\
\indent A distinct pattern of arrival of avian scavengers to carrion has been highlighted before \citep{mundy1982comparative,kendall2013alternative,cortes2012resource}. Indeed, the African White-backed Vulture (\textit{Gyps africanus}) has been noted in using many other scavengers as a means of local enhancement while foraging \citep{kruuk1967competition}. Yet these heterospecific interactions and their potential for information transfer have not been explored in any detail \citep{mundy1982comparative}. \\
\indent Given the current extreme declines in vulture populations \citep{ogada2012dropping,green2004diclofenac} and their key role in many ecosystems as biomass recyclers \citep{ogada2012effects,sekercioglu2006increasing}, understanding vulture foraging ecology is also of applied relevance. Here we provide evidence for producer-scrounger dynamics among scavenging vulture and raptor species by testing the hypothesis that vultures scrounge information from raptors, and explore its evolutionary underpinnings using a game theoretic model. We conclude by outlining the consequences of this system's properties for vulture conservation.


\section{\uppercase{T}est for producer scrounger dynamics}

To test for the occurrence of producer-scrounger dynamics between vultures and raptors we observed arrival times of avian scavengers to a number of experimental carcasses placed out. Our observations were made on 46 videos recorded in the Mpala Research Centre in the Laikipia District of Kenya which had scavenging avifauna (a subset of the videos used in \cite{ogada2012effects}). The carcasses of partially skinned goat and cow carcasses were set out at dawn (06.15 - 07.20) in an open area of the Mpala Ranch which consists of 20,000 ha of savannah. Carrion size ranged from 20 - 340 kg with a mean of 80 kg and did not have a significant effect on the number of avian scavengers present \citep{ogada2012effects}. We focused on the closely related and morphologically similar \textit{Gyps} vultures, the African White-backed Vulture and the R{\"u}ppell's Vulture (\textit {Gyps rueppellii}), as well as the congeneric Tawny (\textit {Aquila rapax}) and Steppe Eagles (\textit {Aquila nipalensis}). Some differences between the eagles include the tendency for the Steppes to be slightly larger and more social \citep{clark1992taxonomy}. These four species were by far the most abundant in the recordings (>95\%) and formed our vulture and raptor groups respectively \citep{ogada2012effects}. 
For each video we noted the arrival time and species of every animal. Initially, we compared the probability of producing information on carcass location by looking at which of the two groups, \textit{Gyps} vultures or raptors, landed at the carcass first. A binomial test on the 46 videos showed that the first bird to land at a carcass was significantly more likely to be a raptor than a vulture (Binomial test, 38 successes, 46 trials, expected probability = 0.5, results in observed probability = 0.83, 95\% C.I. 0.69-0.92, p-value < 0.001 (figure \ref{fig:arrival_plot})).

%------------------------------------------	

% Plot of bird arrival times at carcasses 
\begin{figure}[H] %!htb keeps the figure in this section before moving onto the discussion
	  \centering
	  \includegraphics[keepaspectratio,totalheight=0.8\textheight]{chap2/figures/arrival_plot.pdf}
	    \caption[Arrival times of vultures and raptors to carcasses] %This is the label in table of contents
	    {Recorded arrival times of individual vultures and raptors at carrion across 46 videos. The red lines are raptors, black are vultures. }%this is under the figure
	  \label{fig:arrival_plot}
	\end{figure}
	
%------------------------------------------	

We used randomisation tests to test if the birds were following each other rather than simply arriving independently but with different timing to the carcasses. Where a raptor landed first we generated a null distribution of arrival times for the first scrounger (a \textit{Gyps} vulture) over the length of each recording by randomising the arrival times of the birds. From this distribution we assessed if the first scrounger followed more closely than expected under an assumption of independent foraging. Then, in order to make a population level inference across permutation tests, we use a binomial test where the expected probability of observing a significant result by chance is 0.05 (as per the definition of a p-value where at an alpha of 0.05, we would expect 5\% of test results to be significant according to the null model). Vultures were found to follow raptors more closely than expected by chance (i.e. with p<0.05) in 20 of the 38 videos (figure \ref{fig:pvalue_plot}a) which is significantly more cases  of vultures following raptors than expected (Binomial test with 20 successes, 38 trials, expected probability = 0.05 results in an observed probability = 0.53, 95\% C.I. 0.36-0.69, p-value <0.001).
Similarly, in six of the eight occasions when vultures landed at the carcass first, raptors followed more closely than expected by chance (figure \ref{fig:pvalue_plot}b), which is significantly more cases than expected (Binomial test with 6 successes, 8 trials, expected probability = 0.05, results in an observed probability = 0.75, 95\% C.I. 0.35-0.97, p-value< 0.001). 

%------------------------------------------	

% Plot of p-values
\begin{figure}[H] %!htb keeps the figure in this section before moving onto the discussion
	  \centering
	  \includegraphics[keepaspectratio]{chap2/figures/pvalue_plot.pdf}
	    \caption[Histogram plot of p-values] %This is the label in table of contents
	    {Histograms of p-values showing the number of videos where it was significantly probable that (a) the vultures were following the raptors and (b) the raptors were following the vultures. The vertical lines show the level of significance at p = 0.05.}%this is under the figure
	  \label{fig:pvalue_plot}
	\end{figure}
	
%------------------------------------------	

\section{\uppercase{P}roducer scrounger model}

While our analyses suggest that both raptors and vultures follow each other to carcasses, the higher frequency of raptors being the first to land at a carcass leads to raptors acting predominantly in a producing role. This can manifest by raptors providing information on the location of the resource or by engaging in carcass opening whereby the raptor uses its relatively stronger bill to get through an ungulate hide \citep{kendall2013alternative}, with vultures acting predominantly as scroungers. This result raises the question of how these divergent roles evolved? We hypothesised that competitive ability may have a strong effect on the strategy typically adopted in each species. If individuals of one species can competitively dominate those of another this may favour scrounging by the dominant species: producers may gain an exclusive share of the resource by arriving first at the carcass (a "finder's fee") \citep{vickery1991producers}, but dominant scroungers may be able to effectively monopolise the remainder of the resource once they arrive, while also gaining information available publically on the locations of carcasses. \\
\indent
We used the game-theoretic framework of the producer-scrounger game to test the evolutionary feasibility of this prediction. In a producer-scrounger game an animal needs to invest either in producing some resource (e.g. information), or exploit the investment from another individual \citep{morand2010learning}. We consider a scenario where vultures and raptors forage in the same area. We assume that time spent feeding at a carcass is small relative to search time \citep{mundy1992vultures,barta1998effect}. Individual vultures and raptors can assume one of two strategies: producer or scrounger. Producers find carcasses at a rate proportional to the carcass density; while all scroungers in a group will follow producers to the carcasses they find, but don't find carcasses for themselves (probabilistic following by scroungers cannot qualitatively affect our results). While in reality vultures and raptors will likely use mixed strategies, this simple scenario allows us to abstract the essential elements of the evolutionary dynamics in a simple framework.

From our assumptions we can write the numbers of vultures and raptors at a carcass found by a vulture as $m_{v,v}$ = 1+$v_s$ and $m_{v,r}$ = $r_s$, respectively, where $v_s$ and $r_s$ are the numbers of vultures and raptors that are scroungers. Similarly the numbers of vultures and raptors at a carcass found by a raptor are $m_{r,v}$ = $v_s$ and $m_{r,r}$ = 1 + $r_s$, respectively. We assume that new carcasses arrive in the area and decay at fixed rates (both set at 1) and are consumed almost instantaneously when found. Assuming that carcass dynamics occur on a faster timescale than the population dynamics of vultures and raptors the steady-state density of carcasses is then given as d = 1/(1 + $r_p$ + $v_p$), where $v_p$ and $r_p$ are the numbers of vultures and raptors that are producers. The rates of food consumption for producing and scrounging vultures are:

\begin{equation}
\pi _{v,p} = \left ( \frac{1}{1+r_{p}+v_{p}} \right )\left ( a+(1-a)\frac{x}{xm_{v,v} + m_{v,r}} \right )
\end{equation}

and

\begin{equation}%
\pi _{v,s} = \left ( \frac{1}{1+r_{p}+v_{p}} \right ) \left ( 1-a \right ) \left (\frac{v_{p}x}{xm_{v,v} + m_{v,r} }+\frac{r_{p}x}{xm_{r,v}+m_{r,r}}\right)
\end{equation}

Similarly the rates of food consumption for producing and scrounging raptors are:

\begin{equation}
\pi _{r,p} = \left ( \frac{1}{1+r_{p}+v_{p}} \right )\left ( a+(1-a)\frac{1}{xm_{v,v} + m_{v,r}} \right )
\end{equation}

and

\begin{equation}
\pi _{r,s} = \left ( \frac{1}{1+r_{p}+v_{p}} \right ) \left ( 1-a \right ) \left (\frac{v_{p}}{xm_{v,v} + m_{v,r} }+\frac{r_{p}}{xm_{r,v}+m_{r,r}}\right)
\end{equation}

\vspace{10 mm}

Here a is the proportion of a carcass that is monopolised by the individual that finds it (the "finder's fee"), and the variable x is the competitive ability of vultures compared to that of raptors, specifically the number of raptors that a vulture is equivalent to in terms of competitive ability. The proportion of the carcass remaining after the finder's fee (1-a) is shared among all birds at the carcass proportionally to their relative competitive ability. For example, at a carcass found by a raptor, a scrounging vulture's share would be x/(x$m_{r,v}$ + $m_{r,r}$). The vulture is competitively equivalent to x raptors so gets a positive weighting of x in the numerator. As other vultures will have a similar competitive ability, the number of vultures at the carcass ($m_{r,v}$) is also weighted by x. This leads to each bird receiving a share proportional to its competitive ability relative to the other birds present at the carcass. The probability that a vulture wins a one-on-one interaction with a raptor is then defined as x/(1 + x). This process could then be seen as a series of competitive interactions over small proportions of the carcass, leading to birds on average receiving a share proportional to their relative competitive ability.

While the equilibrium number of carcasses (1/(1 + $r_p$ + $v_p$)) available declines with the density of producers of both species as more carcasses are found and consumed, the food acquisition rate of scroungers is also positively weighted by producer densities as they are able to follow individuals to carcasses more frequently. This means that, while producers are only affected negatively by other producers (owing to reduction in carcass densities), scroungers are affected both positively (owing to their increasing rate of following to carcasses) and negatively (owing to reduction in carcass density) by producer density. 
We write the dynamics \citep{hofbauer2003evolutionary} of producers and scroungers in the vulture and raptor populations as:

\begin{equation}
\frac{\mathrm{dv_{p}} }{\mathrm{d} t} = v_{p}\left ( \pi _{v,p}- \alpha \right )
\end{equation}

\begin{equation}
\frac{\mathrm{dv_{s}} }{\mathrm{d} t} = v_{s}\left ( \pi _{v,s}- \alpha \right )
\end{equation}

\begin{equation}
\frac{\mathrm{dr_{p}} }{\mathrm{d} t} = r_{p}\left ( \pi _{v,p}- \beta + \frac{\gamma }{1+r_{p} +r_{s}}\right )
\end{equation}
and
\begin{equation}
\frac{\mathrm{dr_{s}} }{\mathrm{d} t} = r_{s}\left ( \pi _{r,s}- \beta + \frac{\gamma }{1+r_{p} +r_{s}}\right )
\end{equation}


Here $\alpha$ and $\beta$ are the mortality rates for vultures and raptors, respectively. The additional term $\gamma$ / (1 + $r_p$ + $r_s$) captures additional food intake by raptors owing to their additional source of energy through predation. Here we assume that some prey enters the area at rate $\gamma$, dies at a fixed rate of 1, and is found by raptors at rate 1 and then instantaneously consumed. Again we assume that the dynamics of the prey population happen on a faster time-scale than the raptor population dynamics so that the steady-state density of prey can be used. Varying the parameter $\gamma$ then allows us to vary raptor's relative reliance on carcasses as a food source. 

Unfortunately no analytical solutions are available for our model, so we examine the evolutionary dynamics of the producer-scrounger interaction using numerical evaluation of steady-state of equations 3.5-3.8. The results of the model displayed in figure \ref{fig:game_plot} show the impact of competitive ability, finder's advantage and the availability of prey items to raptors. Notice in plots (a) and (c) that there is a transition from high raptor population densities to high vulture densities as vultures become more dominant over raptors (the switch occurring when the probability a vulture wins is greater than 0.5).  The availability of extra food from predation in plot (c) allows the raptors to persist at higher population densities suppressing the increasing vulture numbers relative to plot (a).  The effect of increasing relative competitive ability is also realised in driving up the proportion of birds scrounging. The outcome of varying the size of the finder's fee is evident as we can see a lower proportion of scroungers when the amount of food consumed by the producer is high. A competitively dominant species gains a larger share of the resource. It follows that any finder's fee is of less value to them than it is to the competitively inferior species. Thus the competitively dominant species is more likely to forego a finder's fee in order to benefit from the increased rate of information acquisition that can be facilitated by scrounging.

%------------------------------------------	

% Plot of game theory
\begin{figure}[H] %!htb keeps the figure in this section before moving onto the discussion
	  \centering
	  \includegraphics[keepaspectratio, totalheight=0.7\textheight]{chap2/figures/game_plot.pdf}
	    \caption[Game theory plot] %This is the label in table of contents
	    {Game theory results. Panels on the left (a, c) are the population densities for each species (vultures, solid lines; raptors, dashed lines), panels on the right (b, d) are the frequencies of scroungers in both the vulture and raptor populations. The top panels (a, b) are where raptors rely strongly on carcasses ($\gamma$ = 0.05); the bottom panels (c, d) are when they rely more weakly on carcasses ($\gamma$ = 0.15). Colours indicate different values for the finder's fee (black, a = 0.05; dark grey, a = 0.2; light grey, a = 0.5) The x-axis is the probability that a vulture wins a one on one interaction, x/ (1+x).  Mortality rates are $\alpha$ = 0.1 and $\beta$ = 0.1 for all panels.}%this is under the figure
	  \label{fig:game_plot}
	\end{figure}
	
%------------------------------------------	

\section{\uppercase{T}est of competitive ability}

The results of our model demonstrate the potential importance of competitive asymmetry in the evolutionary outcome of inter-guild producer-scrounger dynamics. To test our model prediction of competitive dominance by vultures we analysed competitive interactions between \textit{Gyps} and raptor species at carcasses from our videos. We followed \cite{bamford2010associations} in our analysis of agonistic interactions between the birds.  In each case of aggression we noted the initiator, the winner and the loser. The loser was defined as a bird spatially displaced by the direct action of another individual. 
There were 461 interactions in total. We used a binomial generalized linear mixed model with video as a random effect to test the significance of the interactions (figure \ref{fig:competition_plot}). \\ 
\indent
In support of our theoretical predictions we found vultures are more likely to be the initiator of an aggressive interaction (n = 274 Vs 187, $\beta$ = 0.7414, s.e. = 0.1987, p < 0.001, probability = 0.68, 95\% C.I. 0.59-0.76 (figure \ref{fig:competition_plot} a)); vultures are more likely to win when they initiate the contest (n= 265/274, $\beta$ = 3.6942, s.e. = 0.4134, p < 0.001, probability = 0.98, 95\% C.I. 0.95-0.99 (figure \ref{fig:competition_plot} b)) and raptors are more likely to win when they initiate a contest (n = 170/187, $\beta$ = 2.4893, s.e. = 0.3473, p <0.001, probability = 0.92, 95\% C.I. 0.86-0.96). The probability that a vulture wins when they initiate a contest is also significantly greater than the probability that a raptor wins when they are the initiator ($\beta$ = 1.2049, s.e. = 0.4685, p = 0.0101). Finally, vultures are more likely to win overall (n = 282 Vs 179, ($\beta$ = 0.9567, s.e. = 0.2303, p < 0.001, probability = 0.72, 95\% C.I. 0.62-0.80 (figure \ref{fig:competition_plot} c)).

%------------------------------------------	

% Plot of competition
\begin{figure}[H] %!htb keeps the figure in this section before moving onto the discussion
	  \centering
	  \includegraphics[keepaspectratio, totalheight=0.25\textheight]{chap2/figures/competition_plot.pdf}
	    \caption[Competitive interactions between vultures and raptors] %This is the label in table of contents
	    {The results of the competitive interactions (a) shows the probability that a species was an initiator (b) the probability that an initiator wins a contest and (c) the overall competitive ability. Given are the means with 95\% confidence intervals.}%this is under the figure
	  \label{fig:competition_plot}
	\end{figure}
	
%------------------------------------------	

\section{\uppercase{E}ffect of raptor density on vulture foraging efficiency}

The producer-scrounger dynamics that we have illustrated suggest a possible ecological interaction whereby vultures are using raptors to locate carcasses. This would imply that vultures may be vulnerable to declines in raptor populations as their ability to locate food will also decline. To examine this possibility we created an individual based model (IBM) in the program NetLogo \citep{tisue2004netlogo} to explore the effect of raptors on the foraging efficiency of the vultures. Our model is a modified version of \cite{jackson2008effect} and \cite{jackson2011evolutionary}, both of which examined vulture foraging behaviour. The main difference is that we include raptors alongside vultures. 
Our video analysis suggests raptors can find carcasses before vultures. The question is what is it about their biology that allows them to achieve this? A recent study found that Lappet-faced Vultures can discover carrion before African White-backed Vultures despite their smaller population size \citep{spiegel2013factors}. We incorporate the changes they deemed likely to impact differential search efficiencies which are applicable to raptors, namely, visual acuity \citep{howland2004allometry}, flying height, roost departure time and dispersion of the birds at the start of the foraging day owing to different roost arrangements. 

The eye of a Tawny Eagle has an axial length of 26.51 mm (c.f. African White-backed Vultures that measure 20.71 mm) \citep{howland2004allometry}. This gives a measure of visual acuity of 81.5 cycles $m^{-1}$ (again c.f. African White-backed Vultures with a measure 57.5 $m^{-1}$) \citep{spiegel2013factors,howland2004allometry}. So this species has better absolute and relative eyesight if we accept these measures, which is further increased by its probable lower flying height \citep{mundy1992vultures}. Moreover, raptors can depart earlier in the day owing to their lower wing loading relative to the larger \textit{Gyps} vultures (2 kg Vs 5.5 kg) \citep{mundy1992vultures}. The social \textit{Gyps} are also more densely aggregated at their roost sites \citep{mundy1992vultures} relative to the solitary raptors. Vultures are also known to fly at great altitudes \citep{mundy1992vultures}. It has been noted that "producer individuals may fly low to increase their probability of detecting a patch when they fly over it" \citep{vickery1991producers}. By flying above the producing raptors the vultures have the potential to notice any raptor that descends to a carcass. A typical altitude of 350 m and 300 m has been reported for the R{\"u}ppell's Vulture and African White-backed Vulture respectively \citep{mundy1992vultures}. Although the flying height of Tawny and Steppe Eagles is unknown, and despite having different flight styles, these species occupy a similar ecological niche to the Bateleur Eagle (\textit{Terathopius ecaudatus}) which often feeds on carrion \citep{steyn1980breeding}. The Bateleur has been recorded cruising at a height of just 50 m above ground \citep{mundy1992vultures}. We can justifiably make the assumption that the two raptors that predominate our data fly at a similar altitude while foraging or at least below that of the \textit{Gyps} vultures.  

A summary of the model runs can be seen in Table \ref{tab:netlogo}. At the beginning of the IBM the raptors were randomly allocated in the simulation space that corresponds to a square of 100 x 100 km with periodic boundary conditions so that a bird that flies off the edge of the square will reappear on the opposite side. The vultures are located in a single patch which represents their roost. The raptors forage for seven hours and the vultures five hours \citep{spiegel2013factors}. The vultures change direction by 45 degrees once every eight minutes which is based on the time they spend in thermals \citep{xirouchakis2009foraging}; since raptors are less dependent on thermals they change direction at double this rate. Both have a constant speed as they attempt to find a single randomly located carcass. Both vultures and raptors can find the carrion by themselves. We varied the relative detection distances between the groups such that they are equal; and then that raptors are 2, 3 and 4 times better. For each of these we varied the number of raptors from 1-10 relative to the 90 vultures present in the simulation (Table \ref{tab:netlogo}). Vultures can detect carrion at 1 km and other scavengers on a carcass at 4 km \citep{jackson2011evolutionary,pennycuick1972soaring}, the increase in the latter owing to local enhancement \citep{jackson2008effect}. When a vulture discovers a carcass it 'feeds' on it with the model calculating the average amount of food eaten by the vultures at the end of the simulated foraging session. Each simulation was replicated 200 times. We square root transformed our dependent variable data so it would allow us to perform parametric tests and performed the analysis using linear models.
Our simulation results show a significant increase in vulture foraging efficiency with raptor density (figure \ref{fig:netlogo_plot}), indicating that declines in raptor numbers may lead to declines in vulture populations because of a reduced ability to find or open carcasses.

%------------------------------------------

\begin{table}[H]
%\small %!htb keeps the table in this section before moving onto the next block of text
		\caption[NetLogo parameter values] %This goes into  your list of tables
				{Parameter values of the individual-based models that were written in NetLogo.
The first row represents the case of equal detection distance between raptors and vultures; the second is where raptors can see twice the distance and so on. Enhanced range relates to instances where a local enhancement effect is at play, i.e. the carcass is already occupied by a bird.
   
} 
		\input{chap2/tables/netlogo}
		\label{tab:netlogo}
	\end{table}



%------------------------------------------	




%------------------------------------------
% Plot of NetLogo
\begin{figure}[H] %!htb keeps the figure in this section before moving onto the discussion
	  \centering
	  \includegraphics[keepaspectratio,totalheight=0.6\textheight]{chap2/figures/netlogo_plot.pdf}
	    \caption[Results of agent-based model] %This is the label in table of contents
	    {Mean vulture food intake (arbitrary units) as a function of raptor number for four scenarios of differing relative detection distance. (a) equal distance (GLM, $\beta$ = 0.3214, s.e. = 0.1232, p = 0.00916), (b) twice distance (GLM, $\beta$ = 0.3269, s.e. = 0.123, p =0.00794), (c) triple distance (GLM, $\beta$ = 0.5047, s.e. = 0.1198, p <0.001), (d) quadruple distance (GLM, $\beta$ = 0.6052, s.e. = 0.1254, p <0.001). }%this is under the figure
	  \label{fig:netlogo_plot}
	\end{figure}
	
%------------------------------------------	

\section{\uppercase{D}iscussion}

Our results suggest that there is a producer-scrounger game occurring between \textit{Gyps} vultures and scavenging raptors, with the competitive dominance of vultures favouring a scrounging strategy on their part. 
The biology of the two groups further lends itself to the evolution of producer-scrounger dynamics. Flapping flight is far more energetically expensive than thermal soaring for large birds \citep{hedenstrom1993migration} and would prevent vultures from exploring a sufficient area to be effective scavengers \citep{ruxton2004obligate}. Although raptors do exploit thermals as well, their relatively small size allows them to use the weaker early-morning thermals compared with the larger vultures \citep{cone1962thermal}. Thus they are likely to encounter carrion before the vultures. \cite{kendall2013alternative} found that, for their abundance, Tawny Eagles were more likely to discover a carcass than African White-backed Vultures, and R{\"u}ppell's Vultures were never the first to arrive at a carcass which is consistent with producer-scrounger dynamics. She also reported several cases whereby the African White-backed Vultures would not feed at a carcass until a Tawny Eagle began to eat. As mentioned earlier this may be an instance of carcass opening \citep{kendall2013alternative}. The \textit{Gyps} vultures can then dominate the raptor and feed on the previously inaccessible flesh. This would certainly qualify as a producer-scrounger system. A potential follow up to this study would be to include more cameras or observers at the experimental carcasses to note the birds as they arrive from the air. This could be coupled with tracking data to get a better sense of the exact order of arrival of birds in the area.
The proposed dynamics are not the result of an abundance of raptors happening upon carcasses more often than the vultures because raptors occur at much lower densities. In the Masai Mara, for instance, \textit{Gyps} species were recorded at an average density of 85.4 species per 100 km compared with 7.4 for Tawny eagles \citep{virani2011major}.\\
\indent
In sum, we show that foraging behaviour in \textit{Gyps} vultures is more complex than previously thought. Social information transfer flows within and among the vulture and raptor species. In conservation terms, the resultant non-trophic interactions \citep{kefi2012more} mean we should shift our focus to ecosystem-based management \citep{slocombe1993implementing} instead of centring our attention on one species at a time. As our individual-based model shows, in the ecosystem considered here, scrounging vultures will fare poorly with a decline in producing raptors. With raptor populations on the decline \citep{ogada2010decline}, this effect may soon be realised. More generally, we should explore other incidences of socially acquired information transfer between species: inadvertent as it often is, this will be no easy task. 




	 
\chapter{Are vulture restaurants needed to sustain the densest breeding population of the African White-backed Vulture?}
\label{chap:pdp}
%\thesischapter{Are vulture restaurants needed to sustain the densest breeding population of the African White-backed Vulture?}
\textit{Authors:} Adam Kane, Andrew L Jackson, Ara Monadjem, M Angels Colomer \& Antoni Margalida

%Adam Kane$^1$, Andrew L Jackson$^1$, Ara Monadjem$^2$, M Angels Colomer$^3$ \& Antoni Margalida$^3$
\vspace{10 mm}

\noindent
\textit{\uppercase{A}uthor contributions} I conceived the idea, collected the data, incorporated the foraging radius model, helped develop the p-systems model, interpreted the results and wrote the manuscript; AJ gave input on the manuscript; AM supplied some data for carrion availabilty and gave input on the manuscript; MC created the analytical rules for the p-systems model and ran it; AMar developed the p-systems model and commented on drafts of the manuscript.


\vspace{10 mm}

\noindent
\textit{Status:} This manuscript has been published at Animal Conservation.
\newpage

\noindent

\section{\uppercase{A}bstract}

As obligate scavengers, vultures are entirely dependent on carrion resources. In this study we model the carrion ecology of an ecosystem in Swaziland which is home to the densest breeding population of the African White-backed Vulture (\textit{Gyps africanus}). We collected data on life-history parameters of the avian scavenging guild of the area as well as the potential food available from the ungulate fauna. By using novel Population Dynamics P-Systems we show that, despite a closure of supplementary feeding stations in Swaziland, carrion provided by wild ungulate biomass is currently enough to sustain this vulture species. However, in light of forecasted vulture population increases, food will become limiting. We discuss the significance of the cessation of the supplementary feeding sites which now forces these birds to forage farther afield, endangering them to poisoning events. We put these results in the context of carcass biomass management and suggest conservation actions to secure the viability of vulture populations and the important ecosystem services they provide. We argue that the reestablishment of vulture restaurants in Swaziland would lessen their exposure to hazards outside of protected areas.
\newpage

\section{\uppercase{I}ntroduction}

Carrion ecology examines the link between the organic material provided by animal carcasses and ecosystem functioning \citep{barton2013role}. Carcasses are an ephemeral resource and among members of the scavenger community, avian carnivores are probably the group with the most evident adaptive traits to utilise carrion \citep{devault2003scavenging, wilson2011scavenging}. Indeed, the \textit{Gyps} vultures are entirely dependent on dead animal biomass \citep{mundy1992vultures}. Given the patchiness of carrion, vultures are under selection pressures to be as energetically conservative as possible in order to exploit such an unpredictable food \citep{ruxton2002modelling,ruxton2004obligate}.
In sub-Saharan Africa, poisoning, food reduction and habitat loss are key threats to the avian scavenging guild \citep{monadjem2003threatened}. Thus the practice of providing supplementary resources is a common conservation tool \citep{piper1999modelling,piper2005supplementary}. As these 'vulture restaurants' or 'feeding stations' can make carrion more predictable in space and time \citep{oro2008testing}, there are implications for the ecosystem services and foraging behaviour of the species affected \citep{deygout2010impact,margalida2010sanitary,monsarrat2013predictability}. Although these issues have important management and conservation implications, studies focused on the effects of food shortage on vulture species have not been documented until recent times \citep{camina2006griffon,piper2005supplementary,zuberogoitia2010reduced}. \\
\indent
The African White-backed Vulture (AWBV) population of Swaziland is an interesting case study for examining the effect of potential fluctuating food availability given that, at an estimated 300 pairs, it represents the densest nesting population of the birds in the world \citep{monadjem2005nesting}. Historical data show that this AWBV population has increased during the latter part of the 20th century, the reasons for this are unclear \citep{monadjem2003threatened}. In one conservation area (Mkhaya) the species founded a nesting population in the 1980s which grew to 15 pairs in the late 90s \citep{monadjem2003threatened}. Supplementary feeding sites have also been established in the country but of seven vulture restaurants that were operational in Swaziland at the turn of the century only one is currently operational \citep{monadjem2003nesting}. These sites would have provided the vultures with a significant proportion (40\%) of their annual food requirements \citep{monadjem2003nesting}. This represents an apparent paradox because the population of vultures has not declined in the face of this reduction in available food. Thus, we hypothesised that there are enough carcasses from wild fauna to sustain the AWBV population of Swaziland. Our aim was to determine the energy balance for these birds and use this information to inform conservation measures.
In order to address this question we needed to determine where the birds forage, how much carrion they require and how much naturally-occurring food is available to them. 

\section{\uppercase{M}odel building}
\subsection{Study area and species}

The Hlane-Mlawula-Mbuluzi reserve network in Swaziland (figure \ref{fig:radius_map}) contains the majority (at least 202 nests) of the country's AWBV breeding population \citep{monadjem2005nesting}. Located in the east of Swaziland, the area is characterised as Lowveld Savanna \citep{acocks1988veld} with Acacia providing suitable nesting trees for the vultures \citep{monadjem2005nesting}. The rest of the nesting birds of this species are known from other non-contiguous conservation areas (approx. 12 nests at Mkhaya) and protected cattle ranches (19 nests at the Big Bend Conservancy and 6 at IYSIS) \citep{monadjem2005nesting}. AWBVs appear to actively avoid unprotected government ranches in the country for nesting, but their reason for doing so is unclear \citep{monadjem2005nesting}.  In total, Swaziland is home to six other avian scavenger species (obligate and facultative) which take a considerable portion of their diet as carrion. These are the Cape Vulture (which forages but does not nest in Swaziland), the White-headed Vulture (\textit {Trigonoceps occipitalis}), the Lappet-faced Vulture (\textit {Torgos tracheliotos}), the Marabou Stork (\textit {Leptoptilos crumeniferus}), the Tawny Eagle (\textit {Aquila rapax}) and the Bateleur (\textit {Terathopius ecaudatus}).  The majority of the large ungulates in Swaziland (table \ref{tab:carrion_amount}) also live in this Hlane-Mlawula-Mbuluzi reserve \citep{monadjem2003threatened}.

%--------------------------------------
% Map with radius
\begin{figure}[H] %!htb keeps the figure in this section before moving onto the discussion
	  \centering
	  \includegraphics[keepaspectratio, totalheight=0.9\textheight]{chap3/figures/radius_map.pdf}
	    \caption[Map of African White-backed Vulture foraging radius] %This is the label in table of contents
	    {Foraging radius (45 km) of African White-backed Vultures when they have dependants on the nest. The points represent recorded nesting sites with Hlane as the origin of the circles.  Some major national parks are highlighted. The inset shows the possible feeding range (260 km) when there are no young but the bird is still nesting in Swaziland. }%this is under the figure
	  \label{fig:radius_map}
	\end{figure}
	
%------------------------------------------	





%\subsection{Model parameters for avian scavengers}
\subsection{Foraging radius}
We first determined the potential foraging range of the vultures to see which habitats harbouring natural fauna are available to the birds. We used an extension of the central place forager theory known as the foraging radius concept which was developed by Pennycuick \citep{sinclair1995serengeti}. The concept states that every animal is energetically constrained in terms of the spatial range they can cover while foraging \citep{sinclair1995serengeti}. Factors such as cost of movement, basal metabolic rate, presence of dependent young etc. contribute to the overall cost of foraging \citep{ruxton2002modelling}. Many animals must return to a site after they forage every day (as central place foragers). The idea is especially applicable to birds during the breeding season where the origin from which they range is the nest. \\ 
\indent
After foraging the adults must return to incubate the egg, relieve their mate or feed their young. The following is a model describing the energy budget of the closely related R{\"u}ppell's vulture (\textit{Gyps rueppellii}) which was used to estimate the foraging radius of that species \citep{ruxton2002modelling}:

\[r = \frac{Qc + Qdt - T(E_{ma}+\frac{E_{mc}}{2})}{2k +\frac{2Qd}{V}}\]

We employed the same model and applied values from our focal species, the African White-backed Vulture where available. Q is the energy density of the carrion (5.2 x 10$^{-6}$ J) \citep{ruxton2002modelling}; c is the bird's crop capacity (1.2 kg) \citep{houston1975digestive}; d is the digestion rate (0.055 kg hour$^{-1}$) \citep{ruxton2002modelling}; T is the foraging cycle (48 hours) and refers to the fact that the parents take turns to forage with one remaining on the nest each day \citep{mundy1992vultures}; k is the flight cost (2.0 J m$^{-1}$) \citep{pennycuick1972soaring}; V is the flight speed (45 km hr$^{-1}$) \citep{pennycuick1972soaring,tucker1988gliding} and r is the foraging radius. If values for the AWBV were not available we used other \textit{Gyps} species as a close approximation, in this case for t, the foraging time (8 hours) \citep{xirouchakis2007seasonal}; E$_{ma}$ and E$_{mc}$ the adult and chick's metabolic rate respectively (24 Watts and 42 Watts) \citep{sinclair1995serengeti,houston1976breeding}. The energy requirements were calculated using hand-reared individuals but \cite{houston1976breeding} argues the costs between wild and captive birds are similar owing to the low energies that are required for flight. \\
\indent
During the period of nestling dependency the foraging radius of the breeding vultures is estimated at 45 km (figure \ref{fig:radius_map}). If we remove the cost of provisioning the chick it gives a value of approximately 260 km. Thus the foraging radius the vultures have during the time of nestling dependence restricts them to foraging in the national parks of Swaziland.


\subsection{Food requirements of avian scavengers}

The food requirements of individual African White-backed Vultures have been described before \citep{houston1976breeding,mundy1992vultures}. We compiled data on both the adult and nestling energetic requirements for Swaziland using these studies. African White-backed Vultures in Southern Africa typically begin nesting in May-June \citep{monadjem2003nesting} and incubate their single egg for 56-58 days \citep{mundy1992vultures}. Once hatched the altricial chicks are dependent on their parents for a further four months whereupon they fledge from the nest \citep{mundy1992vultures}. The adults are therefore constrained as central place foragers for approximately six months of the year and for four of those they must meet with the additional cost of provisioning the chick. An adult African White-backed Vulture requires approximately 400g of food per day \citep{mundy1992vultures}. The nestling is fed by its parents for approximately four months of the year and consumes a total of 31kg of food during this time (an average of 258g per chick per day) \citep{mundy1992vultures}. Outside of the breeding season adult birds can range away from their typcial nesting sites and may even roost in Kruger National Park (Monadjem pers. comm.). We followed the same approach for the other species in the Swaziland avian scavenging guild (table \ref{tab:food_req}). We also collected data on relevant life history traits such as age at sexual maturity and longevity (table \ref{tab:life_hist}).

\subsection{Model parameters for ungulate carrion}
To determine carrion availability we collected the most recent ungulate population data covering all of Swaziland \citep{monadjem2003threatened}. Ungulate masses and life history traits such as adult and juvenile mortality were taken from the PanTHERIA database \citep{jones2009pantheria}. We also considered the following: 1) 30\% of ungulate mortality is due to direct predation \citep{sinclair1995serengeti}, this is significant because vultures very rarely feed on predator kills and most large mammalian predators are absent from Swaziland \citep{houston1974food,monadjem2003threatened}; 2) 55\% of a carcass is edible by AWBVs \citep{sinclair1995serengeti} because this species consumes the viscera and soft muscles  \citep{kruuk1967competition}; 3) 15\% of the edible carcass is consumed by invertebrates and bacteria \citep{sinclair1995serengeti}; 4) AWBVs avoid montane environments (cf. Cape Vultures, \textit {Gyps coprotheres}) and as such the ungulate population in the Highveld of Swaziland could be discounted from the total carrion available (A. Monadjem unpubl. data.); 5) the total carrion available annually in Swaziland is not evenly spread throughout the year; the dry season of May to August claims approximately half of the ungulate dead \citep{sinclair1995serengeti}. Therefore, only half of this total is spread over the remaining eight months; and 6) Given that the ungulates are contained within actively managed reserves, where culling takes place, we set the population of each species as constant using the surveyed numbers reported by \cite{monadjem2003threatened}.


\subsection{Population dynamics P system model}

Using this information we developed a population dynamics P system (PDP) model \citep{colomer2013population}. This integrated data on food availability, food requirements and population dynamics of the avian scavenging guild and the ungulate populations of Swaziland to determine if carcass availability could meet the demands of the AWBV population over a 20 year period (AWBVs have been recorded as living for 20 years \citep{de2009database}. PDP models are computational methods that are analogous to the machinery of cells \citep{colomer2013population}. 

%------------------------------------------
% Schematic of P-system cell
\begin{figure}[H] %!htb keeps the figure in this section before moving onto the discussion
	  \centering
	  \includegraphics{chap3/figures/psystem}
	    \caption[Schematic of P-system cell] %This is the label in table of contents
	    {Schematic of P-system cell showing a)the cell and its components b) the hierarchy of the membrane structure and c) how this structure is written analytically. Figure is taken from \cite{colomer2013population}.}%this is under the figure
	  \label{fig:psystem}
	\end{figure}
	
%------------------------------------------	


The analogy of a PDP system to a real ecosystem has been drawn before \citep{colomer2013population,colomer2011bio,margalida2011can} and serves well to illustrate the intuition behind this relatively new method. The cells of the model correspond to the physical space of the environment (figure \ref{fig:psystem}). Animals (which, along with things like resources, are represented by model 'objects') will feed, reproduce, develop etc. within an environment which is accounted for by a set of mathematical rules describing these behaviours in the model. Just as animals can move between different areas when circumstances become unfavourable (e.g. food shortages) so their simulated counterparts can migrate between the different spatial environments of the model (e.g. between South Africa and Swaziland). The membranes within the cells of an environment separate out specific processes that are applied to the objects in the model (e.g. the different rules associated with different seasons) (see supplementary information 'PDP Model Components'). The advantage of this approach is that PDP models can integrate a large volume of information and compute the output of a large number of species in parallel and in a relatively short time. In addition, PDP models have been developed and applied to similar ecosystems before \citep{margalida2012modelling}. The syntax of the model rules take the following form:
\[r \equiv \left(x\right)_{e_{j}} \xrightarrow{p\left(r\right)}\left ( y_{l} \right )_{e_{jl}}\]
This is an example of a rule describing the movement of an object x (e.g. an animal) from environment e$_{j}$ to e$_{jl}$  where it becomes object y (e.g. a mature version of an animal). The p(r) element is a function stating the probability of the rule taking place \citep{colomer2013population}. The symbol $\equiv$ means the rule is equal by definition to what follows. The model was implemented in the program MeCoSim \citep{perez2010mecosim}. Please see the supplementary information for the rules of this model. 

%------------------------------------------
% Flow diagram for P-systems
\begin{figure}[H] %!htb keeps the figure in this section before moving onto the discussion
	  \centering
	  \includegraphics{chap3/figures/flow_plot.pdf}
	    \caption[Flow diagram for PDP system model] %This is the label in table of contents
	    {Flow diagram representing changes in the status of the avian scavengers over time. The italicised text show conditionals for food availability which determine the path the birds can take. Starting at the 'Reproduction' stage at the upper left hand side this represents the point at which the birds have their young. They undergo 'Natural Mortality' during this period and 'Feeding'. If there is enough food during this time they can progress to period 2, otherwise they perish and so on through the year until they reproduce again. The 'Movement' stages indicate points whereby the birds can leave Swaziland because they are not constrained by the presence of young. Some of the explicit model rules are highlighted at the boxes for movement, reproduction, mortality and feeding. These are described in detail in the supplementary material chapter for this section. Note that not all rules are laid out here such as those involved in the setup and resetting of the model.}%this is under the figure
	  \label{fig:flow_plot}
	\end{figure}
	
%------------------------------------------	
There are two environments in our model delimited by the foraging radius of the birds when they are on the nest as defined in our foraging radius calculations. So when the reduced foraging radius of the birds is no longer applicable, the birds are able to forage in the new environment where food is not limiting (figure \ref{fig:flow_plot}). This restricts the birds to game reserves in Swaziland when they have a reduced foraging potential. A bird outside the breeding season could range hundreds of kilometres on a single foraging trip. A year in the model was divided into four temporal periods: Period 1 = July-August, Period 2 = September-October, Period 3 = November-April, Period 4 = May-June. These periods reflected differences in food availability, in foraging ranges and in food requirements (table \ref{tab:season_change}). 

\begin{table}
\small %!htb keeps the table in this section before moving onto the next block of text
		\caption[Carcass data] %This goes into  your list of tables
				{Identity, number of species and average carrion provided per year of the ungulate population of Swaziland.} 
		\input{chap3/tables/carrion_amount}
		\label{tab:carrion_amount}
	\end{table}

\section{\uppercase{R}esults}
%\subsection{Foraging radius}

%------------------------------------------
% Carrion dynamics over time
\begin{figure}[H] %!htb keeps the figure in this section before moving onto the discussion
	  \centering
	  \includegraphics[keepaspectratio]{chap3/figures/Carrion_plot.pdf}
	    \caption[Balance of carrion availability across time in Swaziland] %This is the label in table of contents
	    {Carrion balance across time in the different periods of the year. The horizontal bar represents the point below which the demands of the avian scavenging guild are no longer met by the carrion provided by wildlife in Swaziland. Periods 1 and 2 are the life stages when young are present and food requirements are higher as a result. }%this is under the figure
	  \label{fig:Carrion_plot}
	\end{figure}
	
%------------------------------------------	


According to our model most of the biomass available to the scavengers is provided by impala (\textit{Aepyceros melampus}) (33 \%), blue wildebeest (\textit{Connochaetes taurinus}) (13 \%) and zebra (\textit{Equus burchellii}) (9 \%) with the remaining 21 species contributing the remaining carrion (table \ref{tab:carrion_amount}). Initially, only period 3 (from November to April) has insufficient food for the scavenging guild but eventually as the scavenger populations grow, periods 1 (July-August) and 2 (September-October) see a net food deficit at year 5 and 13 respectively. Period 4 (May-June) also develops a decline, although at a shallower rate (figure \ref{fig:Carrion_plot}). Period 3, which never sees enough carrion to sustain the scavengers, is a time when the birds can forage outside of Swaziland.  In general the avian scavengers see an increase in their number over the 20 year run of the model (although, note that stochastic effects see the single pair of Lappet-faced vultures go extinct in the system) (figure \ref{fig:Scavengers_plot}). 



%------------------------------------------	

% Scavenger pop dynamics over time
\begin{figure}[H] %!htb keeps the figure in this section before moving onto the discussion
	  \centering
	  \includegraphics[keepaspectratio, totalheight=0.6\textheight]{chap3/figures/Scavengers_plot.pdf}
	    \caption[Avian scavenging guild population dynamics in Swaziland] %This is the label in table of contents
	    {Number of pairs of avian scavengers living in Swaziland across time. All species except the Lappet-faced Vulture see a population increase due to the amount of wild carrion available. This one extinction occurs owing to its tiny initial population size. The bottom panel shows the AWBV trend separately for reasons of clarity because of its much larger population size. }%this is under the figure
	  \label{fig:Scavengers_plot}
	\end{figure}
	
%------------------------------------------	


\section{\uppercase{D}iscussion}

The key prediction from our model is that the carrion of Swaziland is sufficient to cover the energetic requirements of the current AWBV population and most of the other scavenging avifauna (figure \ref{fig:Scavengers_plot}), but only for the time being (figure \ref{fig:Carrion_plot}). The trend of the energy balance makes it clear that, as vulture numbers rise, food will soon become a limiting factor. There is already a predicted net deficit in energy balance for six months of the year, from November-April (figure \ref{fig:Scavengers_plot}). Fortunately this is a period during which the birds can forage outside of Swaziland. However, this reduction in available food could force the birds to spend more time foraging outside of protected areas, increasing the risk of non-natural mortality factors, like poisoning, with important consequences on population dynamics if adult survival is affected \citep{monadjem2014effect}. Although our model suggests that AWBVs in Swaziland will see an increase in their population for a while, the temporal food shortages identified could reduce their breeding output during this period of growth. In Europe, for example, after the outbreak of bovine spongiform encephalopathy in 2001, carcasses were destroyed in authorised plants which reduced the amount of food available to the vultures \citep{margalida2010sanitary}. A long-term study on bearded vultures (\textit{Gypaetus barbatus}) showed this reduction provoked a delay in laying dates, a regressive trend in clutch size, breeding success and survival following this policy change \citep{margalida2014man}. 


We can be confident in our predictions given that the radius we obtained for the vulture foraging movements is consistent with values reported from the literature. For instance the average movement recorded in the Serengeti was 51 km and in Kruger 34 km \citep{mundy1992vultures}. A more recent study of immature AWBVs showed a mean distance travelled per day of 33.39 km \citep{phipps2013foraging}. It has been noted that, while nesting without a chick, the birds can fly over 240 km between the nest and a carcass and this is consistent with our value of 260 km \citep{houston1975digestive}. Kruger National Park is almost 100 km from Hlane, the main site of the nesting birds, and smaller reserves such as Mawewe Cattle/ Game Project are all in excess of the radius. 
\indent
 
 This indicates that the original vulture resturants in Swaziland that have since closed were not necessary in terms of providing food. However this is not to say they are unneeded. Indeed, these findings suggest further intervention, and modification of existing strategies will be required. Vulture restaurants could be established and stocked more frequently during times of food deficiency. Although there are a number of problems associated with supplying supplementary food to wild populations, such as a conditioned dependence on supplemental food \citep{robb2008food}, a well-managed vulture restaurant system could minimise these issues while maximising the benefits. For instance a study of \textit{Gyps fulvus}, showed "[f]eeding stations were particularly used when resources were scarce (summer) or when flight conditions were poor (winter), limiting long-ranging movements" \citep{monsarrat2013predictability}. Supplementary feeding can minimise the risk of poisoning (increasing survival) that follows from species foraging outside of protected areas which has been documented in other vulture species \citep{oro2008testing}. 
Swaziland seems well placed to act as a habitat for a greater number of AWBVs than that of the current population. With the species on the decline globally it is incumbent on us to secure this valuable population. By ensuring that there is no deficit of carrion at any stage of the year (figure \ref{fig:Carrion_plot}) we would give the birds the best chance to flourish in this area. Indeed it could act as a source population for other suitable areas in the region as its population increases. \\
\indent
There is another advantage to creating a long term vulture restaurant in that it would create the opportunity to capture and tag these birds. High resolution data on the foraging behaviour of this population specifically and AWBVs in general are lacking. We know little of their age class structure and whether other vagrant populations visit Swaziland without nesting in the country \citep{monadjem2003threatened}. Tracking data would improve our knowledge about these issues allowing managers and policy-makers to adopt more objective decisions based on the evidence. 
The value of theoretical modelling, like PDP P systems, is also underscored by the results generated in this study and others like it \citep{margalida2012modelling,margalida2011can}. We should note that any model is only as good as the data used to parameterise it. Consequently basic and up to date biological data are of utmost importance if we are to derive accurate predictions. These methods are another tool for stakeholders to use in identifying threats and solutions to conserving the target species. \\
\indent
The conservation implications obtained with this theoretical approach are that the carrying capacity of African White-backed Vultures in Swaziland is reaching maximum values according to natural food provided by the ecosystem. The dependence of the birds on food resources provided by neighbouring areas shows the importance of international agreements for conservation and the coordination of management actions \citep{lambertucci2014apex}. The establishment of well-managed vulture restaurants in Swaziland should be seriously considered. 





\chapter{Did \textit{Tyrannosaurus rex} tend towards scavenging with age?}
\label{chap:trex}

%\thesischapter{Ontogenetic change in Tyrannosaurus rex did not lead to it adopting a scavenging niche}
\textit{Authors:} Adam Kane, Kevin Healy, Graeme Ruxton \& Andrew Jackson
% Adam Kane$^1$, Kevin Healy$^1$, Graeme Ruxton$^2$ \& Andrew Jackson$^1$

\vspace{10 mm}
\noindent
\textit{\uppercase{A}uthor contributions}
I conceived the idea, collected the data used for the scaling relationships, created and ran the IBM, interpreted the results and wrote the manuscript; KH collected the data used for the scaling relationships, calculated their output, helped with interpretation of results and commented on the manuscript; GR supplied data and commented on an early draft of the manuscript; AJ gave feedback on the manuscript. 


\vspace{10 mm}

\noindent
\textit{Status:} This manuscript is being prepared for publication. Target journal is Biology Letters.

\newpage

\noindent

\section{\uppercase{A}bstract}

The feeding ecology of \textit{Tyrannosaurus rex} remains controversial, with polarised suggestions of either a predatory or scavenging lifestyle. We move away from this dichotomy and propose the animal incorporated more carrion in its diet as it aged. We hypothesise this to be the case owing to factors related to its drastic morphological change across ontogeny notably an increased availability of carrion through dominance of interspecific competitors at carcasses; by exploiting resources unavailable to its smaller competitors such as skeletal material; and by having a larger capacity to store meat as it grew. We develop an individual based model to address our question. Surprisingly, our results show \textit{T.rex} was very unlikely to have specialised on carrion at any life stage because the cost of movement would have been prohibitively expensive as it increased in mass. In spite of our suggested advantages, scavengers require a more economical model of locomotion while foraging than \textit{T.rex} could have managed.



%Scavengers need an economical mode of locomotion to cover wide areas while foraging. This was not true for \textit{T.rex}.
\newpage

\section{\uppercase{I}ntroduction}

Animals can avoid direct competition with conspecifics by foraging at different times and locations, or by targeting different resources. This partitioning of resources can occur across ontogeny when the life stages are distinct; so, for example, an agile juvenile may focus on smaller, faster prey than a less nimble adult. Ontogenetic dietary partitioning is known across a diversity of species including crocodiles, insects, fish etc. \citep{platt2006food,winemiller1989ontogenetic,steyn1980breeding,hirai2002ontogenetic,knoff2008ontogenetic} and is often related to changes in body size \citep{woodward2002body} but can also arise through other means such as different habitat use patterns \citep{carrier2010sharks}.
Here we suggest that such partitioning may have taken place in the theropod dinosaur \textit{Tyrannosaurus rex}. \textit{T. rex} fits into theropod 'morphotype one' as defined by Weishampel \citep{weishampel2004dinosauria}. The carnivorous animals in this category are extremely large, exceeding 10 m in length, have huge skulls, short forelimbs and as such have no living analogues which makes it difficult to draw any conclusions about their mode of life \citep{weishampel2004dinosauria}. 
However, as a species, \textit{T.rex} is atypical because of the large number of fossils available for study \citep{brusatte2010tyrannosaur}. So, for instance, it is known that it exhibited remarkable rate of growth through its development \citep{brusatte2010tyrannosaur} and underwent dramatic ontogenetic changes in its morphology which has led some researchers to argue for a concomitant change in its ecological habits \citep{brusatte2010tyrannosaur}. For instance, adults had a powerful, deep, robust skull with thick teeth in contrast with the more gracile features of juveniles. One suggested result of this change was a dietary shift from fast prey items to larger more cumbersome herbivores \citep{brusatte2010tyrannosaur}. However, we instead explore whether such marked changes caused an increased tendency towards scavenging given that a scavenging lifestyle has been proposed for the animal in recent times \citep{horner1993complete, ruxton2004obligate}. \\
\indent
\cite{ruxton2004obligate} argue there is a "a clear evolutionary pressure towards large size for both birds and mammals when they are feeding on large carcasses. This is largely because larger animals can consume more food from each discovered food fall, and carry greater body reserves, than can small ones."
A pressure that lends support to our proposal. Indeed, a number of recent studies have looked at the theoretical possibility of obligate scavenging across species including \textit{T. rex} \citep{ruxton2004energetic,ruxton2013endurance,ruxton2005searching,ruxton2004obligate,ruxton2003could,carbone2011intra}. For example, Horner pointed to its slow speed, reduced forearms, large olfactory bulb and incredible bite force, to support this view \citep{horner1994steak,horner1993complete}. Brown hyenas (\textit{Hyaena brunnea}) can detect carcasses 2 km downwind \citep{mills1984comparative} and the olfactory bulbs of \textit{T. rex} indicate an impressive ability in this respect \citep{witmer2009new}. Fossil material from a herbivore has also been discovered with bite marks located in a flesh-poor region suggesting a scavenging event after the preferred sections of the animal were consumed \citep{longrich2010cannibalism}. Energetic approaches have concluded both for \citep{ruxton2003could} and against \citep{carbone2011intra} \textit{T.rex} being a scavenger. The latter study argued that interspecific competition would undermine the possibility of a scavenging \textit{T.rex} since smaller, more numerous competing species would find and consume any carcass before the larger \textit{T.rex} could benefit. Obligate scavengers are rare among terrestrial vertebrates with even the most common terrestrial example, the brown hyena, displaying a large degree of variation between scavenging behaviour and active predation \citep{devault2003scavenging,sinclair1995serengeti}.  Instead, most carnivores are opportunistic and will take carrion as well as actively hunting prey \citep{devault2003scavenging} and mounting evidence suggests that \textit{T.rex} behaved similarly to extant opportunistic predators such as hyenas. The most compelling evidence for a predatory lifestyle comes in the form of a subadult \textit{T.rex} tooth that was found embedded in a hadrosaur tail \citep{depalma2013physical}. The wound had healed around the tooth indicating that the prey item escaped and so this was an active predation attempt \citep{depalma2013physical}. \\
\indent
In this study we move away from the polarised predator-scavenger debate and look at the possibility that \textit{T.rex} underwent an ontogenetic dietary shift, increasing the proportion of carrion in its diet as it aged. We hypothesise this to be the case on the basis of increased availability of carrion through domination of interspecific competitors at carcasses, by exploiting resources unavailable to its smaller competitors and by having a larger capacity to store meat as it increased in size ontogenetically.  This shift would not only see adult \textit{T.rex} avoiding intraspecific competition with younger conspecifics but also with interspecific carnivores of the time. Among contemporary competitors from the Late Maastrichtian Hell Creek formation, the one tonne \textit{Albertosaurus} may have been an immature \textit{T.rex} and \textit{Troodon} and other dromaeosaurids were wolf-sized creatures that would be easily dominated by an adult \textit{T.rex} \citep{horner2011dinosaur,carr2004diversity,farlow2002body}. Their small size also means they would leave much of the skeletal material of prey/carcasses untouched. 
However, as alluded to earlier, the morphology of an adult \textit{T. rex} skull suggests an ability to process bone which would be a great boon to a scavenging animal if it were capable of monopolising this resource. Direct evidence comes in the form of distinctive wear marks on its tooth apices \citep{farlow1994wear,schubert2005wear}. The animal also had an enormous bite force, with one estimate putting it at 57000 Newtons \citep{bates2012estimating}. This is noted as being powerful enough to break open skeletal material during feeding \citep{rayfield2001cranial}. A positive allometric scaling relationship in bite performance during ontogeny has also been recorded \citep{bates2012estimating}. This is a similar pattern to that observed in loggerhead sea turtles (\textit{Caretta caretta}) where the bite force exhibits positive allometry across ontogeny allowing adults access to hard benthic prey, a resource inaccessible to other durophagous competitors \citep{marshall2012ontogenetic}. Further, \textit{T. rex} coprolites were discovered with bone fragments, prima facie evidence that it did consume bone \citep{chin2003remarkable}. Osteophagy is known in extant taxa such as the Bearded Vulture (\textit{Gypaetus barbatus}) and hyena species \citep{hone2010feeding}.  Some fat-rich mammalian bones have an energy density (6.7 kJ/g) comparable with that of muscle tissue \citep{margalida2008bearded,brown1989study}, making skeletal remains an enticing resource for any scavenger that could process them.  \\
\indent
We use an energetics method based on a series of scaling relationships to look at how the energy balance of \textit{T.rex} changed across ontogeny and test whether this could have pushed it towards obligate scavenging. We parameterise an agent-based model which better captures the impact of competition than the numerical approaches of previous studies. 

\section{\uppercase{M}ethods}
\subsection{Model parameters}


We took estimated parameter values from the literature for mass (kg), hip height (cm), detection distance (km), search time (fraction of day), energy content of bone (kJ/kg), energy content of flesh (kJ/kg), density of carcasses (kg/km$^2$/day), the size of competitors and the size of carcasses etc. to estimate several scaling relationships describing theropod foraging behaviour including walking speed (m/s), basal metabolic rate (W) and cost of transport (J/m). This is basically an 'energy in' minus 'energy out' calculation i.e. net energy gain (table \ref{tab:model_param}). 

%------------------------------------------

\begin{table}[H]
\small %!htb keeps the table in this section before moving onto the next block of text
		\caption[Model Parameters] %This goes into  your list of tables
				{Model parameters and their values for \textit{T.rex} foraging models.  
} 
		\input{chap4/tables/model_param}
		\label{tab:model_param}
	\end{table}

%------------------------------------------

The daily costs of theropods of various masses were derived using the scaling relationship of mesothermic dinosaurs from \cite{grady2014evidence} for basal metabolic rates. We used these for the \textit{T. rex} ontogenetic stages. We then took the scaling relationships outlined in \cite{pontzer2009biomechanics} to calculate the cost of searching during foraging. This was estimated using the mass-specific locomotor cost of transport that is based on the hip height of the animal, its mass and its speed to calculate the cost of searching over a 12 hour day (table \ref{tab:model_param}). Specifically, the cost of transport, in joules, is the energy required to move 1 kg of mass over 1 metre. From this we can calculate the cost for the whole animal: cost of transport x mass x walking speed (table \ref{tab:scav_mass}). 

We used an estimate of energy density of carrion from the Serengeti as an analogue for terrestrial Mesozoic systems \citep{ruxton2003could,carbone2011intra}. The amount of food available is derived as a function of the energetics of the food source. The energy available from the carcasses in flesh and bone was calculated using allometric scaling relationships (table \ref{tab:model_param}). We used the size bins of a previous study for both the body mass of the prey items and the predators (table \ref{tab:dino_mass}) \citep{carbone2011intra}. So for example, 49.3\% of carrion mass is packaged up in 75 kg carcasses. \cite{carbone2011intra} use mass-abundance relationships (e.g. Damuth's Law) to determine the actual population size of the various species and we followed their approach here. 

%------------------------------------------

\begin{table}[H]
\small %!htb keeps the table in this section before moving onto the next block of text
		\caption[Dinosaur carcass sizes] %This goes into  your list of tables
				{Mass categories for carcasses and allometric scaling of bone with body mass across carcass categories. The contribution column defines the percentage of carrion that is made up by a given species category.  
} 
		\input{chap4/tables/dino_mass}
		\label{tab:dino_mass}
	\end{table}

%------------------------------------------

\begin{table}[H]
\small %!htb keeps the table in this section before moving onto the next block of text
		\caption[Scavenger mass categories] %This goes into  your list of tables
				{Mass categories, gut capacities, walking speeds and daily energetic cost for the ontogenetic stages of \textit{T. rex} and its competitors as used in the individual-based model.  
} 
		\input{chap4/tables/scav_mass}
		\label{tab:scav_mass}
	\end{table}

%------------------------------------------

\subsection{Individual-based model}
We created a spatially explicit agent-based model using the above calculated parameters to allow us better understand the effect of competition on the ontogenetic stages of \textit{T.rex} if they were restricted to scavenging (table \ref{tab:dino_mass}). Our model was designed in the program NetLogo \citep{tisue2004netlogo}. The simulation space was a 50x50km square corresponding to a 2500km$^2$ landscape. Unfortunately, it is not clear if there was a habitat partitioning across ontogeny in Tyrannosaurus which may have implications for factors such as transport costs \citep{ruxton2014energetic}. 

The model had a series of mobile agent types which corresponded to \textit{T.rex} ontogenetic stages and its competitors (table \ref{tab:scav_mass}). There was one \textit{T.rex} ontogenetic stage per run e.g. a 100 kg individual. The initial state of the model had all dinosaurs and carcasses located randomly in the environment. The dinosaurs then set off in a random direction at their assigned speed searching for carrion (table \ref{tab:scav_mass}). They maintained the same walking speed for the duration of the model and had a constant turning rate. Upon finding a carcass in its visual field (500 m) the \textit{T.rex} walked towards it and started to feed, extracting energy until sated as defined by its gut capacity (table \ref{tab:model_param}) or until the carcass was consumed. The competitors behaved similarly only they were incapable of feeding on the bones of carcasses. If nothing remained for the animals to eat and they were not entirely sated they began to forage again. If they were full they moved away from the carcass and wandered until the next day whereupon their gut capacity went back to 0. The gut passage time of \textit{T.rex} is difficult to ascertain but we know it was relatively quick given the state of consumed material in its coprolites \citep{chin1998king,chin2003remarkable}. Without being fed upon, a carcass decayed down to bone and then to nothing because of mammalian scavengers, invertebrates, bacteria etc. \citep{sinclair1995serengeti}. This process took seven days \citep{carbone2011intra}. Competition in our model followed a simple, 'if bigger than me, then avoid' rule, and applied to the interaction between the \textit{T.rex} life stages and their competitors. A foraging day in our model was 12 hours. At the end of the day the 'energy in' of the \textit{T.rex} was subtracted from the 'energy out' (table \ref{tab:scav_mass}). We ran the model for 200 days for each of the seven \textit{T.rex} life stages.  


\section{\uppercase{R}esults}
Our results show that under the conditions of our model \textit{T.rex} was in negative energy balance if it were an obligate scavenger. We can see from figure \ref{fig:trex_energy} that, in contrast to our hypothesis, the net energy intake of \textit{T.rex} from scavenging decreases with their body mass. The horizontal dashed line represents the point where the energy out is balanced by the energy in. Even the average of the smallest ontogenetic stage (a 100 kg individual), falls below this line. 
%------------------------------------------

\begin{figure}[H]
\centering
\includegraphics[keepaspectratio, totalheight=0.7\textheight]{chap4/figures/trex_energy}
		\caption[\textit{T.rex} energy balance] %This goes into  your list of tables
				{The energy balance of the \textit{T.rex} ontogenetic stages. The dashed horizonal line represents the point, below which, the animal has a negative energy balance. Each point in the figure is the mean of the 200 day model run for the given body size with its standard deviation.   
} 
		\label{fig:trex_energy}
	\end{figure}

%------------------------------------------

\section{\uppercase{D}iscussion}
In order to be an effective terrestrial vertebrate scavenger an animal must be able to traverse huge distances at low cost \citep{ruxton2004obligate}. This is why soaring flight allows vultures to occupy the obligate scavenger niche. The cost of locomotion seems to be one of the main barriers precluding \textit{T.rex} from being an obligate scavenger. A lot of energy would have been required to propel an eight tonne adult around its habitat in search of a patchily distributed resource. The increasing variation in energy balance seen in figure \ref{fig:trex_energy} is a result of gut capacity and carcass size distributions. The smaller individuals can fill their gut to capacity every time they encounter a carcass; by contrast the larger animals will only rarely encounter a carcass of sufficient mass to sate them entirely. As such it would be interesting to investigate the effect of different carcass mass distributions on \textit{T.rex} scavenging behaviour. Our model is limited in depending on data from long extinct ecosystems. Consequently there are many unknowns. We do not know how carrion was distributed for instance or if there was habitat partitioning across the ontogeny of \textit{T.rex}. The costs of movement for the \textit{T.rex} stages are based on the most recent studies of dinosaur metabolism \citep{grady2014evidence} but it is unlikely that we will ever know the species level adaptations that some animals had such as the ability of vultures to decrease their metabolic rate at night \citep{ruxton2002modelling}. However, the model is flexible and sensitivity analyses can be readily performed to account for the large margin of error when dealing with prehistoric systems. A development of the model could allow for differing detection radii and turning rates for example.  \\
\indent
Ecologically, \textit{T.rex} was unusual. Census records show it to have had a population higher than would be expected for a typical apex predator suggesting that its mode of life was atypical with Horner and colleagues positing a hyena-like niche for the animal \citep{horner2011dinosaur,mills1984comparative}. The radical change in morphology with ontogeny would then appear to represent a dietary niche shift which has been suggested previously \citep{brusatte2010tyrannosaur}. The change from a gracile to robust form may have allowed the animal to switch its targets from small prey items to the huge armoured herbivores of the time (table \ref{tab:dino_mass}). Species such as \textit{Triceratops} and \textit{Ankylosaurus} would have made formidable opponents for a predaceous \textit{T.rex}. To be able to dispatch such prey items quickly with its enormous bite force would be of great advantage. It has been suggested that an ambush strategy seems the most probable gambit it used to hunt \citep{krauss2013biomechanics}. It could rush from cover and use its arms to hold a prey item while biting it \citep{krauss2013biomechanics}. This is not to discount scavenging entirely. Such was its likely competitive dominance \textit{T. rex} may well have been an effective kleptoparasite. Spotted hyenas (\textit{Crocuta crocuta}) can get over a quarter of their carcasses by stealing from other predators \citep{curio1976ethology}. 
Although we have framed our study such that it focuses on \textit{T.rex}, the same energetic barriers likely prevented other theropods in the morphotype one category \citep{weishampel2004dinosauria} from evolving into scavenging specialists. 




\chapter{Discussion}
\label{chap:discussion}

%---------------------
%
% DISCUSSION - MAKE IT SHORT TO!
% 
%---------------------

\section{The future of the Total evidence method}
Combined with tip-dating is super interesting but we need more data.
To do so we can use plateforms such as morphobank and foster collaboration on big projects.
Also we can make all the data available blablabla.

However, there are some limitations:
Maybe tip-dating isn't that good? Compared to the nice recent node dating models... (Arcila)
Also the Mk model is really crude and overly simplistic.

One way to improve could be a REAL total evidence dating using also trait data, biogeography, etc...
In reality, all this parameters have an influence of lineages history and should technically be taken into account.
But data problem is likely to increase, an needs models need to be improved as well.
And in the end, how many parameters do we want?

\section{Diversity is multidimensional}
Diversity is often just seen as the sheer number of species.
However, the processes that led to this pattern is fundamentaly intangled with all the other aspects of diversity.
For example, specious rich groups have also so traits, etc...
It is important to disentangle.
But other dimensions as well: Ecological, life history, etc.
We need to take into account more of these "disparity" patterns to really understand what happened.
Especially when combining living and fossil, species richness is a really poor indicator of diversity.

However, this is more complex, species diversity is easy to interprate (many populations isolations through time) but disparity is a bit harder.
What IS disparity? What metric to use? How to express the changes etc...
Also, all these metrics are just using proxies.

But  The statistician George Box wrote "essentially, all models are wrong, but some are useful" \citep{box1987empirical}. % Modify
This is still really promising and can be improved first by underestanding how all this works in a theoretical way (building the models).
And only then apply it to observed patterns.

\section{What is the real effect of combining?}
Maybe only important when groups have actually a complex history?
Old clades might have no living descendants and the question is therefore N/A
Recent subclades maybe not have changed much in diversity so adding fossils might not change much.
But we never know! Example of the giant lemur (recently extinct).

 %\pagestyle{empty}% attempt to get rid of chapter 6 headers in bib
\formatbibliography
\bibliographystyle{PLoS-Biology} 
\bibliography{bibfile} 


%\formatbibliography %note there are only a few refs in this example in the appendix
%\bibliographystyle{refstyle} %you may want to make a new reference style. This is refstyle.bst
%\bibliography{refs-thesis} %you may also call you bibliography something different. This is refs-thesis.bib



\formatappendices
\chapter{Vultures acquire information on carcass location from scavenging eagles - Supplementary information}
\label{chap:scrounger}

\begin{table}[H]
\small %!htb keeps the table in this section before moving onto the next block of text
		\caption[Description of parameters used in the game theory model] %This goes into  your list of tables
				{Description of parameters used in the game theory model} 
		\input{scrounge_supp/tables/game_tab}
		\label{tab:game_tab}
	\end{table}

	\newpage

%\begin{figure}[h]
\includepdf[pages={-}, ,scale=0.85]{scrounge_supp/figures/paper.pdf}
%\end{figure}

%\newpage

%\begin{figure}[h]
%\includepdf[pages={2}]{scrounge_supp/figures/paper.pdf}
%\end{figure}

%\newpage

%\begin{figure}[h]
%\includepdf[pages={3}]{scrounge_supp/figures/paper.pdf}
%\end{figure}

%\newpage

%\begin{figure}[h]
%\includepdf[pages={4}]{scrounge_supp/figures/paper.pdf}
%\end{figure}

%\newpage

%\begin{figure}[h]
%\includepdf[pages={5}]{scrounge_supp/figures/paper.pdf}
%\end{figure}

%\newpage

%\begin{figure}[h]
%\includepdf[pages={6}]{scrounge_supp/figures/paper.pdf}
%\end{figure}

%\newpage

%\begin{figure}[h]
%\includepdf[pages={7}]{scrounge_supp/figures/paper.pdf}
%\end{figure}


\chapter{Are vulture restaurants needed to sustain the densest breeding population of the African White-backed Vulture? - Supplementary information}
\label{chap:pdp-supp}

%------------------------------------------	

\begin{table}[!htb]
\small %!htb keeps the table in this section before moving onto the next block of text
		\caption[Life history data] %This goes into  your list of tables
				{Life history parameters for all of the species in the community under study, values are in years \cite{brown1991declining,pennycuick1976breeding,jones2009pantheria,monadjem2012survival,piper1999modelling,de2009database}.} 
		\input{pdp_supp/tables/life_hist}
		\label{tab:life_hist}
	\end{table}

%------------------------------------------	

\begin{table}[H]
\small %!htb keeps the table in this section before moving onto the next block of text
		\caption[Avain scavenger reproductive parameters] %This goes into  your list of tables
				{Reproductive parameters for species of the avian scavenging guild (number of descendants corresponds to the number of fledglings for the following species) \cite{mundy1992vultures,monadjem2005nesting,mundy1982comparative,margalida2012modelling}.} 
		\input{pdp_supp/tables/repro_param}
		\label{tab:repro_param}
	\end{table}

%------------------------------------------	

\begin{table}[H]
\small %!htb keeps the table in this section before moving onto the next block of text
		\caption[Avain scavenger mortality parameters - adults] %This goes into  your list of tables
				{Mortality values for adults of the avian scavenging guild over a year \cite{brown1991declining,pennycuick1976breeding,monadjem2012survival,piper1999modelling,monadjem2013survival}.} 
		\input{pdp_supp/tables/adult_mort}
		\label{tab:adult_mort}
	\end{table}

%------------------------------------------	

\begin{table}[H]
\small %!htb keeps the table in this section before moving onto the next block of text
		\caption[Avain scavenger mortality parameters - immatures] %This goes into  your list of tables
				{Mortality values for immatures of the avian scavenging guild over a year \cite{brown1991declining,pennycuick1976breeding,monadjem2012survival,piper1999modelling,monadjem2013survival}.} 
		\input{pdp_supp/tables/imm_mort}
		\label{tab:imm_mort}
	\end{table}

%------------------------------------------



\begin{table}[H]
\small %!htb keeps the table in this section before moving onto the next block of text
		\caption[Carrion mass] %This goes into  your list of tables
				{Meat available for scavengers from carrion providing species \cite{jones2009pantheria,sinclair1995serengeti}.
} 
		\input{pdp_supp/tables/carrion_mass}
		\label{tab:carrion_mass}
	\end{table}

%------------------------------------------

\begin{table}
\small %!htb keeps the table in this section before moving onto the next block of text
		\caption[Population density of animals in ecosystem] %This goes into  your list of tables
				{Initial and maximum density of all the species considered in the ecosystem/ model. For the birds  we calculated the area potentially available to each species in Swaziland and then used recorded density estimates to determine the maximum number we would expect in such an area. Initial is equivalent to max density for the ungulates given our model assumptions mentioned above. The birds are in pairs, the ungulates are individuals \citep{monadjem2003threatened}.   
} 
		\input{pdp_supp/tables/pop_den}
		\label{tab:pop_den}
	\end{table}

%------------------------------------------

\begin{table}[H]
\small %!htb keeps the table in this section before moving onto the next block of text
		\caption[Avain scavenger food requirments] %This goes into  your list of tables
				{Daily Food Requirements for the avian scavenging guild. Marabou Stork food requirements are taken to be the same as the similarly sized Cape Griffon \cite{mundy1992vultures,mundy1982comparative,calder1996size}.  
} 
		\input{pdp_supp/tables/food_req}
		\label{tab:food_req}
	\end{table}
%------------------------------------------

\begin{table}[H]
\small %!htb keeps the table in this section before moving onto the next block of text
		\caption[Seasonal changes in the model] %This goes into  your list of tables
				{A summary of the changes in ungulate mortality, vulture food requirements and, foraging radius throughout the year and according to the periods considered.  
} 
		\input{pdp_supp/tables/season_change}
		\label{tab:season_change}
	\end{table}

%------------------------------------------

\begin{table}
\small %!htb keeps the table in this section before moving onto the next block of text
		\caption[Description of model parameters] %This goes into  your list of tables
				{Parameters and definitions used in the model.   
} 
		\input{pdp_supp/tables/param_desc}
		\label{tab:param_desc}
	\end{table}
%------------------------------------------


% PDP model
%\begin{figure}[h] %!htb keeps the figure in this section before moving onto the discussion
%	  \centering
%	  \includegraphics[keepaspectratio=true]{pdp_supp/figures/pdp_model.pdf}
%	    \caption[Full P Systems Model] %This is the label in table of contents
%	    {}%this is under the figure
%	  \label{fig:pdp_model}
%	\end{figure}
	
%------------------------------------------	
\newpage
\noindent {\textbf {PDP Model Components}} \\
Following \cite{colomer2013population}, there are four main components to the model:  1) A set of environments that are connected according to some prefixed relation, and which can be formally described by a network; 2) a membrane structure that provides the hierarchy among the different membranes that constitute the cell contained in each environment; 3) objects involved in the system under study (individuals, resources etc.) are represented by alphabetic characters; 4) A set of rules that specify the behaviour of the objects in the model and a set of rules for the model environments which define how individuals can move between the environments as well as generating values for the variables correlated between environments.

\newpage
\includepdf[pages={-}, ,scale=0.90]{pdp_supp/figures/pdp_model.pdf}
	\newpage

\newpage
\includepdf[pages={-}, ,scale=0.85]{pdp_supp/figures/paper.pdf}
	\newpage



\chapter{Other Publications}
\label{chap:other-papers}

%	\newpage
%\begin{figure}[ht]
%\centering
%\includegraphics[scale=0.85,trim = 35mm 20mm 20mm 20mm]{other-papers/paper1.pdf}
%\end{figure}
%	\newpage


	\newpage
\includepdf[pages={-}, ,scale=0.85]{other-papers/paper1.pdf}
	\newpage

\includepdf[pages={-}, ,scale=0.85]{other-papers/paper2.pdf}
	\newpage

\includepdf[pages={-}, ,scale=0.85]{other-papers/paper3.pdf}
	\newpage

\includepdf[pages={-}, ,scale=0.85]{other-papers/paper4.pdf}
	\newpage

\includepdf[pages={-}, ,scale=0.85]{other-papers/paper5.pdf}
	\newpage

%\begin{figure}[h]
%\includepdf[pages={1}]{other-papers/paper1.pdf}
%\end{figure}

	





\end{document}
