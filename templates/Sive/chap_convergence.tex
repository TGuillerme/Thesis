\chapter{Convergence among tenrecs and other small mammals}
\label{chap:convergence}


\section{Introduction}

\section{Methods}

%Phylogeny section from the original disparity paper; probably relevant when I go back to the convergence analyses

\subsection{Phylogeny} % I still have the problem of subsubsections not getting numbered in the main text
	Instead of basing my analyses on individual trees and assuming that their topologies are known without error \citep[e.g.][]{Ruta2013, Foth2012, Brusatte2008, Harmon2003} I used a distribution of 101 pruned phylogenies derived from the randomly resolved mammalian supertrees in \citep{Kuhn2011}. 
		% I used 101 because that was the number in the smaller Fritz file - I could change it to 100 instead if 101 sounds odd?

	Eight species (six \textit{Microgale} tenrecs and two golden moles) in my morphological data sets were not in the phylogenies. Phylogenetic relationships among the \textit{Microgale} have not been resolved more recently than the \citep{Kuhn2011} analysis, therefore I added the additional \textit{Microgale} species at random to the \textit{Microgale} genus within each phylogeny \citep{Revell2012}. I could not use the same approach to add the two missing golden mole species because they were the only representatives of their respective genera within my data. Therefore I randomly added these species to the common ancestral node (using the findMRCA function in phytools \citep{Revell2012}) of all golden moles within each phylogeny. Adding these extra species to the phylogenies created polytomies which I resolved arbitrarily using zero-length branches \citep{Paradis2004}. I calculated pairwise phylogenetic distances among species using the cophenetic function \citep{Team2014}. 

\subsection{Quantifying convergence}


\section{Results}

\section{Discussion}